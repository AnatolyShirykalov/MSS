\section{Лекция 12}
\begin{enumerate}
\item Продолжение задачи об обтекании сферы потоком идеальной несжимаемой жидкости. Распределение давлений по поверхности сферы. Сила, действующая со стороны жидкости. Парадокс Даламбера.
\item Движение сферы в жидкости, которая покоится на бесконечности.
\item Силы, действующая на сферу, двужищуюся с переменной скоростью. Присоединённая масса.
\end{enumerate}

Что мы в прошлый раз уже делали? Мы рассматривали задачу об обтекании сферы.
%\begin{figure}[H]
  
%\end{figure}
% рисунок 1
Постановка задачи такая (отдельно запишем $\ve v = \grad\phi$)
\[
\begin{cases}
  \Delta \phi = 0&\text{всюду вне сферы};\\
  \CP\phi n\Big|_{r=R}=0;\\
  \CP \phi x\Big|_{\infty} = v_0,&\CP\phi y\Big|_\infty=0.
\end{cases}
\]

Находим решение в виде $\phi = v_0 x + \frac{v_0 x R^3}{2r^3}$. Имеем $\phi\sim v_0 x$ при $r\to \infty$ 

Напишем, что на поверхности сферы $\frac v{v_0}=\frac32\sin\theta$. Точки $\theta=0,\pi$ "--- точки торможения, то есть $v=0$. В точке $\theta=\frac\pi2$ достигается $v_{\max} = \frac32v_\infty$.
% рисунок 2

Хотим найти силу, действующую на сферу. Для этого найдём распределение давления. Имеем всегда (не обязательно в идеальной жидкости
\[ \ve f = \Gint{\Sigma_{\text{сферы}}}\ve P_n\,d\sigma,\]
уже в идеальной жидкости
\[
  \ve P-n = -p\ve n,\quad \ve f = - \Gint {\Sigma_{\text{сферы}}}p\ve n\,d\sigma.
\]
Проекция силы, действующей на тело со стороны жидкости на направление скорости набегающего потока, называется силой сопротивления. Сила сопротивления в данной задаче
% рисунок 3
\[
  f_x = -\Gint{\Sigma_{\text{сферы}}} pn_x\,d\sigma
\]

Так как, $v_0=\const$, движение установившееся, не учитываются массовые силы, имеем интеграл Бернулли вдоль линии тока $A'ACBB'$
% рисунок 4 (старый новый рисунок)
\[
  \frac{v^2}2 + \frac p\rho = \frac{v_0^2}2+\frac{p_0}\rho.
\]
Отсюда видно, 
\[
  p = p_0+\frac{\rho v_0^2}2\left(1-\frac{v^2}{v_0^2}\right) = p_0 + \frac{\rho v_0^2}2\left(1-\frac94\sin^2\theta\right).
\]
Выводы из этой формулы.
\begin{roItems}
  \item При $\theta=0,\pi$ (в критических точках $A,B$) $p = p_0 + \frac{\rho v^2}2$.
  \item При $\theta=\frac\pi2$ имеем $p= p_{\min}=p_0 - \frac 58\rho v_0^2$.
  \item При $\theta = \arcsin\frac23$.
\end{roItems}
% рисунок 5

Теперь поговорим о кавитации. Формально мы можем получить нулевое давление и даже отрицательное. В жизни  даже нулевого давления не бывает, так как до наступления этого состояния жидкость успевает превратиться в пар.

Пусть $p_0 = 1\frac{\text{кГ}}{\text{см}^2} = 10^4\iZ[^2]{кГ}{м},\quad \rho = 102 \IZ{кГ с}2{м}4,\ p>0, p_{\min}=0,\ v_0 = \sqrt{\frac85\frac{p_0}\rho} = \sqrt{160}\iz{м}{с} = 45\iz{км}{ч},\ p = p_d = 0.013\text{Па} $

Теперь как вычислить полную силы. Вспомним $f_x = -\IE{сферы}{pn_x}$. Ещё раз нарисуем картинку
% рисунок 6
Тогда $n_x = \cos\theta$, а $d\sigma=2\pi R\sin\theta R\,d\theta$.
% рисунок 7
Когда я буду складывать симметричное давление в разных точках, то эта сила в проекции на ось $x$ получится равна нулю. Это вообще-то видно из вида давления 
\[
 p = p_0 + \frac{\rho v_0^2}2\left(1-\frac94\sin^2\theta\right).
\]
То есть сила сопротивления равна нулю. А мы понимаем, что чтобы человек, например, плыл, нужно грести. Такой результат нулевой силы называется парадоксом Даламбера ($f_y$, кстати, тоже оказывается равным нулю).

Вообще парадокс Даламбера для любого конечного тела это вот что.
% рисунок 8
Можно доказать, что \textbf{сила сопротивления} при обтекании любого конечного тела потоком несжимаемой идеальной жидкости при условии, что движение непрерывное и безотрывное, а скорость и давление в бесконечности выравнивается, \textbf{равна нулю}. Более того, парадокс Даламбера верен и для сжимаемой жидкости, но для баротропного дозвукового движения.

Давайте поговорим о том, как доказывается парадокс Даламбера.
% рисунок 9

Зачем нам его доказывать, раз он неверный? Но доказательство простое. Рассмотрим тело помещённое в трубу. Жидкость идеальная. Скорость набегающего потока $v_0$. Рассмотрим два сечения вдали от тела $\Sigma_1$ и $\Sigma_2$, то есть до и после. Боковые поверхности обозначим $\Sigma_{\text{бок}}$. Запишем закон сохранения количества движения для пространственного объёма $ABCD$ и ещё телом он ограничен. Подробно делать не будем, проговорим, что сколько через границу втекает, столько и вытекает, так как границы далеко от тела. Силы будем рассматривать в проекции на направление скорости. Получаются, что $f_x = 0$ (сила, действиующая на тело). Ну а когда труда расширяется до бесконечности, всё сохраняется. Такое доказательство есть в книге Седова.

Когда же сопротивление не равно нулю? За счёт чего.
\begin{enumerate}
  \item Вязкость (трение);
  \item За счёт того, что жидкость может отрываться от тела (течение не безотрывное). Разберу потом.
  \item Могут образовываться ударные волны (если среда сжимаема).
  \item Тело может быть неконечное, например, бесконечно длинное.
% рисунок 10
  \item Если есть свободные поверхности.
% рисунок 11
  \item Если тело движется с переменной скоростью или набегающий поток имеет переменную скорость.
\end{enumerate} 
Если это всё учесть, силу можно вычеслить. Она будет ненулевая.

Хочу прокомментировать пункт два, а потом вычислим силу из пункта шесть. Вот смотрите,
% рисунок 12
есть зоны повышенного давления, есть зоны пониженного давления. Имеется полная симметрия в симметричных точках. Это всё, если движение безотрывное.
% рисунок 13
Неважно, что так всё симметрично. Зоны повышенного и пониженного давления возникнут и на крыле.

Что такое развитая кавитация. Если струи срываются целой полосой, давление на той части получается не таким, как в симметричной точке.
% рисунок 14

А есть ещё отрывное течение из-за вязкости.
% рисунок 15
Жидкость разгоняется от большого давления к маленькому, а потом тормозится от маленького к большому до критической точки сзади. Если жидкость идеально, ей хватает кинетической энергии, чтобы не остановиться и получается безотрывное движение. Если есть вязкость, то жидкость доходит до малого давление, а область высокого давления сзади может уже и не преодолеть. И сзади получается вихревое движение, давления там другое. Возникает добавочное сопротивление. Что замечательно: был такой замечательный человек Прандтль, гидромеханик. Во время первой мировой войны, он сказал: «давайте приклеим проволочку, и снаряд полетит быстрее». Ведь дейтсвительно, движение становится турбулентным. Скачущие частицы разгоняют затормозившийся поток, отрыв получается меньше.
% рисунок 16

Расчитать точку отрыва и сделать её как можно далье "--- целая наука.

Вообще парадокс Даламбера это неправда, но опытом подтверждается. Вот в каком смысле.
% рисунок 17
Когда задний хвост острый и течение имеет в нём критическую точку, то движение безотрывное. И сопротивление в этом случае, действительно, почти нуль.

\subsection{Задача о движении сферы в жидкости, которая покоится на бесконечности}
Теперь поговорим про движение с переменной скоростью.
% рисунок 18

Если бы было обтекание, то потенциал был $\phi = v_0 x + \frac{ v_0 x R^3}{2r^3}$. Введём систему координат, которая движется вместе с жидкостью со скоростью $v_0$. Тогда на бесконечности скорость будет $0$.
\[
  \phi = v_0 x + \frac{v_0 x R^3}{2r^3} + (-v_0x) = .\frac{v_0 x R^3}{2r^3}.
\]
Если $v_0>0$ движется против оси $x$, а я хочу наоборот. Поэтому потенциал движения сферы будет такой
\[
  \phi = -\frac{v_0 x R^3}{2r^3}.
\]
Вот такой будет потенциал абсолютного движения (относительное движение в том случае, когда тело обтекается жидкостью) в подвижной системе координат, связанной со сферой. Дальше можно вычислить скорости, давления и прочее. Никто не сказал, что $v_0$ будет константа. Она может зависеть от времени $v_0(t)$. Одно но: координаты, в которых записан потенциал, $r$ "--- расстояние от центра сферы, то есть сфера же движется. Наши координаты подвижны.
% рисунок 19

Давление получим из интеграла Коши"--~Лагранжа. Координаты, связанные с пространством, относительно которого происходит движение, обозначим $X^i$. Как пишется интеграл
\[
  \CP\phi t\bigg|_{X^i=\const} t+\frac{v^2}2+\frac P\rho= f(t).
\]
Можно раз и навсегда вывести интеграл Коши"--~Лагранжа в подвижной системе координат. Так и сделаем.
\subsection{Интеграл Коши"--~Лагранжа в подвижной системе координат}
Пусть $X^i$ "--- координаты в неподвижной системе, $x^i$ "--- координаты в системе, связанной со сферой.
% рисунок 20
\[
  X^1 = x^1 + v_0 t;\qquad X^2 = x^2,\qquad X^3 = x^3.
\]
А в общем случае есть какие-то связи $X_i = X_i(x^k,t)$. Нужно вычислить $\CP\phi r\bigg|_{x^k=\const}$
\[
  \CP\phi t\bigg|_{x^k=\const} = \CP\phi{X^i}\DP{X^i}t\bigg|_{x^i=\const} + \CP\phi t\bigg|_{x^i=\const}.
\]
А частная производная
\[
\CP\phi t\bigg|_{x^i=\const} = \CP\phi t\bigg|_{X=\const} + v_iv^i_{\text{перенос}}.
\]
Что-то со знаками неправильно.

Итак, интеграл Коши"--~Лагранжа в подвижной системе координат записывается так
\[
  \phi = \phi(x,t),\quad \CP\phi t + \frac{v^2}2+\frac p\rho-v_iv^i_{\text{перенос}} = f(t).
\]
