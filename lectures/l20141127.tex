\section{Лекция 13}
Будет продолжение
\begin{enumerate}
  \item Движение сферы в идеальной несжимаемой жидкости. Сила, действующая со стороны жидкости на сферу, когда сфера движется с переменной скоростью. Присоединённая масса.
  \item Плоские (плоскопараллельные) движения несжимаемой жидкости. Функция тока, её механический смысл.
  \item Плоские потенциальные движения несжимаемой жидкости. Комплексный потенциал.
  \item Примеры комплексных потенциалов. (На самом деле примеры течений, конечно, которые они описывают.)
\end{enumerate}

\subsection{Движение сферы в идеальной несжимаемой жидкости}
Ну вот, надо постараться выполнить план. А то мы сильно опаздываем. Вторую лекцию подряд будем изучать движение сферы в идеальной жидкости.
% рисунок 1

Есть линии тока, которые идут прямо и упираются в сферу в точках, которые называются критическими. Частицы движутся вдоль сферы, разделившись на два потока, и в другой критической точке соединяются и идут на бесконечность.
% рисунок 2

Теперь движение сферы в пространстве. Частицы подтекают, чтобы не образовывался вакуум. Хотим вычислить силу, которая действует на сферу. Наверно, надо сразу сказать, что раз мы уже получили парадокс Даламбера. Если обтекание с постоянной скоростью на бесконечности, то сила равна нулю. И ясно, что в случае движения сферы будет то же самое. Если же сфера движется с ускорением, сила не будет равна нулю.

Пусть сфера движется со скоростью $v_0$
Не совсем прямым способом мы получили потенциал
\[\label{13potencial}
  \phi = - \frac{v_0 x R^3}{2r^3}.
\]
Это есть решение, если выполнены следующие условия $\Delta\phi = 0$ (это формулировка задачи, из которой $\phi$ находится), а ещё граничное условие на сфере: $\CP\phi{n}\Big|_{\text{на сфере}} = v_0\cos\theta$, и граничное условие на бесконечности $\CP\phi x\Big|_\infty = \CP\phi y\Big|_\infty = 0$.

Я не буду проверять, что потенциал этим условиям удовлетворяет и это есть решение задачи.

Отметим, что $v_0$ может быть как постоянная, так и переменная. Этот потенциал подходит и при $v_0=\const$ и ири $v_0=v_0(t)$.

\subsection{Сила, действующая со стороны жидкости на сферу, когда сфера движется с переменной скоростью}
Сейчас меня больше всего интересует сила. Линии тока, распределение скоростей "--- это меня не интересует. Высислим силу $\ve f_{\text{жидк}}$, действующую со стороны жидкости.
\[
  \ve f_{\text{жидк}} = \Gint{\Sigma_{\text{сф}}}\ve P_n\,d\sigma = - \Gint{\Sigma_{\text{сф}}} p\ve n\,d\sigma.
\]
Значит, $f_{x\text{жидк}} = -  \Gint{\Sigma_{\text{сф}}} pn_x\,d\sigma$. А какое $p$? Движение не установившееся, значит, не можем использовать интеграл Бернулли. Но есть ещё интеграл Коши"--~Лагранжа. Запишем его, не учитывая массовые силы, считая $\CP\phi t$ производной при постоянных пространственных координатах.
\[
  \CP\phi t + \frac{v^2}2 + \frac p\rho = f(t).
\]

В прошлый раз мы вводили обозначения $X^i$ ($X,Y,Z$) для пространственной системы координат, относительно которой рассматривается движение, а $x^i$ "--- координаты относительно сферы.
% рисунок 3
Будет соотношение $x = X + v_0(t)$. Мне хочется $\phi$ дифференцировать. Будем считать, что функция $\phi = \phi \big(X^i(x^k,t)\big)$. Поэтому
\[
  \CP\phi t\bigg|_{x^k=\const} = 
    \CP\phi{X^i}\DP{X^i}t\underbrace{\bigg|_{x=\const}}_{v^i_{\text{перенос}}} + 
    \underbrace{\CP\phi t\bigg|_{X=\const}}.
\]
Последнее слагаемое стоит в интеграл Коши"--~Лагранжа. Оно нам и интересно. Выражаем его и записываем интеграл
\[
  \CP\phi t\bigg|_{x^k=\const} - v_iv^i_{\text{пер}} + \frac{v^2}2 + \frac p\rho = f(t).
\]
Можно в более удобном варианте написать.
\[
  \CP{\phi(x,t)} t - v_iv^i_{\text{пер}} + \frac{v^2}2 + \frac p\rho = f(t).
\]
Что происходит на бесконечности: $\CP\phi t\to 0$, $v\to 0$, а $p\to p_0$ "--- давление в бесконечности. Значит, $f(x) = \frac{p_0}\rho$. И окончательная формула для давления будет такая.
\[
  p = p_0 - \frac{\rho v^2}2 + \rho v_x v_0 - \rho\CP\phi t.
\]
Теперь надо подставить в интеграл по сфере и вычислять. Но можно и упростить себе жизнь таким образом.
\[
  \CP\phi t = \DP{v_0}t\frac{xR^3}{2r^3}
\]
Если бы скорость была константа, то просто не было бы члена $\CP\phi t$. То есть на самом деле нужно вычислить только интеграл от последнего слагаемого, ведь при постоянной скорости полная сила ноль. Кто не верит, может посчитать всё, будет, конечно, ноль. Я этого делать не буду.
\[
  f_{x\text{жидк}} = - \IE{сф}{pn_x} = -\IE{сф}{\rho\CP\phi t n_x}.
\]
При этом $n_x = \cos\theta$, $x = R\cos\theta$, $d\sigma = R\,d\sigma 2\pi R\sin\theta$.
% рисунок 4 (их тут две)

Если всё подставить, что же мы получим.
\[
  f_{x\text{жидк}} = -\rho\DP{v_0}t\frac R2 2\pi R^2\int\limits_{\theta=0}^{\theta = \pi} \cos^2\theta\sin\theta\,d\theta 
  = -\frac23\pi R^3 \rho\DP{v_0}t.
\]

Получается, что сила проаорциональна ускорению $f_{x\text{жидк}} = -\mu\DP{v_0}t$, где $\mu = \frac23\pi R^3\rho = \frac12v_\rho$ "---половина массы жидкости в объёме.

Если сфера останавливается, она тормозит жидкость, а у жидкости есть интерция.
\subsubsection{Присоединённая масса}
Коэффициент $\mu$ называется присоединённой массой. Почему он такое странный. Когда сфера движется, на неё кроме силы со стороны жидкости действует сила $\ve f_{x\text{двиг}}$ (я её назову силой двигателя), которая её движет, со стороны двигателя, мотора.

Уравнение движения сферы выглядит таким образом
\[
  m a_x = f_{x\text{двиг}} + f_{x\text{жидк}} = f_{x\text{дв}} - \mu a_x.
\]

В итоге получается, что уравнение движения сферы в проекции на ось $x$ сейчас пока
\[
  (m+\mu) a_z = f_{x{\text{дв}}}.
\]
Жидкость движется так, как в пустоте, если бы её масса увеличилась на $\mu$. Поэтому $\mu$ называется присоединённой массой.

Если движение вдоль другой оси, то всё то же самое. Но если тело не является симметричным, то присоединённая масса зависит от направления. А если тело несимметричное и вращается (пока сфера крутится, ничего не происходит), от вращения появится добавочная присоединённая масса. Если всё это учесть, возникает целый тензор присоединённых масс $\lambda_{ik}$.
% рисунок 5 (нессиметричное тело)

Обтекание сферы бывает такое, как мы показали, но с увеличением скорости начинают срываться вихри. Оказывается, что в передней части распределение давления реальное близко к теоретическому. А сзади образуются отрывы жидкости. И нужно делать тела как можно более обтекаемые. Это, конечно, ещё в древности знали.
\subsection{Плоские задачи}
\begin{Def} 
  Течение называется плоскопараллельным, если существует такая плоскость, что скорости течения всех точек параллельны этой плоскости и, кроме того, и скорость, и все остальные параметры не зависят от расстояния до этой плоскости. Если эту плоскость назвать плоскостью $x,y$ (вернее оси координат в этой плоскости ввести), то для плоского движения $v_x = v_x(x,y,t)$, $v_y = v_y(x,y,t)$, $v_z = 0$.
% рисунок 6
\end{Def}
В каждой плоскости линии тока такие же, как и в основной плоскости.

\subsubsection{Примеры, когда такая модель применима}
Такое бывает приближённо. Большая колонна моста, есть дно, есть крыша. Для достаточно удалённых от границы плоскостей, всё происходит одинаково.
% рисунок 7

Ну или, например, самолёт.
% рисунок 8 самолёт
Если он имеет длинные крылья, то далеко от фюзеляжа во всех сечениях обтекание одинаковое.

Приток нефти в скважине в каждой плоскости всё происходит одинаково
% рисунок 9

Труба расширяется под давлениям. Можно рассчитать, что расширение во все стороны одинаково и во всех сечениях одинаково
% рисунок 10

А если взрывается, то уже нельзя применять плоские модели.
% рисунок 11 (взрыв трубы_

В лавина всё движется одинаково.
% рисунок 12 лавина и русло что ли

\subsubsection{Функция тока}
Теперь рассмотрим плоские течение несжимаемой жидкости. В этом случае можно ввести функцию тока. Мы имеем уравнение неразрывности $\div \ve v= 0$, можно записать $\CP vx + \CP vy = 0$ или $\CP{}x(v_x) = \CP{}y(-v_y)$. Отсюда можно записать точную дифференциальную форму $-v_y\,dx + v_x\,dy = d\psi$. Тогда $v_x = \CP\psi y$, а $v_y = -\CP\phi x$.

Такую функцию можно ввести, когда задача плоская, а жидкость несжимаема. Неважно, если ли вихри.

Мезанический смысл функции тока. На линии тока $\psi = \const$. Ведь что такое линия тока? Линия, определённая в фиксированный момент времени и такая, что направление касаельной к этой линии в каждой точке совпадает с направлением скорости.
% рисунок 13
Как написать уравнение линии тока: $\frac{dx}{v_x} = \frac{dy}{v_y}$. Теперь если напишем $\frac{dx}{\CP\psi y} = \frac{dy}{-\CP\psi x}$, то получим $-\CP\psi x\,dx - \CP\psi y\,dy = 0$. Вот поэтому она и навывается функцией тока.
% рисунок 14

Если имеется какая-то линия $A,B$, можно вычислить так называемый расход через линию $AB$, то есть $\psi(B)- \psi(A)$.
% рисунок 15 в пространстве
Вообще у нас всё в пространстве. Поверхность с образующей $AB$, единичной высоты. Расход через такую поверхности и есть расход через линию. А вообще это интеграл
\[
  \text{Расх} = Q = \int\limits_A^B v_n\,dl.
\]
Как показать, что это $\psi(B)- \psi(A)$. Можно выбрать $x$ по нормали на минуточку, а $y$ по касательной к линии. Тогда
\[
  Q = \int\limits_A^B\CP\psi l\,dl = \psi(B)-\psi(A).
\]

Теперь связи между $\psi$ и вектором вихря $\ve\w$. Оказывается, что в плоском течении всегда $\w_x=\w_y=0$, а $\w_z = \frac12\left(\CP {v_y}x - \CP{v_x}y\right) = -\frac12\Delta \psi$. Поэтому, если течение вихревое, то оператор Лапласа от функции тока не равен нулю.

\subsection{Плоское потенциальное течение несжимаемой жидкости}
В этом случае в силу потенциальности и из определения функции тока $v_x = \CP\phi x = \CP\psi y$, $v_y = \CP\phi y = -\CP\psi x$. Видим, что $\phi$ и $\psi$ связаны условиями Коши"--~Римана. Это значит, что можно ввести функцию $W = \phi + i\psi$ и это будет аналитической функцией комплексного переменного $x+iy = z$. Функция $W$ называется комплексным потенциалом.
