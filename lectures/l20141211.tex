\section{Лекция 15}
\begin{enumerate}
	\item Обтекание цилиндра идеальной несжимаемой жидкостью с циркуляцией. Подъёмная сила.
	\item Метод комформных отображений для решения задач об обтекании контура идеальной несжимаемой жидкостью.
	\item О величине циркуляции при обтекании контура с одной острой кромкой (постулат Жуковского"--~Чаплыгина).
	\item Потенциальное течение сжимаемой жидкостю. Полная система механических уравнений для баротропных процессов.
	\item Распространение малых возмущений в покоящейся среде. Линеаризация уравнений.
\end{enumerate}
Теперь можно начитать по существу. В прошлый раз занимались задачей об обтекании цилиндра потенциальным несжимаемым потоком.
% рисунок 1

На бесконечности скорость обозначали $v_0$, потенциал получился 
\[
	W = v_0\left( v_0+ z+\frac{a^2}z\right),\quad x = u+ iy,\quad v_0 = -2v_0\sin\theta.
\]	
Это решение называется безциркуляционным обтеканием цилиндра идеальной несжимаемой жидкостью.

По формуле Стокса циркуляция по любому контуру, не обхватывающему цилиндр, равен нулю. А если контур $C$ "--- окружность, внутри которой вложен цилиндр
\[
  \Gamma=\oint\limits_C v_l\,dl\oint\limits_Cv_0a\,d\theta = 0.
\]
Но, оказывается, решение задачи об обтекании цилиндра и других контуров не является единственным. Задача о нахождении $W$
\begin{enumerate}
	\item\label{obcyl1} $W$ "--- аналитическая функция в области, занятой жидкостью.
	\item\label{obcyl2} $\Im W = \const$ на контуре (условие непроницаемости).
	\item\label{obcyl3} $\DP wz = \ol v_0$.
\end{enumerate}
Этим условиям удовлетворяет много решений.
\[
	W = v_0\left(z+\frac{a^2}z\right)+\left( \frac\Gamma{2\pi i}\ln z \right),\quad \Gamma\in\R.
\]
Убедимся, что условия \eqref{obcyl1}--\eqref{obcyl2} выполнены.
\begin{enumerate}
	\item $W$ аналитична при $|z|\ge a$.
	\item При $r=a$ имеем $\Im W = v_0 y\left(1-\frac{a^2}{r^2}\right)-\frac\Gamma{2\pi}\ln r = -\frac\Gamma{2\pi}\ln a = \const$.
	\item $\DP Wz = v_0\left(1-\frac{a^2}{z^2}\right)+\frac\Gamma{2\pi i}\frac1z$. При $z\to\infty$ $\DP Wz\to v_0$.
\end{enumerate}
При этом $\Gamma_C = \oint\limits_C v_0 a\,d\theta = \Gamma$. То есть имеем обтекание с циркуляцией. Добавили точечный вихрь.
% рисунок 2

При $\Gamma=0$ $v(A)=v(B)=0$, $A,B$ "--- критические точки.
% рисунок 3

Ghb $\Gamma\ne0$ критические точки сместятся. При $\Gamma>0$ критические точки сместятся вверх. Линии тока должны приходить в критические точки под углом 90\,${}^\circ$, это мы обсуждали, когда рассматривали поток с потенциалом $\frac1{z^2}$. А на бесконечности линии тока горизонтальны.

Снизу скорости складываются или можно заметить, что частицам снизу нужно больший путь пройти. Так что получится сила, действующая вниз для $\Gamma>0$.
Посмотрим на это через интеграл Бернулли. Проверьте для точечного вихря добавок.
\[
	v_0 = -2v_0\sin\theta + \frac\Gamma{2\pi a},\quad
	v_0^2 = 4v_0^2\sin^2\theta + \frac{\Gamma^2}{4\pi^2a^2}-\frac{2\Gamma v_0\sin\theta}{\pi a},\quad
	p = p_0+\frac\rho2v_0^2-\frac p2v^2,
\]
если массовых сил нет. Силы Архимеда я не учитываю. $n_x = \cos\theta$, $n_y = \sin\theta$, $d\sigma = a\,d\theta$.
\[
	f_{x\text{жидк}} = -\int\limits_0^{2\pi}p\cos\theta a\,d\theta, \quad
	f_{y\text{жидк}} = -\int\limits_0^{2\pi}p\sin\theta a\,d\theta.
\]
Так как есть симметрия по оси $y$ в том смысле, что модули скорости одинаковы с двух сторон и давления одинаковы. И поэтому $f_x$ получится равной нулю. А по $y$ давайте попробуем посчитать. Интеграл от $p_0$ ничего не даст, так как нормали по окружности распределены симметрично. Второе слагаемое постоянно, оно тоже ничего не даст. Интегрировать нужно только последнее слагаемое из $v_0^2$ (от него и зависит $p$)
\[
	f_{y\text{жидк}} = -\int\limits_0^{2\pi}\left(\frac\rho2\frac{2\Gamma v_0\sin\theta}{\pi a}\right)\sin\theta a\,d\theta.
\]
Если проинтегрировать получится
\[
f_{y\text{жидк}} = -\rho v_0\Gamma
\]
Таким образом для известной скорости потока на бесконечности $v_0$ имеем формулу Кутта"--~Жуковского
\begin{equation}
	f_{\text{жидк}} = -i\rho v_0\Gamma.
	\label{<++>}
\end{equation}
$f_{\text{жидк}}\perp$ скорости набегающего потока называется подъёмной силой.
% рисунок 4 судно может двигаться, вместо паруса крутящий цилиндр


\subsection{Метод комформных отображений для решения задач об обтекании профилей идеальной несжимаемой жидкостью}
А зачем мы безотрывно обтекаем цилиндр на лекции. На практике безотрывно почти никогда не обтекается. Образуются срывы
% рисунок 5 
Но есть формы, которые обтекаются безотрывно. Формы торпед делает именно с этим соображением.

Чтобы обтекать произвольные контуры, можно применить так называемый метод комформных отображений, который сводит обтекание произвольного тела к обтеканию цилиндра. А почему не к плоскости сводить? По плоскости нельзя сделать циркуляцию.

Пусть есть замкнутый плоский контур $L$. Задача 1. найти аналитическую вне контура $L$ функцию $W(z)$
% рисунок 6
Можно всё это обобщать, а мы напишем, что ищем такую $W$, что $\Im W=\const$ на $L$ и $\DP Wz\big|_\infty = v_0$.

Пусть на плоскости $\zeta$ знаем аналитическую функцию $W_1(\zeta)$, для которой $\Im W_1\big|_{L_1} = \const$ и $\DP {W_1}\zeta = \frac1kv_0$. Пусть также мы знаем комформное отображение внешности $L_1$ на внешность $L$ и обратно
% рисунок 7
Известна теорема Римана о том, что такое отображение существует и единственно при дополнительных условиях $z_\infty \rightleftharpoons\zeta=\infty$, $\DP\zeta z = k$ "--- действительное число.

Если я подставлю $W_1\big(f(z)\big) = W(z)$, то получу решение задачи 1. Условие аналитичности выполнено, так как $\exists\ \DP {W_1}\zeta,\DP\zeta z, \DP Wz = \DP{W_1}\zeta\CP\zeta z$. Условие на контуре выполняется, так как граница обрасти комформности перейдёт в границу области комформности. 
% рисунок 8
Условие на бесконечности $\DP Wz\big|_\infty = \frac 1kv_0 k = v_0$.

Мы не просто доказали, что решение существует, мы даже формулу можем написать
\[
  W_1 = v_0\left(z+\frac{a^2}z\right) + \frac{\Gamma}{2\pi i}\ln\zeta,\quad \zeta = f(z).
\]

\subsection{Постулат Чаплыгина}
Как выбрать $\Gamma$?
% рисунок 9
При $\Gamma=0$ должна быть симметрия сил и особые точки располагаются симметрично. Но ясно, что такое обтекание никогда не выполняется, происходит срыв потока. Мы хотим безотрывное обтекание выбором циркуляции. Мы выберем так, чтобы задняя критическая точку сместилась перешла в острый конец профиля. На нашем рисунке критические точки сдвинутся вниз. Комформное отображение на даст нужное расположение критических точек на цилиндре. Нам нужно $\Gamma<0$ выбрать.

Чаплыгин говорит, что безотрывное движение осуществляется, если и только если задняя критическая точка является точкой излома контура.
\subsection{Потенциальное движение сжимаемой жидкости}
Предположим, что
\begin{enumerate}
	\item Жидкость идеальная.
	\item Массовые силы не учитываются.
	\item Движение безотрывное.
\end{enumerate}
Обычно течение потенциальное.

Запишем систему уравнений
\begin{eqnarray}
	\matder\rho + \rho\div\ve v &=& 0
	\label{nerazr};\\
	\CP\phi r+\frac{v^2}2+\mathcal P &=& f(t);\\
	\rho = rho(p).
\end{eqnarray}
Если учесть потенциальность в уравнении неразрывности, получим
\begin{eqnarray}
	\matder[\CP\phi ]\rho+\rho\Delta\phi &=&0;\\
	\CP\phi t +\frac12\left[\left(\CP\phi x\right)^2+\left(\CP\phi y\right)^2+\left(\CP\phi z\right)^2\right]+\mathcal P(p) &=&f(t);\\
	\rho &=&\rho(p).
\end{eqnarray}
\subsection{Распространение малых возмущений}
Никаких комплексных переменных, ничего сделать нельзя. Нелинейности, всё очень сложно. Но мы упростим себе задачу. Рассмотрим распространение малых возмущений, я хочу обхяснить, как распространяется звук.
Фон "--- это основное состояние. $v=0$, $p=p_0$, $\rho=\rho_0$ "--- такое состояние будем называть фоном. Введём малые возмущения: $v = \Til v$ "--- малая величина (их квадратами будем пренебрегать по сравнению с их первыми степенями), $p=p_0+\Til p$ "--- малая добавка, $\rho = \rho_0+\Til \rho$. Тогда
\begin{eqnarray}
	\CP{\Til \rho}t+\rho_0\div\ve v=0&\mapsto& \CP{\Til\rho}t +\rho_0\Delta \phi = 0;\\
	\CP\phi t+\mathcal P_0 +\frac{a^2_0}{\rho_0}+\frac{a^2_0}{\rho_0}\Til\rho = f(t)&\mapsto& \CP{\phi_1}t+\frac{a_0^2}{\rho_0}\Til\rho = 0.
\end{eqnarray}
Здесь $\phi_1 = \phi+\mathcal P_0 t-\int f(t)\,dt$. Это линеаризованная система. $\grad \phi_1 = \grad\phi = v$.
