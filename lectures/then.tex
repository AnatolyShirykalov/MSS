\section{16 октября 2014}
\begin{roItems}
\item Установившееся адиабатическое движение совершенного газа;
\item Интеграл Бернулли для адиабатического движения совершенного газа;
\item Параметры торможения. Максимальная скорость;
\item Число Маха. Другие формы интеграла Бернулли;
\item Примеры расчётов с помощью интеграла Бернулли.
\end{roItems}

Вчера прилетела из Таджикистана. Там было очень хорошо, но схватила грипп или что ещё. Попробуем с набранном тексте, а то я возможно не выстою писать у доски. Вообще на мех-мате считает плохим тоном 

В прошлой лекции был интеграл Бернулли для несжимаемой жидкости. А сегодня мы должны рассмотреть интеграл Бернулли для адиабатического движения совершенного газа. Если автомобиль едет со скоростью 250~км/ч, то можно считать, что он обтекается несжимаемой жидкостью.

\subsection{Установившееся адиабатическое движение совершенного газа}
Итак что же такое совершенный газ. Он удовлетворяет двум соотношениям.
\begin{equation}
  p = R\rho T;\qquad u = c_VT+\const.
\end{equation}
Теперь что такое абиабатическое движение: это движение, в котором отсутствуют притоки тепла к частицам, то есть $dq=0$ для каждой частицы, нет обмена телом даже между частицами.

Для абиабатического движения идеального совершенного газа из уравнения притока тепла можно вывести связь «адиабата Пуассона». Она имеет вид $p=A\rho^\gamma$, где $A$ есть функция энтропии $s$, $\gamma=\frac{c_P}{c_V}>1$. В данном случае в каждой частице $s=\frac{dq}T=0$ и $A=\const$. Тем не менее в разных частицах $A$ может быть разной и в точке пространства не является постоянной по времени.

\subsection{Интеграл Бернулли для адиабатического движения совершенного газа}
Теперь нужно писать интеграл Бернулли. Это соотношение, которое выполняется для линии тока при установившемся движении.
%рисунок 1
Если течение установившееся, то линии тока не меняются со временем, и частицы движутся именно по линиям тока, хоть и сложным образом (ускоряются, замедляются), но скорости все направлены по касательной.

У нас есть постоянная в каждой частице величина энтропия. Так вот эта энтропия будет постоянна вдоль линии тока, так как частицы движутся вдоль линии тока. На другой линии тока энтропия может быть и другая. Поэтому замечательно то, что хоть давление зависит от плотности и энтропии, но вдоль линии тока только от плотности.

Выведем интеграл Бернулли для адиабатического движения совершенного газа при отсутствии массовых сил. Общий вид:
\begin{equation}
  \frac{v^2}2 + P = C(L),\qquad P =\Gint L\frac1\rho\,dp,
\end{equation}
$C(L)$ "--- константа вдоль линии тока. Это соотношение выводится из уравнения Эйлера, лишь бы движение были установившееся, жидкость идеальная, то есть нет трения, массовые силы потенциальны.

Если бы была несжимаемая жидкость, всё было бы проще. А теперь нам нужно интегрировать, так как $\rho = \frac{p^{\frac1\gamma}}{A^{\frac1\gamma}}$.
Тогда
\[ P = \Gint L A^{\frac1\gamma}p^{-\frac1\gamma}\,dp = \frac{\gamma}{\gamma-1}A^{\frac1\gamma}p^{1-\frac1\gamma}.\]
И тогда интеграл Бернулли записывается в данном случае так:
\[ \frac{v^2}2 + \frac{\gamma}{\gamma-1}A^{\frac1\gamma}p^{1-\frac1\gamma} = C(L).\]

Можно получить и другие выражения для функции $P$ с помощью адиабаты Пуассона и уравнения состояния. Имеем
\[ P = \frac{\gamma}{\gamma-1}A^{\frac1\gamma}p^{1-\frac1\gamma} = \frac{\gamma}{\gamma-1}\frac p\rho = \frac{\gamma}{\gamma-1} A \rho^{\gamma-1} = c_PT = \frac{a^2}{\gamma-1}.\]
Здесь $a^2 = \gamma\frac p\rho = \gamma R T$ "--- это производная $\CP p\rho\Big|_{s=\const}$, оказывается (но мы этого пока не знаем) скоростью звука в газе. Формулу для распространения звука мы ещё только будем изучать.

Также мы использовали, что
$p = R\rho T$, $\gamma = \frac{c_P}{c_V}$, $R = c_P-c_V$, то есть $\gamma R = c_P(\gamma-1)$. Поэтому $\frac{\gamma}{\gamma-1}\frac p\rho = \frac{\gamma R}{\gamma-1}T = c_P T$.

Вторая форма интеграла Бернулли: $\frac{v^2}2+\frac{\gamma}{\gamma-1}\frac p\rho = C(L)$.

Третья форма: $\frac{v^2}2+c_PT = C(L)$.

Четвёртая $\frac{v^2}2+ \frac{\gamma}{\gamma-1}A\rho^{\gamma-1} = C(L)$.

Пятая форма: $\frac{v^2}2+\frac{a^2}{\gamma-1} = C(L)$.


Из первой формы видно, что чем больше скорость, меньше давление

Чем больше скорость, тем меньше температура, видно из третьей формы.

Четвёрная форма показывает, что плотность уменьшается при разгоне.

Пятая показывает, что скоростью звука уменьшается при разгоне.

\subsection{Параметры торможения. Максимальная скорость}
Параметры торможения "--- это значения $p_*,\rho_*,T_*, a_*$ величин $p,\rho,T, a$ в той точке линии тока, где $v=0$. Она называются давлением торможения, плотностью торможения, температурой торможения, скоростью звука торможения. При установившемся движении это максимальные значения на линии тока. Они вводятся, как величины параметров, которые получились бы, если поток адиабатички затормозить до нулевой скорости "--- это такое более общее определение (ведь может не быть точек с нулевой скоростью).

Теперь мы ешё введём понятие максимальной скорости. Из интеграла Бернулли видно, что при уменьшении температуры, скорость увеличивается на линии тока. Самое большое значени скорости на данной линии тока получилось бы, если бы давление обратилось бы в нуль. Если из сопла вытекание жидкости в вакуум, то скорость будет максимальная. То есть какие бы вы двигатели не ставили, больше чем $v_{\max}$ для данного $C$ скорость вы не получите.

Константу $C$ интеграла Бернулли можно выразить через параметры торможения и через максимальную скорость
\begin{equation}
c(L) = \frac{\gamma}{\gamma-1}\frac {p_0}{\rho_*}=c_PT_* =\frac{\gamma}{\gamma-1}A^{\frac1\gamma}p_*^{1-\frac1\gamma} = \frac{a_*^2}{\gamma-1} = \frac{v_{\max}^2}2.
\end{equation}
Эти формулы дают и связи между параметрами торможения и максимальной скоростью.

Запишем ещё формы интеграла Бернулли, где $C(L)$ мне надоело писать: $\frac{v^2}2 + c_P T = c_P T_*$ или $\frac{v^2}2+c_P T = \frac{v_{\max}^2}2$. Ну и разные другие формы.

\subsection{Число маха}
Число Маха $M$ "--- это соотношения величины скорости потока к скорости звука в рассматриваемом потоке. Очень важное число для установившегося движения. Если $M>1$, течение сверхзвуковое, $<1$ "--- дозвуковое. Как выразить интеграл Бернулли через число Маха.

Какое здесь есть рассуждения. Разделим все члены соотношения $\frac{v^2}2+P = C(L)$ на $P$.
\[\frac{v^2}{2P} + 1 = \frac{C(L)}P.\]
Для $P$ и $C$ можно использовать разные эквивалентные выражения. Мы будем в левой части $P$ выражать через число Маха, справа через давление, а $C$ через давление торможения. Левый член получится $\frac{\gamma-1} M^2$. Получим в итоге
\[
  1+\frac{\gamma-1}2M^2 = \left(\frac{p_*}p\right)^{\frac{\gamma-1}\gamma};\qquad \frac{P_*}p = \left(1+\frac{\gamma-1}2 M^2\right)^{\frac\gamma{\gamma-1}}.
\]

Аналогично можно получить и следующие формы интеграла Бернулли
\[
  \frac{\rho_*}\rho = \left(1+\frac{\gamma-1}2 M^2\right)^{\frac1{\gamma-1}},\qquad \frac{T_*}T = 1+\frac{\gamma-1}2 M^2;\qquad \frac{a_*}a=\left(1+\frac{\gamma-1}2M^2\right)^{\frac12}.
\]

\subsection{Примеры расчётов}
Пусть температура набегающего потока $T=288$\,K, $v=300$\,м/с. Нам нужно ещё знать теплоёмкость при постоянном давлении для воздуха $c_P\cong 10^3\,\frac{\text{м}^2}{\text{с}^2\text{К}^2}$. Скорость в точке $C$ равна нулю, поэтому $T_C = T_*$. Из интеграла в Бернулли $\frac{v^2}2+c_P T=c_PT_*$. Тогда $T_* = 333\,$К. При скорости $v = 200$\,км/ч, это ещё ничего. Тогда $T_C = T_* = T+1{,}8\text{К}=289{,}8\text{К}$. А при скорости набегающего потока $v = 2\,000\,$\,м/с $T = 2\,288$\,К. При таких температурах тело уже может начать гореть.

Есть такое замечание. Если сверхзвуковое течени, то оно такое как на рисунке. Образуется ударная волна, давление меняется скачком. Тут интеграл Бернулли годятся только перед ударной волны и после ударной волны, ведь мы его выводили из дифференциальных уравнений. Но оказывается, что если внимательно посмотреть на условия на границе разрыва, то константа интеграла Бернулли сохранится на линии тока, проходящей через границу разрыва.

%%Насадок сопло
Пусть газ вытекается через сопло из большого резервуара. Пусть дело происходит летом, то есть $T_* = 300$\,К, а $c_P = 1\,000\,\frac{\text{м}^2}{\text{с}^2\cdot \text{К}}$. Тогда $v_{\max} = \sqrt{2 c_P T_*} = 770\,\text{м}/\text{с}$ "--- такая вот непреодолимая в установившемся течении скорость. Теперь при $v=770\,\text{м}/\text{с}$ получим из интеграла Бернулли $T = 55$\,К. Раствор из этого резервуара вытекает, покрывая металлическую пластинку, помещённую вдоль оси такой струи. Это я своими глазами всё видела.

%% первая картинка в тетрадке
Поговорим немного про насадки. Вот имеется у меня трубка тока. Имеется какой-то контур замкнутый, через него проходят какие-то линии тока, мы следим за ними. Следим за площадью сечения и скоростью. Как они между собой связаны при установившемся движении. Пишем закон сохранения массы (или закон сохранения расхода это называется) $\rho v \Sigma = \const$. Для несжимаемой жидкости  $\rho=\const$, значит $v\Sigma=\const$ и $\frac{dv}{v} = -\frac{d\Sigma}\Sigma$. А если сжимаемая жидкость и установившееся движение. Тогда $v\,dv + \frac1\rho\,dp = 0$. Можно ввести сюда скорость звука $v\,dv + \frac{a^2}\rho\,d\rho = 0$, откуда видно, что $d\rho = -\frac{\rho v}{a^2}\,dv$. Результат получается такой: если течение сверхзвуковое, то поток будет разгоняться только при увеличении площади сечения. Если сопло сужается, то никогда нельзя получить сверхзвуковое течение.


\section{23 октября 2014}
\begin{enumerate}
  \item О форме трубок тока при установившемся движении несжимаемой и сжимаемой жидкости. Понятие о сопле Лаваля.
  \item Оценка влияния сжимаемости при установившемся движении газа.
  \item Потенциальные движения. Интеграл Коши"--~Лагранжа.
  \item Циркуляция скорости и формула Стокса.
  \item Теорема Томсона о циркуляции скорости.
\end{enumerate}
На самом деле теорема Томсона хоть и знаменита, она лишь частный случай теоремы Лагранжа, которую мы скорее всего не успеем.
\subsection{О форме трубок тока при установившемся движении несжимаемой и сжимаемой жидкости. Понятие о сопле Лаваля}
Рассмотрим трубку тока. Что это такое? Рассматривается установившееся движение. Что такое установившееся движение: в каждой точке области, занятой средой, со временем ничего не меняется. Всегда можно ввести понятие линий тока. Возьмём малый контур, через каждую его точку проведём линию тока, получим трубку.
%рис 1

Можно считать, что трубка составлена из линий тока, а можно считать ограниченной жёсткими стенками.

Считаем, что скорости в каждом сечении во всех точках сечения одинаковы и направлены по нормали к сечению. Но в разных сечениях, конечно, разные. Тогда рассмотрим, что будет, если сечение трубки увеличивается, уменьшается? Как ведёт себя среда при установившемся движении, если трубки расширяются и сужаются. В основном будем пользоваться законом сохранения масс. Мы пишем такое соотношение, если плотность в сечении $\Sigma(x)$ ($x$ вдоль трубки) обозначим $\rho$, скорость $v$
 \[
   \rho v\Sigma = \const.
 \]
$\rho v\Sigma$ называют расходом. Расход константа для трубки тока. А если дополнительно сделать предположения, получим больше.
\begin{roItems}
  \item Жидкость несжимаема, то есть $\rho =\const$, тогда и $v\Sigma = \const$. Формально это обозначает, что $\ln v + \ln\Sigma = \const$, можно продифференцировать:
\[
  \frac{dv}{v} = -\frac{d\Sigma}{\Sigma}.
\]
В сужающейся трубке скорость увеличивается, с расширяющейся уменьшается, если жидкость несжимаемая.
%рис 2
Здесь мы даже не пользовались интегралом Бернулли: $\frac{v^2}2+\frac{P}\rho = \const$, но он даёт те же выводы.

\item Сжимаемая жидкость. $\rho$ изменяется, но $\rho v\Sigma=\const$ и
\[
   \frac{d\rho}{\rho} + \frac{dv}{v}+\frac{d\Sigma}{\Sigma}=0.
\]
Введём дополнительные предположения.
\begin{enumerate}
  \item Жидкость идеальная;
  \item Массовые силы отсутствуют;
  \item Теплообмен отсутствует ($dq=0$).
\end{enumerate}
Тогда $p = p(\rho)$ и $\frac{v^2}2+\mathcal P=\const$, где $\mathcal P = \int\frac1\rho\,dp$. Следовательно,
\[
  u\,dv+\frac1\rho\,dp = 0;\qquad v\,dv+\frac1\rho\frac{dp}{\rho}\,d\rho=0.
\]
Введём обозначение $\frac{dp}{d\rho}\Big|_{s=\const} = a^2$ "--- это, как мы позже узнаем, скорость звука в среде. Тогда
\[
  v\,dv = -\frac{a^2}\rho\,d\rho.
\]
Я вынесу $\frac{dv}{v}$
\[
  \frac{dv}{v}\left(1-\frac{v^2}{a^2}\right) = -\frac{d\Sigma}{\Sigma}.
\]
Обозначают $M=\frac{v}a$ число Маха. Тогда
\[
  \frac{dv}{v}(1-M^2) = -\frac{d\Sigma}{\Sigma}.
\]
Если $M<1$, то поток называется дозвуковым. Он ведёт себя качественно как несжимаемая жидкость: в сужающейся трубке увеличение скорости, а расширяющейся трубке получается уменьшение скорости.

А вот если потом сверхзвуковой, то всё наоборот: в сужающейся трубке потом замедляется, а чтобы его разограть, нужна расширяющаяся трубка. Для чего нам нужно, чтобы на выходе была большая скорость. Эти трубки, эти насадки часто называются соплом. Для ракетных двигателей, например.
Чему равняется поток количества движения через сечение трубки? $\rho v^2 \Sigma$ и чтобы увеличить тягу, нужка большая скорость.

Так как скорость увеличить? Газ течёт за счёт разности давлений. Если давление уменьшаем, скорость увеличиваем. Но если газ движется из баллона, где скорость ноль, то при сужающемся сопле не будет никогда достигнута сверхзвуковая скорость. А если сопло будет расширятся после какого-то момента, то может быть достигнута максимальная скорость. Такое сопло (которое сначала сужается, потом расширяется) называется соплом Лаваля.
%рис 3
\end{roItems}
До Лаваля паровые турбины были неэффиктивны. Лаваль сделал их эффективными. Насадки в паровых турбин маленькие, на истребителе уже большие.

\subsection{Оценка влияния сжимаемости при установившемся движении газа}
Оказывается, что при малых скоростях и установившемся движении можно принебречь сжимаемостью. Какую мы при этом можем сделать ошибку?

Пусть 
\begin{enumerate}
\item жидкость идеальная;
\item движение установившееся;
\item массовые силы не учитываются.
\end{enumerate}
Тогда для несжимаемой жидкости сначала можно написать интеграл Бернулли
\[
  \frac{v^2}2+\frac{P}\rho = \frac{p_*}\rho,
\]
где $P_*$ "--- давление торможения (сейчас обычно обозначают $\rho_0$, но раз на прошлой лекции было со звёздочкой, пусть будет со свёздочкой). $v=0$, $p=p_*$, $\rho = \const=\rho_*$, тогда 
\[
  p = p_* - \frac{\rho v^2}2=P_*-\frac{\rho_*v^2}2.
\]
Теперь я хочу применить это уравнения для газа.

Добавочные предположения
\begin{enumerate}
\item газ совершенный;
\item движение адиабатическое.
\end{enumerate}
Тогда интеграл Бернулли
\[
  \frac{v^2}2 + \frac{\gamma}{\gamma-1}A^{\frac1\gamma}p^{1-\frac1\gamma} = \frac{a_*^2}{\gamma-1} = \frac{\gamma}{\gamma-1}A^{\frac1\gamma}P_*^{1-\frac1\gamma}.
\]
Я это перепишу
\[
  \frac{\gamma-1}2\frac{v^2}{a_*^2}+\left(\frac p{p_*}\right)^{\frac{\gamma-1}\gamma}=1.
\]
Далее $\frac p{p_*} = \left(1-\frac{\gamma-1}2\frac{v^2}{a_*^2}\right)^{\frac\gamma{\gamma-1}}$. Разложу по биному Ньютона, считая, что $\frac{v^2}{a_*^2}$ мал
\[
  p=p_*\left[
	     1-\frac{\gamma}2\frac{v^2}{a_*^2}+\frac{\frac\gamma{\gamma-1}\left(\frac\gamma{\gamma-1}-1\right)}{2!}\frac{(\gamma-1)^2}{4}\frac{v^4}{a_*^4}+\dots
 	\right]
  =p_*\left[
	    1-\frac{\gamma}2\frac{v^2}{a_*^2}\left(1-\frac14\frac{v^2}{a_*^2}+\ldots\right)
	\right].
\]
А что такое $a_*^2 = \left(\frac{dp}{d\rho}\right)_* = \gamma\frac{p_*}{\rho_*}$ в адиабатическом процессе в совершенном газе.
\[
  p=p_* - \frac{\rho_*v^2}2\left(1-\frac14\frac{v^2}{a_*^2}+\dots\right).
\]
Если я буду пользоваться относительной погрешностью, рассмотрим величину: отношение перепада давления к величине напора
\[
  \left|
    \frac{p-p_*}{\frac{\rho v^2}2}
  \right| = 1 - \frac14 \frac{v^2}{a_*^2}+\dots
\]
А в несжимаемой жидкости эта величина была равна единице. Значит, если мы хотим, чтобы относительная погрешность была порядка процента, то нужно
\[
  \frac14\frac{v^2}{a_*^2}\le 0{,}01\qquad \frac v{a_*}\le\frac15.
\]

Какие это числа? В нормальных условиях $a_* = 340\,\text{м}/\text{с}$, то $v< 68\,\text{м}/\text{с}\approx 245\,\text{км}/\text{ч}$.
Эту оценку мы слелали для того, чтобы вы не думали, что несжимаемая жидкость "--- это что-то несуществующее.

Но это всё для установившегося движения. Для неустановившегося нужно делать ещё оценки.

\subsection{Потенциальные движения. Интеграл Коши"--~Лагранжа}
Рассмотрим потенциальное движение, то есть существует такая функция $\phi$, что скорость $\ve v$ есть градиент $\ve v= \grad\phi$, $v_i = \CP\phi{x^i}$.

Если течение потенциально $\ve v= \grad \phi$, то течение безвихревое $\ve\omega = \frac12\rot\ve v = 0$. И обратное верно: $\ve\omega=0\imp \ve v= \grad\phi$. В этом случае есть ещё один интеграл, который и называется интегралом Коши"--~Лагранжа.

Пусть выполнены следующие условия:
\begin{enumerate}
\item жидкость идеальна (то есть верны уравнения Эйлера);
\item движение потенциально, то есть $\ve v=\grad\phi$;
\item движение баротропное, то есть $\rho = \rho(p)$ или $\rho=\const$;
\item массовые силы имеют потенциал, то есть $\ve F = \grad W$.
\end{enumerate}
Тогда из уравнений Эйлера выводится следующее соотношение
\begin{equation}\label{IntKL}
  \CP\phi t+\frac{v^2}2+\mathcal P-W = f(t),
\end{equation}
где $\mathcal P = \int\frac1\rho\,dp$ называется функцией давления, а $f(t)$ "--- произвольная функция времени, то есть не зависит от координат. Соотношение \eqref{IntKL} и называется интегралом Коши"--~Лагранжа. Вывел это соотношение тоже Эйлер, но Коши и Лагранж их использовали очень широко.

Как этот интеграл выводится? Берём уравнение Эйлера в форме Громеки"--~Лэмба (Lamb)
\[
  \frac{d\ve v}{dt}+\grad\frac{v^2}2+2[\ve\w\times\ve v] = \ve F - \frac1\rho\grad p.
\]
А так как $\ve = \grad\phi$, то $\frac{d\ve v}{t} = \grad\CP\phi t$. Пояснение
\[
  \CP{v_x}{t} = \CP{}T\left(\CP\phi x\right) = \CP{^2\phi}{t\dl x}= \CP{^2\phi}{x\dl t}.
\]
Далее так как  $\rho = \rho(p)$, то $\frac1\rho\grad p = \grad\mathcal P$. ($\mathcal P=\mathcal P(p)$ и $\CP{\mathcal P}x = \frac{d\mathcal P}{dp}\CP px = \frac1\rho\CP px$.)

Тогда уравнение Эйлера переписывается в виде
\[
  \grad\left(\CP\phi t+\frac{v^2}2-W+\mathcal P\right) = 0.
\]
А раз градиент равняется нулю, то выражение в скобках не зависит от $x,y,z$, но может зависеть от времени.
\[
\CP\phi t+\frac{v^2}2-W+\mathcal P = f(t).
\]
Чтобы найти $f(t)$, нужно знать левую часть выражения ровно в одной точке потока. Этот интеграл уже не на линии тока, а на всём потоке выполняется.

Но есть одно замечение. Оно очень серьёзно, ведь этот интеграл мы будем использовать постоянно. Замечание такое: можно всегда переопределить потенциал $\phi$ так, чтобы в интеграле Коши"--~Лагранжа справа стояла любая константа и, в частности, ноль. Как это сделать?
\[ \phi_1 = \phi - \int f(t)\,dt + C_1.\]
Скорость для этого потенцила будет та же самая, что и для $\phi$.

Мне сегодня четвёртый курс сдавал практикум. Там студенты сказали, что интеграл Бернулли выводится из интеграла Коши"--~Лагранжа. Это нехорошо.
\subsubsection{Различия между интегралом Бернулли и интегралом Коши"--~Лагранжа}
\begin{tabular}{|c|c|c|c|c|}
\hline
			 	& Движение уст. & Движение потенц. & Баротропия & Выпролняется где \\\hline
 интеграл Бернулли 		& обязательно		  & не обязательно	   & не обязательно& вдоль линии тока\\\hline
 интеграл Коши"--~Лагранжа 	& не обязательно	  & обязательно		   & обязательно& всюду в потоке\\\hline
\end{tabular}
Если движение установившееся и потенциальное, и баротропное, то интеграл Коши"--~Лагранжа по форме совпадает с интегралом Бернулли
\[
  \frac{v^2}{2}+\mathcal P- W = \const.
\]

\section{30 октября 2014}
Сегодня будет основная теорема о вихрях. Но у нас есть пункы, которые остались с прошлой лекции.
\begin{enumerate}
  \item Циркуляция скорости и формула Стокса.
  \item Теорема Томсона о циркуляции скорости.
  \item Теорема Лагранжа о сохранении свойства потенциальности движения.
  \item Причины возникновения вихрей.
  \item Граничные условия в задачах о движении идеальной жидкости.
  \item Постановки задач о потенциальном движении несжимаемой идеальной жидкости. Уравнения Лапласа для потенциала скорости и граничные условия.
\end{enumerate}

В прошлый раз мы вводили интеграл Коши"--~Лагранжа в случае, когда движение потенциальное. Этот интеграл в отличие от интеграла Бернули только для потенциальных движений. Остаётся вопрос, а когда же движение будет потенциальным.

Я напомню, что движение потенциально, если $\ve v= \grad \phi$. Это, конечно же, бывает далеко не всегда. Всё же это бывает довольно часто. Первая половина лекции посвящена этому вопросу.
\subsection{Циркуляция скорости и формула Стокса}
Прежде чем упоминать теоремы, надо упомянуть формулу Стокса. У неё есть хороший механический вывод.

Основное понятие здесь "--- это циркуляция скорости по замкнутому контуру $C$. Это вот такой интеграл.
%%%% рисунок 1
Поскольку контур в среде, могу в каждой точке нарисовать вектор скорости. Радиус-вектор точки контура будет $\ve r$, $d\ve r$ "--- вектор, который соединяет концы маленького отрезка контура. Нам нужно интегрировать скалярное произведение, которое можно записать по-разному. Например, через $\cos$, или как сумму произведений компонент.
\[
  \oint\limits_C (\ve v\cdot d\ve r) = \oint\limits_C (v_x\,dx + v_y\,dy+v_z\,dz)\equiv \Gamma_C.
\]

Почему циркуляция? Если жидкость движется по контуру замкнутому, то циркуляцие будет не нулевая, ясное дело. 
% рисунок 2
Но вообще не обязательно, чтобы что-нибудь циркулировало.
%% рисунок 3
Потом идёт только в одном направлении. Тогда $\Gamma_C=$ чему? Будет ли она равна $v_0\cdot BC$? Вот нет, она не будет, потому что циркуляция всегда определяется так, чтобы обход контура был против часовой стрелки. Скорее всего здесь будет минус $-v_0\cdot BC$.

Формула Стокса заключается в следующем.
\[
  \oint\limits_C (\ve v\cdot d\ve r)=2\Gint\Sigma\w_n\,d\sigma,
\]
где $\ve \w = \frac12\rot\ve v$ "--- вектор вихря, а $\w_n$ "--- это проекция вихря на нормаль к поверхности $\sigma$, причём нормаль эта должна быть направлена так, чтобы обход этого контура $C$ шёл против часовой стрелки.

Интеграл $\Gint\Sigma \w_n\,d\sigma$ называется потоком вектора вихря через поверхность $\sigma$. Если бы под интегралом была скорость, это был бы просто поток вещества.

Такая формула верна, если существует такая поверхность $\Sigma$, надетая на контур $C$, в каждой точке которой $\ve v $ и $\rot \ve v$ сущесвуют.
% рисунок 4
В каком-то смысле вращательное движение присутсвует при ненулевой циркуляции. Оно может быть спрятано деформациями.

Ещё одно замечание. Когда это бывает, что не существует скорости или её производных в какой-то точке? Это бывает, когда изучают плоскопараллельное движение. Формула Стокса может быть неприменима в теории плоских задач. Что это такое? Когда рассматривается очень длинное тело. В гидромеханике это, например, крыло самолёта
% рисунок 5
В области не примыкающей к фюзеляжу или к концу крыла, я могу рассечь крыло вертикальной плоскости. В сечении крыло будет иметь вот такую форму
% рисунок 6
Можем считать, что всё во всех плоскостях происходит одинаковым.
\begin{Def}
  Течение называется плоскопараллельным, если существует такая плоскость, что $\ve v\parallel$ этой плоскости и не зависит от расстояния до этой плоскости.
\end{Def}
Если возьму другие плоскости, то будет всё то же самое. То есть $v_x = v_x(x,y,t)$, $v_y = v_y(x,y,t)$, $v_z=0$ и другие параметры не зависят от $z$. Конечно, это упрощение. Но когда длинные объекты.

Тогда смотрите, что получается. Возьмё контур $C$, которое окружает профиль крыла, то формула Стокса неприменима, ведь скорость неопределена в точках профиля крыла. Возникает в безвихревом течении циркуляция. Она и даёт подъёмную силу.

\subsection{Теорема Томсона о циркуляции скорости}
Это тот самый Томсон, которому звание Лорда и дали имя Кельвин. Поэтому эта теорема встречается ещё как теоерма Кельвина.
\begin{The}[Теорема Томсона о циркуляции скорости]
  Пусть выполнены следующие условия:
\begin{enumerate}
  \item Жидкость идеальная (верны уравнения Эйлера);
  \item Массовые силы имеют потенциал $\ve F = \grad W$;
  \item Движение баротропное, то есть $\rho = \rho(p)$ или $\rho = \const$;
  \item Движение непрерывное\footnote{В частности мы под этим понимаем, что везде производные существуют.}, то есть нет поверхностей разрыва;
  \item Контур $C$ "--- так называемый «жидкий», то есть движется вместе с частицами жидкости;
\end{enumerate}
Тогда циркуляция скорости по этому контуру $C$ сохраняется по времени, то есть
\[
  \DP{}t\Gamma+C = 0.
\]
\end{The}

\begin{Proof}
 Состоит из двух частей.
\begin{enumerate}
  \item Сначала доказывается, что если $C$ "--- жидкий контур, то $\DP{}t\oint\limits_C(\ve v\cdot d\ve r) = \oint\limits_C\left(\DP{\ve v}t\cdot d\ve r\right)$, а $\DP{\ve v}t = \ve a$ "--- ускорение точек среды. Это обычно называется первой теоремой Кельвина.
  \item Затем доказывается, что при условиях 1--3 $\oint\limits_C (\ve a\cdot d\ve r) = 0$.
\end{enumerate}

Первая часть. Давайте попробуем доказать теорему. Нам надо продифференцировать
\[
  \DP{}t\oint\limits_C (\ve v\cdot d\ve r) ---?
\]
% рисунок 7
Конут $C$ подвижный, поэтому чтобы продифференцировать этот интеграл, переходим к лагранжевым координатам $\xi^i$. Так как лагранжевы кординаты частиц(это их фамилия, имя, отчество) на контуре не меняются, мы можем написать
\[
  \DP{}t\oint\limits_C (\ve v\cdot d\ve r) = \underbrace{\oint\limits_C\left(\CP{\ve v(t,\ve\xi)}t\cdot d\ve r\right)}_{No1} + \underbrace{\oint\limits_C\left(\ve v\cdot\CP{d\ve r}t\right)}_{No2}.
\]
А теперь смотрим: $N01 = \oint\limits_C (\ve a\cdot d\ve r)$, а в $No2$ имеем
\[
  d \ve r = \CP{\ve r}{\xi^i}\,d\xi^i,\quad \CP{d\ve r}{t} = \left(\CP{}t\frac{\ve r}{\xi^i}\right)\,d\xi^i = \CP{}{\xi^i}\left(\CP{\ve r}t\right)\,d\xi^i = \CP{\ve v}{\xi^i}\,d\xi^i = d\ve v.
\]
Таким образом $No2=\oint\limits_C(\ve v\cdot d\ve r) = \oint\limits_C d\frac{d^2}2 = 0$. И первая часть доказана. Нужно только, чтобы ускорение существовало.

Доказательство второй части теоремы Томсона. Берём уравнение Эйлера
\[
  \ve a = \ve F - \frac1\rho\grad p.
\]
И всё, это уравнение движения идеальной жидкости. Теперь учитываем предположения 2--3: $\ve F = \grad W$, а
\[
  \frac1\rho\grad p = \grad\mathcal P,
\]
где $\mathcal P = \int\frac1\rho\,dp$. Таким образом, при условиях 1--3 теоремы Томсона уравнения Эйлера имеют вид
\[
  \ve a = \grad (W-\mathcal P).
\]
Теперь рассмотрим циркуляцию
\[
  \oint\limits_C(\ve a\cdot d\ve r) = \oint\limits_C\big(\grad(W-\mathcal P)\cdot d\ve r\big) =
  \oint\limits_C\CP{W-\mathcal P}x\,dx+\CP{W-\mathcal P}y\,dy+\CP{W-\mathcal P}z\,dz = \oint\limits_Cd(W-\mathcal P) = 0.
\]
И теорема доказана.
\end{Proof}
\subsection{Теорема Лагранжа о сохранении свойства потенциальности движения}
Теперь давайте обсудим условия теоремы Томсона.
\begin{The}
 Пусть выполнены условия теоремы Томсона, то есть 
\begin{enumerate}
  \item Жидкость идеальная (верны уравнения Эйлера);
  \item Массовые силы имеют потенциал $\ve F = \grad W$;
  \item Движение баротропное, то есть $\rho = \rho(p)$ или $\rho = \const$;
  \item Движение непрерывное\footnote{В частности мы под этим понимаем, что везде производные существуют.}, то есть нет поверхностей разрыва.
 \end{enumerate}
Тогда если в какой-то массе жидкости в какой-то момент времени $t_0$ движение потенциально (то есть нет вихрей, $\ve\w=0$), то в этой массе жидкости движение было потенциальным раньше и будет потенциально в дальнейшем, то есть выполнены условия 1--4, вихри не могут возникнуть.
\end{The}
\begin{Proof}
 Я хочу показать, что не может быть $\ve \w\ne0$. 
% рисунок 8
От противного. Пусть в момент $t$ в какой-то точке $\ve \w\ne0$. Тогда, поскольку движение непрерывное, могу провести контур маленький в плоскости, перпендикулярной вектору $\w$. Можем применить формулу Стокса, получив, что поток вектора вихря отличен от нуля и циркуляция по контуру $C$ отличен от нуля. То есть существует контур $C$, хотя бы маленький, такой, что $\Gamma_C\ne0$.

Но это же та же самая жидкость. Значит, этот контур $C$ возник из какого-то контура $C_0$, он есть образ контура $C_0$. Тогда по теореме Томсона $\Gamma_{C_0} = \Gamma_C\ne0$. Но так как в момент времени $t_0$ по теореме Стокса $\Gamma_{C_0} = 0$, то и $\Gamma_C=0$. Получили противоречие. Ну и всё доказательство.
\end{Proof}

Значит, вихри не могут возникнуть.

Но ведь на самом деле вихри возникают. Над Исландией такое вихревое течение в атмосфере или вообще где-нибудь смерчи возникают. Из-за чего вихри возникают в атмосфере и в океане? Надо посмотреть на наши доказательства. На самом деле что мы можем сказать про циркуляцию ускорения вообще? Всегда можем написать уравнения движения для жидкости в таком виде
\[
  \ve a = \ve F - \frac 1\rho\grad p + \nabla_j\tau^{ij}\ve {\text{\textbf{э}}}_i
\]
Ну и имеем
\[
  \oint\limits_C(\ve a\cdot d\ve r) = \oint\limits_C(\ve F\cdot d\ve r) - \oint\limits_C\frac1\rho (\grad p\cdot d\ve r)+\oint\limits_C (\nabla_j\tau^{ij}\ve {\text{\textbf{э}}}_i).
\]
Видим, что все предположения важны. Причинами вихрей могут быть
\begin{enumerate}
  \item Массывые силы не потенциальны (первые интеграл не равен нулю);
  \item Нет баротропии;
  \item Действует вязкость.
\end{enumerate}

Когда силы могут быть не потенциальны? Например, на Земле иногда существенна сила Кориолиса, которая не является потенциальной.
% рисунок 9
На экваторе воздух сильно нагревается и поднямается, на полюсах опускается вниз. Возникает такое течение. И есть сила Кориолиса, которая действует перпендикулярно скорост. Возникают циклоны, антициклоны\ldotst

Баротропия. Тоже возникает в океанах и в атмосфере. В разных участках Земли температура разные. Возникают муссоны, бризы.

Действие вязкости. Если её предположить, то течение может лишь случайно оказаться потенциальным. Но это только случайно.

\subsection{Граничные условия в задачах о движении идеальной жидкости}
Какие бывают граничные условия в задачах движения идеальной жидкости? Границы бывают двух типов (в каком-то смысле даже трёх)\footnote{Бывают границы выделенными мысленно.}:
\begin{enumerate}
  \item Твёрдые границы: положения которых заданы\footnote{Например, твёрдое тело, движущееся в жидкости.};
  \item Свободные границы: положения и движения границы заранее не известны, нужно их найти\footnote{Например, поверхность моря (волны).}.
\end{enumerate}

Примёр твёрдой границы: дно озера.
% рисунок 10
В озере что-то там плывёт.

Или труба "--- это тоже твёрдые границы.

Условие на твёрдой границе в идеальной жидкости называется условим непроницаемости и формулируется так:
\[
 \ve v_n\big|_{\substack{\text{на тв.}\\\text{гран.}}} = \ve v_n\big|_{\text{гран.}};\qquad 
 \ve v_n\big|_{\substack{\text{на пов.}\\\text{тела}}} = \ve v_n\big|_{\text{тела}};
\]
% Рисунок 11
Скорости на границе тела должно совпадать. Если скорость поверхности тела меньше скорости жидкости, жидкость оторвётся. Это на самом деле и условие непроницаемости, и безотрывности. Второе слово «безотрывность» почему-то никогда не произносится.
% рисунок 12
Знаете, как движется жидкость? В одном месте тело жидкость выталкивает, с другой стороны, жидкость подтекает.

Если тело неподвижно, что граничное условие $\ve v_n\big|_{\substack{\text{на пов.}\\\text{тела}}} = 0$. Если вы изучаете обтекание автомобиля (поставили в аэродинамическую трубу), то граничное условие как раз такое.

Есть ещё свободные границы.
% рисунок 13
Есть твёрдая и свободная граница одновременно. Например, обтекание тела с капитационно полостью.

 Свободная граница "--- бак с неполтностью заполненный жидкостью и вы его везёте.

Ещё нефтянники говорят, что трубы, как правило, с не полным заполнением. Так может эта волна трубу заткнуть.

Все струи, например, струя реактивная.

Тут требуется больше больше граничных условий.

\section{6 ноября 2014}
Мы сегодня должны с вами лекцию 10 попробовать. Мы хотели заниматься граничными условиями. Мы их не разобрали, а они важны.

\begin{enumerate}
  \item Граничные условия на свободной поверхности в идеальной жидкости.
  \item Потенциальные течения несжимаемой идеальной жидкости. Уравнение Лапласа для потенциала скорости. Граничные условия.
  \item Задачи о нахождении потенциала скорости при отсутствии свободных поверхностей в качестве границ. Свойство линейности этих задач (важно именно это подчеркнуть).
  \item Примеры потенциальных течений незжимаемой жидкости.
\end{enumerate}

\subsection{Граничные условия на свободной поверхности в идеальной жидкости}
  Примеры твёрдых поверхностей: подводная лодка в воде, обтекание автомобиля, рыба-меч (скорость до 230 км/ч), примеры, касающиеся самолётов, аэродинамические трубы, обтекание зданий. Суть условий на твёрдой границы это жидкость не протекает через границу. Если жидкость протекает, поверхность называется пористой. Иногда специально отсасывают протекающую жидкость.

Второй тип поверхностей: форма и движение не заданы. Говорим мы про поверхности разрыва.
% рисунок 1
Разрыв чего? Скорости, плотности и ещё может быть чего. Свободный границы "--- это границы, через которые среда не перетекает. Все колебания связаны с тем, что сама жидкость ходит. Нет потока массы через границу. Движение этой границы нужно найти.

Примеры свободных поверхностей: поверхность волны (цунами, например, или корабельный волны), кавитация на торпеде Шквал,
% рисунок 2 (корабельные волны)
нефтепровод с неполным заполнением
% рисунок 3 (нефтепровод)
Если образуются волны, запирающие сечение, то течение может затормозиться.

Ещё пример это гидравлический прыжок. Справа идёт река, а слева начинается прилив из моря. И вода поступает из моря в реку и они сшибаются.
% рисунок 4 (гидр. прыжок)
 Прыжок распространяеится с довольно большой скоростью
% рисунок 5 (раковина)
Даже когда вы в раковине умываетесь, происхоит стоячий гидравлический прыжок.

Какие условия на свободный границе? Одного условия не хватает, нужно два. Одно называется кинематическое, другое "--- динамическое.
% рисунок 6 (
По одну сторону одна среда, по другую "--- другая. Условие сохранение массы говорит о том, что 
\[
  \ve v_n - D\big|_{\text{пов. }1} = (\ve n_2 - D); \ve v_n\big|_{\text{пов. разр.}} = \ve D_n.
\]
Это условие непроницаемости.

Кинематическое условие на свободной поверхности это как раз
\[ \ve v_n = \ve D_n,\]
где $\ve D_n$ "--- скорость поверхности.

Условие из закона сохранение количества движения
\[
  (\ve P_n)_1 = (\ve P_n)_2.
\]
Это называется динамическим граничным условием на свободной границе.
% рисунок 7

Тензон напряжений мы задать не можем. Это 9 компонент. Мы можем задать только вектор напряжений относительно нормали к границе.

Поскольку скорость границы мы не знаем, а нам хотелось бы форму границы вычислить. Поэтому делается так. Есть другая запись кинематического условия на свободной поверхности. Пусть уравнение свободной поверхность заранее неизвестно, но обозначим его $f(t,x,y,z)=0$. Условие, что $\ve v_n = \ve D_n$, означает, что 
% рисунок 8
частицы, которые находятся на поверхности, находятся на ней всегда, не могут с неё сойти (и по физике и по формальному), то есть для координат этих частиц всегда выполнено условие $f(t,x,y,z)=0$. Здесь уже $x,y,z$ "--- эйлеровы координаты точек частиц среды, лежащий на поверхности. Поэтому мы можем это соотношение продифференцировать по времени, считая, что $\DP xt = v_x$ и аналогично с $y,z$. Когда продифференцируем, получится кинематическое условие такое
\[
  \CP ft + v_x\CP fx+v_y\CP fy+v_z \CP fz = 0
\]
при $f(t,x,y,z)=0$. Или $\DP ft=0$ при $f(t,x,y,z)=0$.

Теперь это уравнение на $f$.

Пример. Если я рассматриваю волны на поверхности воды.
% рисунок 9 (волны на поверхности воды)
Есть дно и есть свободная поверхность. Уравнение поверхности будет такое $z = h(t,x,y)$, ну или можно написать $f = h(t,x,y) - z= 0$. Тогда кинематические условия пишутся вот так:
\[
  \CP ht + v_x \CP hx+v_y \CP hy - v_z = 0 \Leftarrow z=h(x,y,z).
\]
Или $v_z = \CP ht$ при $z = h(t,x,y)$ "--- кинематическое граничное условие на поверхности волны.

\subsubsection{Динамическое граничное условие в идеальной жидкости}
Теперь давайте динамическое условие, считая жидкость идеальной. Что мы делаем? Мы должны это $\ve P_n\big|_{\text{св. пов.}} = \ve P_n\big|_{\text{внеш.}} = \ve P_{na}$ (индекс ${}_a$ обозначает атмосферное) преобразовать с предположением $\ve P_n = -p\ve n$. Поэтому динамическое условие пишется, как
$ p = p_a$ на поверхности $f(t,x,y,z)= 0$.

В примере с волнами $p = p_a$ при $z = h(t,x,y)$.

Это условие можно переписать, как условие на скорость. Это можно сделать, например, если существует интеграл Коши"--~Лагранжа. Но пока я этого не буду делать.

Мы написали условия довольно сложные. Какая-то неизвестная скорость, всё нелинейное. Задачи со свободными границами неизмеримо сложнее чем задачи, где этих свободных границ нет.

Сейчас есть много пакетов для расчёта движения сред. Чтобы ими пользоваться, главное уметь правильно задать граничные условия. Тогда можно уже на кнопку нажать.

\subsection{Потенциальные течения несжимаемой идеальной жидкости. Уравнение Лапласа для потенциала скорости. Граничные условия}
Течения в океанах все вихревые. Плотность везде разная, всё достаточно трудно устроено. Поэтому сразу напижу, что жидкость несжимаемая.

Движение потенциально означает, что $\ve v = \grad\phi$ "--- уравнение потенциальности. Жидкость несжимаема, значит $\div\ve v = 0$ "--- уравнение неразрывности. Ну что это означает вместе взятое:
\[
  \CP{^2\phi}{x^2}+  \CP{^2\phi}{y^2}+  \CP{^2\phi}{z^2} = 0.
\]
Это уравнение называется уравнением Лапласа. А $\Delta = \CP{^2}{x^2}+\CP{^2}{y^2}+\CP{^2}{z^2}$ называется оператором Лапласа. Функция $\phi$, являющаяся решением, называется гармонической функцией.

Из этого уравнения можно найти $\phi$ и, соответственно, скорости. А другие величины уже будут находиться из других законов.

Теперь надо надо написать для этого уравнения Лапласа граничные условия. Тогда уже можно будет решать задачу. Коронной нашей задачей будет задача о движение сферы в нехжимаемой жидкости.
\subsubsection{Граничные условия для уравнения Лапласа в задачах о движении жидкости}
Сначала запишем условия на поверхности твёрдого тела.
% рисунок 10
\[
  \ve v_n\big|_{\text{пов. тв. тела}} = \ve v_{n\text{ тела}}.
\]

Если же $\ve v = \grad \phi$, то $v_x = \CP \phi x$ и~т.\,д. Тогда можно написать, что $\ve v_n = \CP\phi { n}$, где $\dl n$ "--- расстояние вдоль нормали. Я могу взять на минуточку ось $x$ в направлении нормали.
% рисунок 11
$ \CP\phi n = \yo {\Delta n}0 \frac{\phi(n_0+\Delta n) - \phi(n_0)}{\Delta n}$.

Просто вот такой вот факт $v_s = \CP\phi s$. Тогда условие на поверхности твёрдого тела в случае потенциального движения пишется так
\[
  \CP\phi n\bigg|_{\text{на пов. тела}} = v_n\big|_{\text{тела}}.
\]
Вот так задаётся ещё условие непроницаемости. Если нет землетрясений.
% рисунок 12
А что написать на свободной границе.

В Алма-Ате строили плотину методом взрыва. Ученики Лаврентьева расчитали, чтобы при взрыве все камни летели в одно место.

Давайте теперь напишем условие на свободной поверхности для потенциальной скорости. Можно ли их записать, как условие на $\phi$? Сначала кинематическое условие. Это легко
\[
  \CP ft + \CP\phi x\CP fx+\CP\phi y\CP fy+\CP\phi z\CP fz = 0 \Leftarrow f(t,x,y,z)=0.
\]
Неизвестные здесь уже теперь $\phi$ и $f$.

Теперь динамическое условие: $p\big|_{f=0} = p_a$. Можно записать, как условие на потенциал, если существует интеграл Коши"--~Лагранжа. А он существует? жидкость идеальна, движение баротропно, движение потенциальное, а потенциал массовых сил тоже существует, иначе бы движение тоже не было  бы потенциальным. Значит, интеграл существует. Осталось понять, как именно будет выглядеть динамическое условие на свободной границе.

Интеграл Коши"--~Лагранжа выполнен везде в жидкости:
\[
  \CP\phi t +\frac{v^2}2 + \frac p\rho-W= \frac{p_a}{\rho}.
\]
Справа вообще может стоять любая функция времени, поставим удобную нам константу. Это влияет на вид зависимости потенциала от времени, но не влияет на скорость. $\frac{p_a}{\rho}$ пишут тут довольно часто.

Тогда на поверхности $f=0$ условие $p=p_a$ и интеграл Коши"--~Лагранжа вместе дают
\[
  \CP\phi t + \frac12 \left[\left(\CP\phi x\right)^2+\left(\CP\phi y\right)^2+\left(\CP\phi z\right)^2\right] - W = 0\Leftarrow f=0.
\]
Это условие очень сложное. Поэтому книги по задачам о движении волн очень большие и сложные. Исследовать это дело очень сложно.

Задачи о схлопывании пузырьков (о разрушениях, ими наносимых), задачи о взрывах, задачи о струях.
\subsection{Задачи о потенциальном движении несжимаемой жидкости в случае отсутствия свободных поверхностей}
Будем считать, что
\begin{enumerate}
  \item Область, где движется жидкость, ограничена.
% рисунок 13
Бак в ракете. В Баке находится какое-то тело. Тогда уравнение $\Delta \phi = 0$, если на граниее заданы условия $\CP\phi n\big|_{\text{на гр.}}=0$, называется внутренней задачей Неймана.

Если же задана сама функция $\phi$, то имеем задачу Дирихле.

Уравнение Лапласа линейно и граничные условия тоже линейны. Значит, внутрення задача Неймана линейна. Раз задача линейная, то для неё есть много хороших методов решения.
  \item Область неограничена, то есть содержит бесконечно удалённую точку. Это идеализация, но оказывается, для этого уравнения её можно сделать.

Здесь задачи бывают двух типов. Например
\begin{enumerate}
  \item Тело движется в безграничной покоящейся жидкости. (Внешняя задача Неймана.)
% рисунок 14
Если нет земли, сплошной воздух, летит самолёт. Если я отойду от тела на расстояние порядка нескольких линейных размеров, то можно считать, что точка бесконечно удалена.

Итак задача: $\Delta \phi = 0$ всюду вне тела, а на границе тела меем $\CP \phi n\big|_{\text{на пов. тела}} = v_{n\text{ тела}}$, и на бесконечности задана скорость $\ve v=0$, то есть $\grad\phi\big|_\infty =0$. (Одно добавочное условие появилось, градиент на бесконечности.)
  \item Ещё бывает задача, где тело обтекается потоком, но это то же самое. Например, исследования в аэродинамической трубе. Или просто здания обтекаются потоком.
% рисунок 15
В этом случае на поверхности тела $\CP \phi n\big|_{\text{на пов. тела}} = 0$ и ещё $\grad\phi\big|_\infty = \ve v_\infty$ задано. Это тоже называется внешней задачей Неймана.
\end{enumerate}
\end{enumerate}
В общем, если нет свободный границ, то задача линейна.
\subsection{Примеры потенциальных течений несжимаемой жидкости}
Из сложения этих течений я буду потом получать более сложные.
\begin{enumerate}
  \item Поступательное движени вдоль оси $x$. Здесь $\phi = v_0 x$, $\CP \phi y = \CP\phi z = 0$, $\Delta \phi = 0$.
  \item Источник или сток в начале координат. Что это такое. Потенциал $\phi = - Q{4\pi r}$, где $r = \sqrt{x^2+y^2+z^2}$. Хочется это разобрать, как устроено течение, удовлетворяет ли это течение уравнению Лапласы.

Как поступим? Можно ввести наукообразие, ввести сферические координаты. Но здесь всё достаточно просто, будем работать в декартовых.
\[
  v_x = \frac{Q}{4\pi r^2}\frac{x}r = \frac{Qx}{4\pi r^3},\quad v_y = \frac Q{4\pi r^2}\frac yr,\quad v_z = \frac Q{4\pi r^2}\frac zr.
\]
Или, векторно
\[
  \ve v = \frac{Q}{4\pi r^2}\frac{\ve r}{r}.
\]
% рисунок 16 (или два рисунка, источник и сток)
Если $Q>0$, источник; $Q<0$ "--- сток.
\end{enumerate}
\section{13 ноября 2014}

\begin{enumerate}
  \item Примеры потенциальных потоков идеальной несжимаемой жидкости.
  \item Безотрывное обтекание сферы потенциальным потоком несжимаемой жидкости.
    Распределение скоростей и давление на поверхности сферы. Парадокс Даламбера"--~Эйлера.
  \item Движение сферы в безграничной с постоянной и переменной скоростью. Присоединённая масса сферы.
\end{enumerate}

Мы рассматриваем потенциальное движение несжимаемой жидкости. Значит, $\div\ve v = 0$ и $\ve v = \grad \phi$. Вместе уравнение Лапласа $\Delta \phi = \div\grad \phi 0$.

Потенциалы сложных движений будем получать, как суммы потенциалов простых.

Пример простого движения.
\begin{Exa}
  Поступательный поток со скоростью $\ve v_0(t)$ вдоль оси $x$. Потенциал $\phi = v_0(t) x$.
\end{Exa}

\begin{Exa}
  Источник или сток в начале
\[
  \phi = -\frac{Q(t)}{4\pi r},\quad r = \sqrt{x^2+y^2+z^2},\quad v_x = \CP\phi x = \frac Q{4\pi r^2}\frac xr,\dots \ve v = \frac{Q}{4\pi r^2}\frac{\ve r}r.
\]
Если $Q>0$, источник, если $Q<0$ "--- сток.
\end{Exa}
% рисунок 1 (их тут два) источник и сток

Удовлетворяет ли такое течение уравнению Лапласа? То есть является ли жидкость в примере несжимаемой? Можно было бы написать в сферических координатах. Тогда пришлось бы высчитывать символы Кристофелая. Мы не будем, давайте в этом простом случае проверим выполнения уравнения Лапласа в декартовых координатах.
\[
  \CP{^2\phi}{x^2} = -\frac{3 Q x^2}{4\pi r^5} + \frac Q{4\pi r^3},\quad   \CP{^2\phi}{y^2} = \dots, \CP{^2\phi}{z^2} = \dots\qquad \Delta \phi = -\frac{3 Q}{4\pi r^3} + \frac{3 Q}{4\pi r^3} = 0
\]
% рисунок 2 ещё раз про источник и сток
Например, если источник. Скорость направлена по радиусу.
% рисунок 3 физический смысл Q
Как посчитать, сколько протекает через поверхность за единицу времени. Надо взять маленькую площадку. Через время $t$ частицы этой площадки куда-то сместятся на
$v_R\,d\sigma\,dt$. Тогда $ \Gint \Sigma v_n\,d\sigma$ "--- количество жидкости, которая протекает через сферу $\Sigma$ за единицу времени. При этом $|v_n| = |v_r|$. Величина
\[
  \Gint{r=\const}\frac{|Q|}{4\pi r^2}\,d\sigma\frac{|Q|}{\cancel{4\pi r}}\cancel{4\pi r} = |Q|
\]
называется расходом источника или стока, мощностью или интенсивностью.


\begin{Exa}
  Сумма источника ($Q>0$ и источник положим в начале координат) и поступательного потока (обтекание полубесконечного тела).
% рисунок 4 источник и сток
Опять какое-то течение несжимаемой жидкости.
\[
  \phi = v_0 x- \frac Q{4\pi r},\qquad \Delta \phi = 0, r\to \infty,\phi\sim v_0 x\qquad r\to 0,\phi\sim -\frac Q{4\pi r}.
\]
\end{Exa}
Найдётся точка (см рис), где скорость поступательного потока равна скорости источника (источник как бы сдувает линии поступательного потока). И оказывается, что поток выглядит таким образом: он разделяется на часть, которая идёт от источника (это не линии на рис, это поверхности), поступательный поток идёт вне этой области. Найдё координаты точки $A$ и расстояние от $B$ до оси $x$. Это очень легко найти. Внутренность зоны $ABC$ могу заменить на твёрдое тело. Скорость на границе равна по касетельной "--- это такое же граничное условие, как для обтекания потоком твёрдого тела.

Будем считать, что решаем задачу обтекания ракеты.

Найдём $x_A$, то есть коодинату точки, где $v=0$. Так как $\phi$ известно, скорость можно получить, как
\[
  v_x = v_0 + \frac Q{4\pi r^2}\frac xr.
\]
Эта формула верна везде. Хочу искать точку на оси $x$, то есть $r = |x_A$ и $v_x = 0$. Получается для $x_A$ такая формула
\[
  x_A = -\frac{v_04\pi |x_A|^3}Q\imp |x_A| = \sqrt{\frac Q{4\pi v_0}}.
\]
Форма поверхности $ABC$ регулируется скоростью $v_0$ (обозначение для скорости поступательного потока на бесконечности).

Обозначим $R_\infty$ "--- радиус сечения обтекаемого тела в $\infty$.
Тогда расход через это поперечное сечение (в $\infty$) $v_0\pi R_\infty^2$, а с другой стороны он равняется $Q$. Получается формула
\[
  R_\infty = \sqrt{\frac Q{\pi v_0}}.
\]

\begin{Exa}
  Источник или сток не в начале координат, а в точке $x_0,y_0,z_0$. Ну это совсем просто
% рисунок 5

Нужно просто ввести систему координат $x' = x-x_0$, $y' = y-y_0$,$z' = z-z_0$. Тогда
\[
  \phi = -\frac Q{4\pi\sqrt{(x')^2 + (y')^2+(z')^2}} = -\frac Q{4\pi\sqrt{(x-x_0)^2 + (y-y_0)^2+(z-z_0)^2}}.
\]
\end{Exa}

\begin{Exa}
  Источник плюс сток с равными расходами на расстоянии $l$. 
% рисунок 6
 Сток поместим в начале координат, а источник на отрицательной полуоси $x$. Линии тока вытекают из источника и втекают в сток. Это всё тривиально. Как выглядит потенциал:
\[
  \phi = \frac Q{4\pi r(x,y,z)} - \frac Q{4\pi r(x+l,y,z)}.
\]
\end{Exa}

\begin{Exa} Диполь. Возьмём источник и сток на расстоянии $\Delta x$ и устремим $\Delta x\to 0$. Так просто мы ничего не получим. Но если ещё и $Q = \frac{M}x \to \infty$, где $M=\const$ (не зависит от $x$, зависит от времени).
 % рисунок 7
Какой получается предел:
\[
  \phi = - \yo {\Delta x}0\frac M{4\pi \Delta x}\left(
    \frac1{r(x+\Delta x,y,z)} - \frac1{r(x,y,z)}
  \right) = - \frac M{4\pi}\CP {}x\left(\frac1r\right).
\]
Это называется потенциалом диполя.
\end{Exa}
В каждый момент перехода к пределу уравнение Лапласа выполнено, значит, и в пределе выполнено. Или можно так рассуждать: если $\phi$ удовлетворяет уравнению Лапласа, то и её первая производная, вторая производная, миллионная производная тоже будут удовлетворять уравнению Лапласа. Функция, удовлетворяющая уравнению Лапласа, называется Гармонической.

Линии тока выглядят таким образом
% рисунок 8 линии тока диполя.

А теперь я хочу сделать ещё один пример.
\begin{Exa}
  Источник плюс сток плюс поступательный поток. Оказывается, вот что получается.
 % рисунок 9
 Вот есть источник, есть сток и вдали есть поступательный поток. Надо всё, что мы знаем сложить.
\[
  \phi = v_0 x + \frac Q{4\pi r(x,y,z)} - \frac Q{4\pi r(x+l,y,z)}
\]
\end{Exa}

Будут характерные точки $A$ и $B$, где скорости друг друга компенсируют. Через точки $A,B$ проходит поверхность, ограничивающая овальное тело. Либо линии тока внутри этого тела, либо вне.

\begin{Exa}
  Диполь плюс поступательный поток. Оказывается, что это и будет обтекание сферы. Мы это сейчас прямо докажем
 % рисунок 10
\end{Exa}

Есть такое метод источников и стоков. Не обязательно их ставить два. Их можно ставить много и получать разные тела.
% рисунок 11
Есть математическая теорема о том, что источниками и стоками можно получить любое осесимметричное тело. Если задано осесимметричное тело, то можно найти такое распределение источников и стоков на оси симметрии, чтобы получилось обтекание этого тела.

Несмотря на то, что потенциалы, нами рассмотренные, довольно простые, они решают достаточно серьёзные задачи.

\subsection{Задача о безотрывном обтекании сферы}
Рассмотрим безотрывное обтекание сферы потоком со скоростью на $\infty$, равной $v_0$. Жидкость будем считать несжимаемой, течение потенциальным.
% рисунок 12

Радиус сферы обозначим через $R$. $\Delta \phi = 0$ всюду вне сферы ($r>R$). Граничные условия на поверхности сферы: $\CP\phi n\Big|_{r=R}=0$.
И граничные условия на бесконечности: $\CP\phi x\Big|_\infty = v_0,\ \CP\phi y\Big|_\infty = 0$, если ось $x$ направлена по скорости набегающего потока.

Ищем решение в виде 
\[
  \phi = v_0 x - \frac{M}{4\pi}\CP{}x\left(\frac 1r\right) = v_0 x +\frac{Mx}{4\pi r^3}.
\]
У нас $v_0$ уже задано. Можно ли подобрать $M$, чтобы были выполнены граничные условия на сфере. Ведь уравнение Лапласа мы уже проверили для этого потенциала.

Итак, граничные условия на сфера:
\[
  \CP\phi n  \equiv\CP\phi r\overset?=0.
\]
Лучше всего ввести сферические координаты.
% рисунок 13 сферические координаты
Будем проводить сечения шар в рассматриваемой точке, перпендикулярные оси $x$. Это получаются окружности. Вводим угол $\theta$, как угол между осью $x$ и радиус-вектором рассматриваемой точки, $\alpha$ "--- угол на сечении, отсчитываемый от вертикали. Тогда
\[
 x = r\cos\theta,\ y = r\sin\theta\cos\alpha,\ z = r\sin \theta\sin\alpha.
\]
И теперь считаем производную потенциала по $r$ в точке $r=R$.
\[
  \phi  = \left(v_0 + \frac M{4\pi r^3}\right)r\cos\theta,\qquad
  \CP\phi r\bigg|_{r=R} = v_0\cos\theta - \frac{2M}{4\pi r^3}\cos\theta\bigg|_{r=R} = 0.
\]
Отсюда получается, что если $M = 2\pi R^3 v_0$, то $\CP\phi r = 0$ на $r = R$.

Итак, потенциал получается
\[
  \phi = v_0x + \frac{v_0 R^3 x}{2r^3} = v_0 x\left(1+\frac{R^3}{2r^3}\right) = v_0 r\left(1+\frac{R^3}{2r^3}\right)\cos\theta.
\]

Это всё пока формулы какие-то, а я хочу найти расределение скорости на поверхности сферы.
% рисунок 14

Всё будет симметрично относительно оси $x$. Нормальная составляющая скорости на границе равна нулю по граничному условию. Скорость направлена по касательной. Как скорость проще всего найти. Пусть $s$ "--- натуральный параметр от левого края в правый через верхнюю дугу. Тогда
\[
  v_s = \CP\phi s,\quad ds = - R d\theta;\qquad v_s\big|_{r=R} = -\frac1R\CP\phi\theta\bigg|_{r=R} = \frac32v_0\sin\theta.
\]
Что мы видим из этого распределения скорости? (Потом надо будет ещё давление получить.) Действительно есть критические точки и это $A,B$. А где скорость максимальна? В точках $C,D$. И что это за скорость: $v_{\max} = \frac32 v_0$.
