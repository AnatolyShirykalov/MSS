\section{Лекция Веденеева}

Маргарита Эрнестовна попросила заменить жидкости. Речь пойдёт о модели идеальной жидкости.

Известно, что реальные жидкости обладают вязкостью, но уравнения Навье"--~Стокса очень сложны. В реальности вязкие свойста проявляются очень слабо, вязкости очень часто не имеют значение. Поэтому модель идеальной жидкости "--- это некое приближение, но оно простое.

Часто движение жидкости является потенциальным, то есть $\ve v = \grad \phi$.

Какие здесь уравнения. Вы уже знаете: уравнение неразрывности и уравнение Эйлера
\begin{equation}\label{eq1}
  \begin{cases}
\frac{d\rho}{d t} +\rho\diV\ve v = 0;\\
\frac{d\ve v}{dt} = -\frac1\rho\grad p+\ve F.
\end{cases}
\end{equation}

\begin{Def}
  Линия тока "--- линия, котороая определяется для фиксированного момента времени и обладает тем свойством, что в любой точке направление касательной совпадает с направлением вектора $\ve v$.
\end{Def}
\begin{Def}
Траектория "--- путь одной индивидуальной частицы.
\end{Def}

Есть линии тока, есть траектории. Траектория от времени не зависит. Если течение установившееся, то траектории и линии тока совпадают.

Течение баротропно, если плотность есть однозначная функция давления. Например, изотермическое движение совершенного газа $p=\rho R T$, $T=T_0$. Другой пример "--- это несжимаемая жидкость. Движение так же баротропно.

Приведём уравнения движения к специальной форме, которая называется формой Громека"--~Лэмба. Для этого представим полную производную скорости в следующем виде.
\begin{equation}
\label{eq2}
\frac{d\ve v}{d t} = \CP{\ve v}{t}+\grad\left(\frac{ v^2}2\right) + 2\ve\omega\times \ve v,\qquad v^2 = v_1^2+v_2^2+v_3^2.
\end{equation}

Чтобы убедиться в правильности этого выражения, нужно посчитать такой определитель
\[\ve\omega = \frac12
\begin{vmatrix}
\CP{}x&v_1&\ve e_1\\
\CP{}y&v_2&\ve e_2\\
\CP{}z&v_3&\ve e_3\\
\end{vmatrix}
=\frac12\left(\CP{v_z}y-\CP{v_y}z\right)\ve e_1
+\frac12\left(\CP{v_x}z-\CP{v_z}x\right)\ve e_2
+\frac12\left(\CP{v_y}x-\CP{v_x}y\right)\ve e_3
\]

Теперь считаем векторное произведение
\[ \ve\w\times\ve v = 
\vprod \w v{\ve e}=(\w_2v_3 - \w_3v_2)\ve e_1+\ldots\]

Получаем производную в форме \eqref{eq2}
\[
  \frac{ d v_1}{dt} = 
  \setcounter{vars}{2}\matder{v_1} = 
  \CP{v_1}t+\CP{v_1}xv_1+\CP{v_2}xv_2+\CP{v_3}xv_3 + \left(\CP{v_1}z-\CP{v_3}x\right)v_3-\left(\CP{v_2}x-\CP{v_1}y\right)v_2  
\]

Само уравнение движения в форме Громеки"--~Лэмба имеет вид
\begin{equation}
\label{Gromeki-Lemb}
\CP{\ve v}t + \grad\left(\frac{v^2}2\right) + 2\ve\w\times\ve v = \ve F - \frac1\rho\grad p.
\end{equation}

\subsection{Интеграл Бернулли}

Докажем, что уравнение 
\eqref{Gromeki-Lemb}
имеют первый интеграл.


Будем выводить интеграл Бернулли.
Будем считать, что 
\begin{roItems}
 \item Жидкость идеальна;
 \item Движение установившееся $\CP{\ve v}t =0$.
 \item Массовые силы потенциальны $\ve F = \grad W$.
\end{roItems}
С учётом этих предположений уравнение \eqref{Gromeki-Lemb} имеем вид
\[
  \grad\left(\frac{v^2}2\right) + 2\ve \w\times\ve v = \grad W -\frac1\rho\grad p.
\]

Пусть $\ve l\in\R^3\colon |\ve l|=1$. Тогда $\CP{}{\ve l} = (\ve l,\grad\cdot)$. Умножим уравнение на направление $l$ и используем свойство производной по направлению.
Проектируем градиент на касательное направление ($|v| \ve l = \ve v$)
\[\CP{}{\ve l}\left(\frac{v^2}2\right)+\underbrace{2 \ve l\cdot[\ve \w\times\ve v]}_{0}+\frac1{\rho}\CP p{\ve l}-\CP W{\ve l}=0.\]

Вдоль линии тока у нас давление и плотность являются функциями линии тока $p = p(l)\iff l = l(p)$, $\rho=\rho(l)\iff l = l(\rho)$. На линии тока всегда можно посчитать интеграл $\mathcal P=\int\frac{dp}{\rho(p)}$. Делаем замену.
\[\CP{}{\ve l}\left(\frac{v^2}2+\mathcal P-W\right)=0.\]
Это соотношение можно проинтегрировать вдоль линии тока. Полученное и называется интегралом Бернулли.
\[\frac{v^2}2+\mathcal P-W=C(L).\]
Констатна зависит, вообще говоря, от линии тока.

Если летит самолёт, то все линии тока обтекают это тело и уходят в бесконечность, где если поток однородный, то все константы одинаковы (на бесконечности).

Пусть есть несжимаемая жидкость в поле силы тяжести. В этом случае $\mathcal P=\int\frac{dp}\rho=\frac1\rho\int dp=\frac p\rho$. В случае силы тяжести $F_x=F_y=0$, $F_z=-g$. Очевидно, что потенциал этого дела $W=-gz$.
А уравнение бернулли тогда принимает вид
\[
  \frac{v^2}2 + \frac p\rho + gz = C(l).
\]
Хочется сделать замечание. Разные частицы могут иметь разную плотность. Но на одной линии тока плотности одинаковы (ведь плотности не меняются на траекториях, а при установившемся движении траектории и линии тока совпадают). Таким образом, интеграл Бернулли верен и для неоднородного потока.

Интеграл Бернулли очень важен во многих физических приложениях. Что он означает: в тех точках, где скорость больше, давление меньше. Это позволяет сделать общий вывод о характере течения.

\subsection{Различные приложения интеграла Бернулли}
Начнём с понятия кавитации.

Давление не может падать бесконечно.
 Есть граница, после которой жидкость перестанет быть жидкостью, то есть $p>p_{\rus{н.п.}}$.
 Мы знаем, что при $T=100^\circ C$ давление насыщеных паров одна атмосфера $10^5$ Па. Для $T=20^\circ$ C давление насыщенных паров $p_{\rus{н.п.}}=0{,}023$ см $=23\cdot 10^2$ Па. Явление вскипания жидкости при определённом давлении называется кавитацией.
\begin{Def}
Кавитация "--- это образование пузырьков (возникновение полостей, заполненных парами жидкости и выделившимся из жидкости газом) за счёт понижения давления.
\end{Def}
%%%%картинка
Рассмотрим пример возникновения кавитации в трубе с пережатием.
Течение сначала сужается, затем расширяется. Будем считать, что труба практически горизонтальна и потенциал почти не меняется. Индекс $0$ будет в широком начале трубы, без индекса в узкой части. Площадь поперечного сечения $S_0$, но труба имеет местное сужение, в самом узком месте площадь $S_A$. Если считать, что массовые силы отсутствуют, а жидкость несжимаема, имеем интеграл Бернулли в виде
\[\frac{v_0^2}2+\frac{p_0}\rho = \frac{v^2}2+\frac p\rho.\]
Воспользуемся законом сохранения массы $v S = v_0 S_0$. Поставим отсюда $v = \frac{v_0\,S_0}S$ в интеграл Бернулли
\[\frac{v_0^2}2+\frac{p_0}\rho = \frac{v_0^2}2
  \left(\frac{S_0}S\right)^2+\frac p\rho.\]
Теперь отсюда выразим $p$.
\[ p = p_0+\frac{\rho v_0^2}2\left(1-\left(\frac{S_0}S\right)^2\right).\]
Видим, что если $S$ достаточно маленькая, то давление снизится насколько, что возникнет явление кавитации. Появятся пузырьки будут двигаться дальше, а дальше давление будет повышаться. Будет происходить схлопывание пузырька, что эквивалентно маленькому ядерному взрыву. Если смотреть схлопывание около стенок, то пузыри вырывают куски материала стенки. Это называется кавитационной эрозией. Это есть негативная сторона явления кавитации.

%%%%%%% Рисунок с обтеканием тела
Ещё один пример. Будем считать, что на бесконечности поток однородный. Рассмотрим линию тока, которая исходит из точки $A$, врезается в тело в точке $B$ обтекает его через точку $C$, отходит от тела в точке $D$ и утекает в точку $E$. Имеем $\frac{v^2}2+\frac p\rho=C$ для каждой точки.
\[\frac{v^2_\infty}2+\frac{p_\infty}\rho = \frac{v_C^2}2+\frac{p_C}\rho.\]
Часто если движение безразрывно (не отрывается от поверхности тела), то $v_C = 2v_\infty$ "--- это некий факт, который в лекциях будет доказан.

Если снова выражать, $p_c = p_\infty+\frac\rho2\left(v^2_\infty-v_c^2\right)=p_\infty - \frac32\rho v_\infty^2$.

Представте, что мы берём цилиндр и начинаем его двигать в воде. Увеличивая скорость, уменьшаем давление. Будут снова образовываться пузырьки и возникнет кавитационная эрозия. Это большая проблема для двигателей судов. Чем быстрее крутится винт, тем меньше давление. В начале двадцатого века это не представляло никаких проблем. Потом выявилось. Поставили новый винт на пассажирский корабль, полчаса проплыл и остановился. Водолазы посмотрели, а винта уже нет.

Есть и положительная сторона кавитации. Если пузырьков очень много и находятся они в локальной какой-то области, они превращаются в один большой паровой пузырь. Если он не схлопывается, то не представляет опасности для тела. Это называется развитая кавитация: из-за развитых пузырей снижается трение. Иногда специально делают воздушный пузырь, такой называется каверной, если используется специально для снижения трения. Где-то вдали каверна схлопывается, но это уже нас не волнует.
У торпеды Шквал на носу стоит каветатор, подаётся воздух. Сопротивление снижается до минимальных значений, что позволяет развить скорость 300--400 км/ч.

Ещё пример: липосакция. Если кто-то страдает ожирением, можно удалить жир с помощью ультразвуковой кавитации. Звуковые волны "--- это колебания давления. Где давление сильно отрицательное, появляется вскипание воды, образуются пузыри, затем пузыри схлопываются и разрушают жир.

\subsection{Трубка Пито"--~Прандтля}

Две трубки с продолжением наверх. Одна тонкая с отверстием $B$, другая шире с отверстием $A$.
\[\frac{p_A}\rho = \frac{p_B}\rho+\frac{v_B^2}2,\qquad v_B=\sqrt{\frac{2(p_A-p_B)}{\rho}}.\]
Если трубочка достаточно тонкая в $B$, то $v_B\sim v_\infty$. А разницу давлений мерием в вертикальной части трубок $P_A-P_B = g\Delta h\rho$. Можно просто глазом.

