\section{Лекция 14}
\begin{enumerate}
  \item Плоские потенциальные течения несжимаемой жидкости. Комплексный потенциал, комплексная скорость. Примеры комплексных потенциалов некоторых течений.
  \item Постановка об определении комплексного потенциала при обтекании тел идеальной несжимаемой жидкостью.
  \item Обтекание цилиндра (безциркуляционное  и с циркуляцией). Подъёмная сила.
  \item Метод комформных отображений при решении плоских задач об обтекании тел.
\end{enumerate}
\subsection{Примеры комплексных потенциалов}
Мы рассматриваем плоское движение. Это значит, что у нас есть только $v_x = v_x(x,y,t)$ и $v_y = v_y(x,y,t)$, время нам сейчас неважно, посколько, как мы увидим, время в решения будет входить как параметр. Жидкость несжимаема, значит, 
\[
  \CP{v_x}x + \CP{v_y}y = 0\imp v_x = \CP\psi y,\ v_y = -\CP\psi x,
\]
где $\psi$ "--- функция тока. Потенциальность же гарантирует $v_x = \CP\phi x$, $x_y = \CP\phi y$. Значит, $\Delta\phi \Delta \psi= 0$. Можно соорудить комплексно-аналитическую функцию $\phi + i\psi = W(z)$, где $z=x+i y$. Она аналитическая, так как выполнены условия Коши"--~Римана. $W(z)$ называют комплексным потенциалом.

Теперь я хочу посмотреть, чему равна производная от комплексного потенциала. Если функция $\C$-дифференцируема, то можно брать производную по любому направлению. Я сейчас возьму по направлению оси $x$.
% рисунок 1
\[
  \DP Wz = \CP\phi x+ i\CP\psi x = v_x - i v_y.
\]
Если я представлю скорость, как комплексное число, то есть $v = v_x + i v_y$, то получится
\[
  \DP Wz = v_x - i v_y = \ol v.
\]
Эта величина называется \textbf{комплексной скоростью.}

Любая дифференцируемая функция может быть представлена в виде потока на комплексной плоскости. Например, $W = az$ "--- поступательный поток вдоль оси $x$ со скоростью $a$, если $a\in\R$. Действительно, $W = ax + iay$, значит, $\phi = ax$, $\psi = ay$. Линиями тока являются линии $\psi = \const$, то есть $y=\const$.
% рисунок 2

А если я напишу так: $W = (a + bi)z$, ясно, что это будет тоже поступательный поток. $\DP Wz = a+bi = v_x - i v_y$. 
% Ниже рисунок для $a,b>0$
% рисунок 3
\subsubsection{Квадратичный потенциал}
Ещё пример $W = az^2$, где $a\in\R$. Значит, $W = a(x^2-y^2 + 2ixy)$. Функция тока $\psi = 2xy$ порождает линии тока $x=0$, $y=0$ либо $y = \frac1{2x}$.
% рисунок 4
При $z=0$ чему равняется скорость? $\DP Wz = 2az$, значит при $z=0$ скорость нулевая. Это критическая точка называется. Каким потокам это течение соответствует? Говорят, это поток внутри прямого угла. Я всегда в идеальной жидкости могу линию тока заменить на твёрдую стенку. Это может может быть представлено, как течение внутри прямого угла с нулевой скоростью в самом углу.
% рисунок 5
Или это может быть набеганием на плоскую стенку
% рисунок 6 (б)
Или это может быть набеганием на тело. Обтекание всегда выглядит таким образом, что есть линия тока разделяющая поток, входящая в критическую точку. В окрестности критической точки течение имеет как раз наш вид.
% рисунок 7 (в)
В малой окрестности критической точки раскладываю в ряд комплексный потенциал $W = a_0 + a_1 z + a_2 z^2 + \dots$ Тогда $\DP Wz = a_1 + 2a_2 z$. При $z=0$ скорость должна быть нулевая, если $z=0$ "--- критическая точка, значит, $a_1=0$. Таким образом, всегда будет угол девяносто градусов.
% рисунок 8

Что ещё тут может быть? Соударение двух потоков (г) и (д) "--- это течение в изогнутом канале (у которого стенки "--- гиперболы).

\subsubsection{Моном}
Пусть $W = a z^n$, где $a\in\R$. Тогда в тригонометрической форме $z = r e^{i\theta}$
\[
  W(z) = r^n(\cos n\theta + i \sin n\theta),\pau \psi = r^n\sin n\theta.
\]
Линии тока, например, $\psi = 0 = \sin n\theta$. $n\theta = k\pi$, $\theta= k\frac\pi n$.
% рисунок 9
Говорят, что это течение вне угла. При этом $\DP Wz = a n z^{n-1}$. Это может быть кусочком какой-то задачки. Обтекается какой-нибудь ёжик. И в окрестности любого угла будет такое течение. В вершине угла  всегда скорость ноль.
\subsubsection{Источник или сток}
% рисунок 10
Плоские задачи означают, что есть не одна точка-источник, а прямая. Задача о взрыве шнура. В каждой плоскости получается точечный источник. 
\[
  W = \frac{Q}{2\pi}\ln z = \frac Q{2\pi}(\ln r + i\theta).
\]
Линии тока $\psi = \frac Q{2\pi} \theta=\const\iff \theta = \const$. Можно вычислить скорость через полярные координаты $v_r = \CP\phi r = \frac Q{2\pi r}$.
% рисунок 11
Расход на линии $AB$ из окружности всегда равен $\psi(B) =\frac Q{2\pi}(\theta_b-\theta_a)$. Расход на всей окружности равен $Q$. Если $Q>0$ это называется источник, иначе сток.

Можно рассмотреть источник плюс поступательный поток.
% рисунок 12

Опять же метод источников и стоков применять для обтекания тел.

\subsubsection{Точечный вихрь}
$W = \frac{\Gamma}{2\pi i}\ln z$, где $\Gamma$ "--- действительное число. В тригонометрическом виде
\[
  W = \frac{\Gamma}{2\pi i}(\ln r + i\theta) = \frac{\Gamma}{2\pi}\theta - \frac{\Gamma}{2\pi}i\ln r,\quad
  \psi = -\frac\Gamma{2\pi}\ln r,\quad \phi = \frac\Gamma{2\pi}\theta.
\]
Линии тока в данном случае $\psi = \const$, то есть $r=\const$ "--- окружности.
% рисунок 13

Что значит $v_\theta = \frac1r\CP\psi\theta = \frac{\Gamma}{2\pi r}$? Если $\Gamma>0$, движение по часовой стрелки, $\Gamma<0$ "--- против часовой.
% рисунок 14 (два рисунка)
Скорость растёт всё ближе к нулю. Твёрдое тело крутится не так как жидкость. Хотя жидкость может крутиться и как твёрдое тело.

А почему потенциальное течение $W = \frac{\Gamma}{2\pi i}\ln z$ называется точечным вихрем? Вихрь равен везде нулю кроме начала координат. Вычислим циркуляцию скорости по окружности, обхватывающей начало координат.
\[
  \text{Ц} = \oint\limits_C \ve v \cdot \ve{dl} = \oint\limits_C (v_r\,dr + v_\theta t\,d\theta) = \oint\limits_C\frac\Gamma{2\pi}\,dt\theta = \Gamma,
\]
% рисунок 15 $dl = dr + ir\,d\theta$
если начало координат внутри контура. Если же начало координат лежит вне контура, то циркуляция будет равна нулю.
% рисунок 16

У нас есть формула Стокса
\[
  \Gamma_C = \oint\limits_C v_l\,dl =2\IE{}{\w_n},
\]
если сущесвует такая $\Sigma$, что во всех её точках $\w_n$ определена, то есть скорость дифференцируемая функция.
% рисунок 17

У нас $\Gamma\ne0$, так как скорость не дифференцируема. Циркуляция не зависит от формы контура вокруг начала координат.  Если в малой окрестности нуля заменить течение на непрерывное, то формула Стокса будет годиться. Тогда $\Gamma = \w_n\Delta\sigma$. Дальше мы радиус окружности устремим к нулю.
\subsubsection{Сток плюс вихрь}
$ W = \left(\frac Q{2\pi}+\frac\Gamma{2\pi i}\right) \ln z$. Ещё это называют вихрестоком.
% рисунок 18 (много рисунков)
Это можно рассматривать как модель торнадо. Давление на оси очень маленькое, туда будет всё засасывать.

\subsection{Задача об обтекании тела в терминах комплексного потенциала}
В терминах $\phi$ имеем следующую задачу
\begin{roItems}
  \item  $\Delta\phi = 0$ всюду вне контура тела;
  \item  $\CP\phi n=0$ на контуре;
  \item  $\CP\phi x\big|_{\infty} = v_{x\infty},\pau \CP\phi y\big|_{\infty} = v_{y\infty}.$
\end{roItems}

Теперь  то же самое в терминах комплескного потенциала $W$
\begin{roItems}
  \item \label{kp1} Найти $W$ "--- аналитическую всюду вне контура;
  \item \label{kp2} $\Im W = \const$ на контуре;
  \item \label{kp3} $\DP Wz\big|_\infty = \ol v_{\infty}$.
\end{roItems}

В терминах $W$ решать задачу проще, так как найти решение уравнения Лапласа так сходу непросто.
\subsection{Задача об обтекании цилиндра радиуса $a$ потоком несжимаемой жидкости}
% рисунок 19

Можно было и построить рещение, но я просто скажу ответ. 
\[
  W = v_\infty\left(z + \frac{a^2}z\right).
\]
Докажем, что выполнены условия \eqref{kp1}--\eqref{kp3} выполнены.
\begin{roItems}
  \item $\DP Wz = v_\infty\left(1 - \frac{a^2}{z^2}\right)$ "--- вне цилиндра. Функция удовлетворяет условиям Коши"--~Римана.
  \item $ W = v_\infty\left(x+i y + \frac{a^2}{r^2}x-i\frac{a^2}{r^2}y\right)$. Значит, $\psi = v_\infty\left(y-\frac{a^2}{r^2} y\right)$. Значит, линии тока есть либо $y=0$, либо $r=\const$. В частности $r=a$, то нам и нужно было.
% рисунок 20
  \item $\phi = v_\infty x\left(1+\frac{a^2}{r^2}\right)$. На поверхности тела 
\end{roItems}

Найдём распределение скоростей на поверхности $r=a$. $\phi = v_\infty\cos\theta\cdot 2 a$. $v_\theta = -2v_\infty\sin\theta$ на поверхности цилиндра $|v| = 2 v_\infty \sin\theta$.

Суммарная сила будет ноль. Парадокс Даламбера. Однако, решение не единственно.
