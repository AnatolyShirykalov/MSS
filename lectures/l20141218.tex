\section{Распространение малых возмущений в покоящейся сжимаемой идеальной жидкости или газе}
\begin{enumerate}
	\item Линеаризованная система уравнений;
	\item Волновые уравнения для скорости, плотности и давления;
	\item Одномерные движения с плоскими волнами.
		\begin{enumerate}
			\item Решение Даламбера. Физический смысл.
			\item Решение задачи Коши. Область влияния начальных данных и область зависимости решения.
			\item Условия Куранта.
			\item Связь между скоростью и плотностью в бегущей волне.
			\item Гармонические бегущие волны.
		\end{enumerate}
\end{enumerate}
Мы сейчас сделаем такие предположения о среде и движении.
\begin{enumerate}
	\item Жидкость идеальна, сжимаема; (если вязка, нужно сделать поправки на выводы;)
	\item Массовые силы учитывать не будем;
	\item Движение является баротропным. Это значит, что $\rho = \rho(p)$.
	\item Движение потенциально;
	\item Движение есть малое возмущение состояния покоя.
\end{enumerate}

Из условий 1--4 какая следует система уравнений (если условие 5 пока не учитывать)? Мы уже начинали её писать, ничего, напишем ещё раз. Уравнение неразрывности и интеграл Коши"--~Лагранжа
\begin{eqnarray}
	\matder\rho +\rho \div\ve v&=&0;\\
	\CP\phi t + \frac12(v_x^2+v_y^2+v_z^2) + \mathcal P &=& f(t).
\end{eqnarray}
Здесь $\ve v = \grad\phi$, $\mathcal P = \int\frac1\rho\,d p = \mathcal P$, $f(t)$ "--- произвольная функция $t$.

В случае общего положения эти уравнения сложно исследовать. Но мы будем изучать их при предположении 5. Основное состояние покой\footnote{Основным состоянием могло бы быть какое-то движение, оно ещё называется фоном.} значит $v=0$, $\rho = \rho_0$, $\mathcal P = \mathcal P_0$. Движение с малыми возмущениями значит $v$ мало, $\rho = \rho_0+\Til\rho$, где $\Til\rho$ мало; $p=p_0+\Til p$, где $\Til p$ мало. Все параметры равны фоновым плюс малые добавли. Добавки малы вместе со своими производными.

Понятие малости имеет смысл в безразмерных переменных. Нужно ввесли величины $\frac{\Til \rho}{\rho_0}$ и $\frac{\Til p}{p_0}$ малые. А $v$ с чем сравнивать? Мы сегодня ответим на этот вопрос.

Мы пока будем считать факт малости как возвожность не учитывать члены второго и более порядков.

Так что делается с исходной системой уравнений в случае малого возмущения покоя?
\begin{eqnarray}
	\CP{\Til \rho} t +\rho_0 \div\ve v&=&0;\\
	\CP\phi t + \mathcal P_0 + \Til{\mathcal P} &=& f(t).
\end{eqnarray}
Учтём, что $\ve v = \grad \phi$, мы будем на $\phi$ уравнение составлять. А ещё учтём, что $\mathcal P = \mathcal P_0 + \left(\CP{\mathcal P}\rho\DP p\rho\right)_{\rho = \rho_0}\Til\rho$. Величина $\CP{\mathcal P}p = \frac1{\rho_0}$, а $\DP p\rho = a_0^2$. Тогда получаем линейную систему следующего вида
\[
	\begin{cases}
		\CP{\Til\rho}t + \rho_0\Delta\phi = 0, &
		\CP{^2\phi}{t^2} = a_0^2\Delta\phi;\\
		\CP\phi t + \frac{a_0^2}{\rho_0}\Til\rho = 0,&
		\CP{^2\Til\rho}{t^2} = a_0^2\Delta\Til\rho.	
	\end{cases}
\]
У нас ведь $\Til \rho = -\frac{\rho_0}{a_0^2}\CP\phi t$. А для давления $p = p(\rho) = p_0+\underbrace{a_0^2\Til\rho}_{\Til p}$. Получаем три волновых уравнения (они называются волновыми, как вы знаете)
\[
	\begin{cases}
		\CP{^2\phi}{t^2} = a_0^2\Delta\phi;\\
		\CP{^2\Til\rho}{t^2} = a_0^2\Delta\Til\rho;\\
		\CP{^2\Til p}{t^2} = a_0^2\Delta p.
	\end{cases}
\]
Я не буду решать всё. Рассмотрим одномерные движения с плоскии волнами. Это значит, что в декартовой системе координат $v_y= v_z=0$, $v_z = v(x,t)$, $\rho = \rho(x,t)$, $p = p(x,t)$.
% рисунок 1
В каждом сечении, перпендикулярном оси $x$, всё происходит одинаково; вот, что означают слова «движение с плоскими волнами».
% рисунок 2
Бывает ещё одномерное движение с цилиндрическимми волнами, когда $v_r = v_r(r,t)$.
% рисунок 3
И одномерное движение со сферическими полнами, когда сфера пульсирует, $v_r(r,t)$.

У нас только одномерное движение с плоскими волнами. Уравнение $\CP{^2\phi}{t^2} = a_0^2\CP{^2\phi}{x^2}$. Решение вы знаете
\[
	\phi = f_1(x-a_0t) + f_2(x+a_0t).
\]
Это называется решением Даламбера.
Здесь $f_1,f_2$ произвольные. В конкретной задаче эти функции определяются из граничных условий. В задаче с цилиндрическими волнами такого не получится. Со сферическими и общими волнами тем более. 

Какой физический смысл этого решения Даламбера.
Рассмотрим $f_1(x-at)$. 
% рисунок 4

В какой-то момент $t=t_1$ можно нарисовать график функции $f_1(x - a_0 t_1)$, как функции от $x$. Обозначим некоторую точку $x_A$ и $f_A:=f_1(x_A - a_0t_1)$. В момент $t=t_2$ в точке $x$, для которой выполнено равенство $x_A - a_0 t_1 = x - a_2 t_2\iff x = x_A + a_0(t_2-t_1)$. В этой точке $f_1(x) = f_A$. Это означает, что состояние $f_A$ сдвигает по оси $x$ в точку $x$. Также сдвигается весь график. Волна "--- это перемещение состояния, а не перемещение точек. Это скорость продвижения фазы, скорость продвижения значения, без изменения формы. Говорят, что это бегущая волна. Форма не меняется, а график перемещается.

Можно рассмотреть функцию $f(x+a_0t)$, будет волна, бегущая влево. Говорят, что $a_0$ есть скорость малых возмущений. Когда я говорю, происходят очень малые позмущения давления, плотности. Поэтому $a_0$ ещё называют скорость звука $a_0 = \left(\sqrt{\DP p\rho}\right)_{\rho_0}$. Но на самом деле малые возмущения бывают разные. Зависит от того, какой происходит процесс. Я могу рассмотреть случай совершенного газа и $T = \const$. Тогда $ p = RT_0\rho$. Скорость звука будет $\sqrt{RT_0}$. А если считать, что звук распространяется адиабатически, то зависимость будет $p =  A\rho^\gamma$. Тогда будет $\left(\DP p\rho\right)_{\text{ад}} = \gamma A\rho^{\gamma -1} = \gamma\frac p\rho = \gamma RT$. Тогда $a_0 = \sqrt{\gamma R T_0}$. 

Так какая формула для скорости звука? Ньютон предложил $\sqrt{RT_0}$. Начинаем считать на практике, получается, что формула даёт заниженную скорость звука. А Лаплас рассуждал так: звук распространяется очень быстро, звук состояние частиц изменил и убежал, поэтому нет притока тепла. Формула Лапласа это как раз $a_0 = \sqrt{\gamma RT_0}$. Для воздуха $\gamma = 1.4$.

Несколько слов о решении задачи. Задачи могут быть, например, граничные. Есть у области границы. Она начинает колебаться. Может быть иначе, задача Коши. Может в начальный момент быть задано значение потенциала и значение производной по времени. Я хочу задачу Коши сформулировать не как математическую задачу, а с точки зрения механики спрлошных сред.

Для уравнения $\CP{^2\phi}{t^2} = a_0^2\CP{^2\phi}{x^2}$ задача Коши имеет следующий вид. При $t=0\pau \phi = F(x)$, $\CP\phi t = \Phi(x)$. Что это значит с точки зрения механики? Какой смысл в задании $\phi$ и $\CP\phi t$.
Мы задаём $v =V(x)$ в начальный момент и плотность $\Til\rho = \rho(x) = \mathcal R(x)$. Но ведь $v = \CP\phi x$, а $\phi = \int V(x)\,dx = F(x)$. А производную я откуда задам? Всегда, не только в начальным момент, выполнено $\Til \rho = -\frac{\rho_0}{a_0^2}\CP\phi t$.
Получаю, что в начальный момент $\CP \phi t = -\frac{a_0}{\rho_0}\mathcal R(x)$.

Вот мы имеем $\phi = f_1(x-a_0t) + f_2(x+a_0t)$. При $t=0 $ имеем 
\[\begin{cases}
	f_1(x) + f_2(x) = F(x);\\
	-a_0 f'_1(x) + a_0f'_2(x) = \Phi(x).
	\end{cases}
\]
Отсюда находим $f_1(x)$ и $f_2(x)$, и, соответствнно, решение.

Теперь немного поговорим про область влияния начальных данных. Это очень важно при построении разных решений. Это область на плоскости $(x,t)$. Всё движение у нас одномерное.
% рисунок 5

Пусть есть некоторая волна между $x_I$ и $X_{II}$ в начальный момент. Через некоторое время получится две волны. Они будут перемещаться по полосами. В середине между полосами ничего не будет, если вне $(x_I,x_{II})$ ничего не было. Если в этом интервале изменить начальные условия, то в двух полосах через некоторое время что-то изменится. В других областях не изменится ничего. Только наблюдатель из полосы заметит изменения. На границе таких областей могу быть склеены совершенно разные, они друг на друга никак не влияют. Это область влияния.

А теперь область зависимости решения.
% рисунок 6

Берём точку из $(x,t)$, от чего зависит состояние в этой точке, от начальных условий на какой области? Нужно провести через точку $M$ две прямые. Проекция $x_M$ зажата между $x_I$ (пересечение первой прямой с положительным наклоном $\frac1{a_0}$) и $x_{II}$. В $(x_{I},x_{II})$ условия будут влиять на точку $M$. А то, что задано справа или слева не влияет на $M$.

Это всё очень важно в численных методах. Какие там методы. Чуть-чуть покажу метод конечных разностей. Производные заменяются $\CP vx\sim\frac{\Delta v}{\Delta x}$, $\CP vt\sim\frac{\Delta v}{\Delta t}$.
% рисунок 7

Всегда хочется брать по времени как можно больше шаги, чтобы быстрее считать. Если я выбрала $\Delta x$, то $\Delta t$ нужно брать только $\Delta t<\frac{\Delta x}a$
% рисунок 8
Это условие называется условием К\'{у}ранта.

Поговорим о связи между $\Til v$ и $\Til \rho$ в волне, бегущей в одну сторону. Например, пусть $\phi = f(x-a_0t)$.
Как же эту связь получить? Зачем мне эта связь? Я хочу показать, какое условие нужно выставлять на $v$, чтобы можно было пользоваться линеаризацией.
Получается оно очень просто.
$v = \CP\phi x = f'_1(x-a_0t);\ 
\Til \rho = -\frac{\rho_0}{a_0^2}\CP\phi t = \frac{\rho_0}{a_0}f'_1(x-a_0t)$. Получается связь, что $\Til \rho = \frac{\rho_0}{a_0} v$, а $\frac{\Til\rho}{\rho_0} = \frac{v}{a_0}$. Таким образом, условие малости возмущений плотности $\frac{\Til\rho}{\rho_0}\ll1$ означает, что $\frac v{a_0}\ll1$.

Ну ещё хотела несколько слов про гармонические волны и закончить. Общая философия такая, что на самом деле решения волнового уравнения можно часто искать в виде произведений рядов Фурье. Механика и плоха (труднее) и хороша по сравнению с математикой.
Механики не могут ввести определение и для полученного объекта доказать свойства. Мы, механики, не можем сказать, что звук "--- это радуга, нам нужен физический смысл. С другой стороны разрывы, которые часто используются в математических моделях, "--- это идеализация. На самом деле всё всегда непрерывно меняется. Другое дело, что изменения часто бывает очень резким.

Малые возмущения "--- это некие синусоиды. Часто достаточно изучить поведение решений в виде
\[
	\phi = A\sin(kx-\w t); \quad\text{или}\quad A e^{i(kx-\w t)}.
\]
Если решения растут, то есть неустойчивость, и так далее. Можно такие решения исседовать. Они многое покзывают.

Пусть $\phi =  A\sin k\left(x - \frac\w kt\right)$. Значит, $a = \frac\w k$. При $t=\const$ имеем $A\sin(kx +\const)$. $k$ называется волновым числом, а $\lambda = \frac{2\pi}k$ длинной волны. Ещё можно рассмотреть период по времени. Нужно $x$ завиксировать, получим функцию времени $A\sin(\const - \w t)$. Период по времени есть $T = \frac{2\pi}\w$.

