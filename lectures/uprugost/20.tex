\section{Лекция 22}
\begin{enumerate}
\item Плоские задачи теории упругости. Примеры использования функции Эри.
\item Теорема Клапейрона.
\item Теорема единственности статических задач линейной теории упругости.
\item Волны в неогграниченной линейно-упругой среде. Плоские волны. Продольные и поперечные волны.
\end{enumerate}
У нас сегодня сборная солянка.

Что мы сейчас рассматриваем. Плоское деформированное состояние. Что это такое: когда существует такая плоскость $(x,y)$, такая, что $\e_{11},\e_{12},\e_{22}$ зависят только от $x,y,t$, $\e_{13} =\e_{23}=0$, $\e_{33}=0$. Ну и напряжения $P_{11},P_{12},P_{22}$ зависят от $x,y,t$, $P_{13} = P_{23}=0$, $P_{33} = \sigma (P_{11} + P_{22})$ "--- функция $x,y,t$.
% рисунок

Обобщённое плоское напряжённое состояние вводится наоборот очень тонкая пластинка.
%ещё картинка
Воодятся средние по толщине пластинки величины $P^*_{11} = \frac1{2\pi}\int\limits_{-h}^h$. $ P_{11}\,dz$. $P_{12}^*,P_{11}^*,P_{22}^*$ зависят только $x,y,t$, $P_{13}^* = P_{23}^*=0$, $P_{33}^* = 0$ (с точностью до малых порядка $h^2)$. Также $\e^*_{12},\e_{11}^*,\e_{22}^*$ зависят от $x,y,t$, $\e_{22}^* = \e_{13}^*=0$; $\e_{33}^* = \lambda(P_{11}^* + P_{22}^*)$ "--- функция от $x,y,t$.

И в плоском деформированном состоянии, и в плоском обобщённом напряжении при равновесии в отсутствие массовых сил можно ввести функцию Эри $\Psi$, для которой (звёздочку опускаем, но имеем в виду)
\[
  P_{11} = \CP{^2\Psi}{y^2},\quad
  P_{22} = \CP{^2\Psi}{x^2},\quad
  P_{12} = -\CP{^2\Psi}{x\dl y}.
\]
Тогда для любой $\Psi$ уравнения равновесия удовлетворяются тождественно.

Из уравнений совметности деформаций следует, что $\Psi$ удовлетворяет уравнению
\[
  \Delta\Delta\Psi = \Delta\left( \CP{^2\Psi}{x^2} + \CP{^2\Psi}{y^2} \right)=
  \CP{^4\Psi}{x^4} + 2\CP{^4\Psi}{x^2\dl y^2} + \Psi{^4}{y^4} = 0.
\]
В книге \cite{timo} примерно 200 страниц с примерами функции Эри.

\begin{Exa}
  $\Psi_1$ "--- линейная функция $x,y$. Отсюда $P_{ij} = 0$, это неинтересно.
\end{Exa}
\begin{Exa}
  $\Psi_2 = \frac12 a_2 x^2 + b_2 xy + \frac12 c_2 y^2$. Тогда $P_{11} = c_2$, $P_{22} =a_2$, $P_{12}= - b_2$. Частные случаи, например, $a_2,b_2$, но $c_2\ne 0$. Тогда $P_{11} = c_2$, остальные нули "--- простое растяжение пластинки вдоль оси $x$.
\end{Exa}

Какие граничные условия на поверхности пластинки?
% рисунок
\begin{itemize}
\item На боковой поверхности $x=\frac12l$ имеем $\ve n\parallel x$. Значит, $\ve P_n = P_{11}\ve n = P_{11}\ve e_1 = c_2$.

\item 
На боковой поверхности $x = -\frac12l$ имеем $\ve n\parallel x$, $\ve n = -\ve e_1$, $\ve P_n = -P_{11}\ve e_1$.
\item На остальных поверхностях вектор напряжений равен нулю, так как $n_x=0$ на этих поверхностях.
\end{itemize}
% рисунок
% здесь про примеры набирать не стал

\subsection{Общие теоремы}
Часто говорят: «давайте угадаем решение». Если решение не единственное, то это пустые слова. Так что теоремы единственности довольно важны: раз нашли решение, оно и правильно.
Но часто теоремы единственности нет вообще. Начинаются исследования на устойчивость.
\subsubsection{Теорема Клапейрона}
  Рассматривается равновесие упругой среды. Среда линейно-упругая, $T=T_0$. Уравнение равновесия 
\[
  \nabla_j P^{ij} = \rho_0 F^i = 0
\]
умножим на компоненты вектора перемещений $w_i$.
\[
  \nabla_j (P^{ij} w_i) - P^{ij}\nabla_j w_i + \rho_0 F^i w_i = 0.
\]
И проинтегрируем по объёму тела.
\[
  \underbrace{\int\limits_V \nabla_j (P^{ij}w_j)\,dV}_{I_1}
 - \underbrace{\int\limits_V P^{ij} \nabla_j w_i\,dV}_{I_2}
 + \underbrace{\Gint V \rho_0 F^i w_i\,dV}_{I_3}
\]
Тогда по формуле Гаусса"--~Остроградского
\[
 I_1 = \Gint\Sigma P^{ij} w_i n_j\,d\sigma = \Gint\Sigma P^i_n w_i\,d\sigma = 
  \Gint\Sigma(\ve P_n,\ve w)\,d\sigma.
\]
Второе слагаемое лучше было расписывать без интеграла.
\[
  I_2 = \Gint V P^{ij}\nabla_j w_i\,dV =
   \Gint V\left( \frac12 P^{ij}\nabla_j w_i + \frac12 P^{ji} \nabla_i w_j \right)\,dV = 
   \Gint V P^{ij}\frac12 (\nabla_{i w_j + \nabla_j w_i}\,V = \Gint V P^{ij}\e_{ij}\,dV
\]
Третье слагаемое совсем просто
\[
  I_3 = \Gint V\rho (\ve F,\ve w)\,dV.
\]

Итак, из уравнений равновесия, умноженных на $w_i$ и проинтегрированных по $V$ получаем соотношение
\begin{equation}\label{prekla}
  \Gint\Sigma(\ve P_n,\ve w)\,d\sigma + \Gint V \rho(\ve F,\ve w)\,dV = \Gint VP^{ij}\e_{ij}\,dV.
\end{equation}

Дальше используем, что $P^{ij} = \CP{\Phi}{\e_{ij}}$, где $\Phi$ "--- свободная энеркия единицы объёма,то есть $\Phi = \rho_0 \mathcal F$.

Из термодинамики для линейно-упругой среды при $T=T_0$ 
\[
  \Phi = \frac12 A^{ijkl} \e_{ij}\e_{kl};\quad
  P^{ij}\e_{ij} = \CP\Phi{\e_{ij}}\e_{ij} = 2\Phi.
\]
Таким образом \eqref{prekla} преобразуется в
\[
  \Gint\Sigma(\ve P_n,\ve w)\,d\sigma + \Gint V \rho(\ve F,\ve w)\,dV = 2\Gint V\Phi\,dV.
\]

\begin{The}[Клапейрона]
  $\ve \Phi = \Gint V\Phi\,dV = \frac12 \Big[\Gint V\rho(\ve F,\ve w)\,dV + \Gint \Sigma (\ve P_n,\ve w)\,d\sigma\Big)$. Полная упругая энергия (при $T=T_0$ равна половине работы внешних сил на перемещении $\w$.
\end{The}

\subsubsection{Идея теоремы единственности}
Пусть рассматривается равновесие, $T=T_0$ и задача линейная, а деформации малы. И рассматриваются два решения, удовлетворяющие одним и тем же граничным условиям. Тогда разность этих решений тоже будет решением задаче с нулевыми массовыми силами и по теореме Клапейрона получим, что эта разность ноль.
