\section{Лекция 16}
\begin{enumerate}
\item Уравнение притока тепла и уравнения второго закона термодинамики для упругих сред (в классической теории). Тождество Гиббса.
\item Внутренняя и свободная энергии как термодинамические потенциалы.
\item Свободная энергия и энтропия линейно-упругой анизотропной среды. Уравнение притока тепла. Число коэффициентов, определяющих анизотропную линейно-упругую среду.
\item Свободная энегрия и энтропия линейно-упругой изотропной среды.
\item Плоная система уравнения изотропной линейно-упругой среды.
\item Закон Гука для адиабатических процессов в линейной термоупругой среде.
\item Задача о температурных напряжений в стенках трубы.
\end{enumerate}

В классической теории упругости три предположения.
\begin{enumerate}
\item Кроме работы сил и тепла, других притоков энергии нет;
\item Тензор напряжений симметричен;
\item Процессы еформирования обратимые.
\end{enumerate}

Напишем два закона
\begin{itemize}
\item Первый закон термодинамики $dU + dE_{\text{кин}} = dA^{(e)} + dQ^{(e)} (+\cancel{dQ^{**}})$;
\item Теорема живых сил $dE_{\text{кин}} = dA^{(i)} + dA^{(e)}$.
\end{itemize}
Отсюда следует уравнение притока тепла $dU = -dA^{(i)}+dQ^{(e)}$. Или, в дифференциальной форме,
\[
  du = -\frac1{dm} dq^{(i)} + dq.
\]

Здесь $u$ "--- плотность внутренней энегрии; $dq$ "--- приток тепла к единице массы, $\frac1{dm} d^{(i)}$ "--- работа внутренний сил на единицу массы;
$\frac1{dm} dA^{(i)} = \frac1\rho P^{ij} \nabla_j v_i\,dt$, $v_i$ "--- компоненты скорости.

Найдём выражение для плотности работы внутренних сил
\[
  \frac1{dm} dA^{(i)} = -\frac1\rho P^{ij} \nabla_j v_i\,dt = -\frac1\rho P^{ij} 
  \bigg[ \underbrace{ \frac12(\nabla_j v_i + \nabla_j v_i)}_{e_{ij}} + \underbrace{\frac12(\nabla_j v_i - \nabla_i v_j)}_{\text{тензор вихря}}\bigg] \,dt
  =- \frac1\rho P^{ij} e_{ij}\,dt,
\]
так как $P^{ij} = P^{ji}$.

Далее верно ли $e_{ij} \,dt = d\e_{ij}$? Это верно в общем случае при лагранжевом описании в лагранжевой системе координат. При эйлеровом описании неверно. Но мы рассматриваем теорию, в которой деформации и перемещения, и скорости малы. Тогда, в частности, отличие лагранжевой системы координат от пространственной мало.

В геометрически линейной (малы деформации и перемещения) теории упругости. В этой теории формула $e_{ij}\,dt = de_{ij}$, даже когда не выполняется закон Гука, нужна именно геометрическая линейность.

Тогда в результате всех этих разговоров получается, что в геометрически линейной теории (если ещё $P^{ij}= P^{ji}$)
\[
  \frac1{dm} dA^{(i)} = -\frac1\rho P^{ij}\,d\e_{ij}.
\]
Кроме того, вместо $\rho$ можно писать $\rho_0$. В этом слагаемом учитывается только главный член.

Уравнение притока тепла в теории упругости
\begin{equation}
  du = \frac1{\rho_0} P^{ij} d\e_{ij} + dq.\label{1zt}
\end{equation}

Теперь второй закон термодинавики в дифференциальной форме $T\,ds = dq + dq'$. Здесь $T$ "--- абсолютная температура, $s$ "--- плотность энтропии, $dq'$ "--- некомпенсированное тепло, то есть $dq' = T\ol{\ol{d_is}}$, где $\ol{\ol{d_is}}$ "--- производство энтропии на единицу массы за счёт всех процессов, кроме теплопроводности.

В теории упругости $dq'=0$, так как деформация "--- обратимый процесс. Значит, второй закон термодамики в теории упругости 
\begin{equation}
R\,ds = dq.\label{2zt}
\end{equation}


Из первого и второго законов получается, так называемое, тождество Гиббса
\begin{equation}
  du = \frac1\rho O^{ij}\,d\e_{ij} + T\,ds.
\label{tg}
\end{equation}

Из тождества Гиббса следует, что
\[
  u = u(\e_{ij},s).
\]
Если эти аргументы константы, то $u$ тоже константа.

Если $d\e_{ij}$ независимы, то 
\begin{equation}
  P^{ij} = \rho\left( \CP u{\e_{ij}} \right)_{s=\const};\quad
  T = \left( \CP us \right)_{\e_{ij}=\const}. \label{eq4}
\end{equation}

Замечание. Для несжимаемой среды $g^{ij}\,d\e_{ij} = 0$. Для сжимаемой среды, этого нет и $d\e_{ij}$ независимы. Мы рассматриваем сжимаемые среды, поэтому \eqref{eq4} верно.

Если откуда-то известна $u = (\e_{ij},s)$, то из \eqref{eq4}  находим $T = T(\e_{ij},s)$ или $s= s(\e_{ij},T)$ и $P^{ij} = P^{ij}(\e_{ij},s)$, то есть $P^{ij} = P^{ij}(\e_{ij},T)$.
$u$ называют термодинамическим потенциалом, если $u = u(\e_{ij},s)$.

Если вы не хотите иметь дело с энтропией, но сразу с температурой, то можно ввести другой термодинамический потенциал. А именно свободную энергию. В расчёте на единицу массы свободная энергия определяется следующим образом
\[
\mathcal F = u - Ts;\quad u = \mathcal F + Ts.
\]
Если это  подставить в тождество Гиббса, найдём
\[
  du = d\mathcal F + T\,ds + s\,dT = \frac1\rho P^{ij}\e_{ij} + T\,ds.
\]

Таким образом
\begin{equation}
d\mathcal F = \frac1\rho P^{ij}\,d\e_{ij} - s\,dT.
\label{eq6}
\end{equation}

Отсюда следует, что $\mathcal F = \mathcal F (\e_{ij},T)$. Тогда
\begin{equation}
P^{ij} = \rho\left( \CP{\mathcal F}{\e_{ij}} \right)_{T=\const};\quad
s = -\left( \CP{\mathcal F}T \right)_{\e_{ij}=\const}
\label{eq78}
\end{equation}

Таким образом, закон Гука задавать не обязательно для линейно-упругой среды. Можно вместо этого определить внутреннюю энергию, как функцию от деформаций и энтропии или свободную энергию, как функцию деформаций и температуре.

Обычно вводят свободную энергию единицы объёма $\Phi = \rho_0\mathcal F$. Тогда формулы \eqref{eq78} переписываются таким образом
$P^{ij} = \left(\CP{\Phi}{\e_{ij}}\right)_{T=\const}$,
$s = -\frac1{\rho_0}\left( \CP\Phi T \right)_{\e_{ij}=\const}$.

Как задать $\Phi$? Будем считать, что $\e_{ij}$ малы и $T-T_0$ мало ($T_0$ "--- температура, при котороый $\e_{ij}=0$, если не действуют силы). Тогда можно представить $\Phi$ в виде ряда
\begin{multline*}
\Phi\big(\e_{ij},(T-T_0)\big) = \Phi_0 + \left( \CP\Phi{\e_{ij}} \right)_0\e_{ij} + \left( \CP\Phi t \right)_0(T-T_0) +\\+
\frac12 \left( \CP{^2\Phi}{\e_{ij}\dl\e_{kl}} \right)_0\e_{ij}\e_{kl} + \frac12\left( \CP{^2\Phi}{\e_{ij}\dl T} \right)_0\e_{ij}(T-T_0) + 
\frac12 \left( \CP{^2\Phi}{T^2} \right)_0(T-T_0)^2 + \dots
\end{multline*}
Если $\e_{ij}$, $T-T_0$ малы, то следующие члены можно не учитывать.

Обозначим $A^{ijkl} = \left( \CP{^2\Phi}{\e_{ij}\dl \e_{kl}} \right)_0$, $B^{ij} = \frac12\left( \CP{^2\Phi}{\e_{ij}\dl T} \right)_0$, $c = \left( \CP{^2\Phi}{T^2} \right)_0$. Тогда
\[
  \Phi = \Phi_0 + \left( \CP\Phi{\e_{ij}} \right)_0\e_{ij} + \left( \CP\Phi T \right)_0(T-T_0) + \frac12 A^{ijkl}\e_{ij}\e_{kl} +
  B_{ij}\e_{ij}(T-T_0) + \frac12 c (T-T_0)^2.
\]
Тогда $P^{ij} = \CP\Phi{\e_{ij}}$, $s = -\frac1{\rho_0}\CP\Phi T$.
\[
  \begin{cases}
    P^{ij} = \left( \CP\Phi{\e_{ij}} \right)_0 + A^{ijkl}\e_{kl} + B^{ij}(T-T_0);\\
    s = -\frac1{\rho_0}\left( \CP\Phi T \right)_0 - B^{ij}\e_{ij} - c(T-T_0).
\end{cases}
\]
Если начальное состояние выбрано так, что напряжения нулевое при $T=T_0$ и $\e_{ij}=0$, то $\left( \CP\Phi{\e_{ij}} \right)_0=0$. Обозначим
\[
  s_0 = -\frac1\rho_0\left( \CP\Phi T \right)_0
\]
плотность энтропии в недеформированном состоянии.

Итак, в линейно-упругой анизотропной среде
\begin{eqnarray*}
  P^{ij} &=&  A^{ijkl} \e_{kl} + B^{ij}(T-T_0);\\
  s 	 &=&  s_0-\frac1{\rho_0} B^{ij}\e_{ij} - \frac c{\rho_0} (T-T_0).
\end{eqnarray*}
Сначала выясним физический смысл коэффициента $c$. Про $A^{ijkl}$ и $B^{ij}$ мы что-то знаем. А вот коэффициент $c$ для нас новый. Для понимания его физического смысла рассмотрим уравнение притока тепла.
\[
  du = \frac1{\rho}P^{ij}\,d\e_{ij} + dq.
\]
Надо задать $du$ и $dq$, тогда будет уравнение притока тепла. Что такое $u$? Можно вывести, в задачнике оно написано. Но можно без $u$ обойтись, потому что из тождества Гиббса следует, что
\[
  u - \frac1\rho P^{ij}\,d{\e_{ij}} = T\,ds.
\]
Тогда получим собственно второй закон термодинамики: $T\,ds = dq$. Уравнение притока тепла и второй закон термодинамики равносильны, если учесть тождество Гиббса.

Будем писать $T\,ds = dq$. Более того, будем писать $T_0\,ds = dq$. Теперь чему равняется $ds$.
\[
  -\frac{T_0}{\rho_0} B^{ij}\,d\e_{ij} - \frac{cT_0}{\rho_0}\,dT = dq.
\]
Если мы рассмотрим процесс, в котором деформаций нет, то есть $d\e_{ij}=0$, закладываем в жёсткий контейнер и греем. Тогда
\[
\left( \DP qT \right)_{\e_{ij}=\const} = -\frac{cT_0}{\rho_0} = c_\e.
\]
Таким образом, $c = -\frac{\rho_0}{T_0}$, где $c_\e$ "--- теплоёмкости при постоянной деформации.


Сколько коэффициентов надо задать, чтобы задать анизотропную упругую среду? $A^{ijkl}, B^{ij}, c_\e$ и ещё, может быть, коэффициент теплороводности, если среда теплопроводная и подчиняется закону Фурье. Сколько независимых $A^{ijkl}$? Так как $P^{ij}$ симметричны, и $A^{ijkl}$ "--- это вторые производные, независимых компонент не так много. Вторые производные не зависят от порядка дифференцирования. Итого, всего независимых $A^{ijkl}$ двадцать одна штука. Всего коэффициентов $21 + 6 +1 +1 = 29$, если есть теплопроводность.

Это для произвольного анизотропного тела.
