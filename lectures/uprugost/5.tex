\section{Лекция 5}
\begin{enumerate}
	\item Отступление: об ударных волнах в сжимаемых жидкостях или газах.
	\item Плоское течение Пуазейля.
	\item Комбинация решений для течения Куэтто и Пуазейля.
	\item Оценка величин различных членов уравнения Навье"--~Стокса. Число Рейнольдса.
	\item Приближение Стокса для течений с малыми числами Рейнольдса.
\end{enumerate}
\subsection{Об ударных волнах}
Возвращаемся на время к модели идеальной жидкости. В идеальной жидкости приходится строить решения с поверхностями разныва. Например, волны Римана не могут существовать достаточно долгое время (именно волны сжатия), волны опрокидываются, поэтому приходится строить решения с ударными волнами. И не только волны Римана, это часто случается.

Поверхности разрыва "--- поверхности, по разные стороны которых разные значения скорости, давления, плотности, температуры. С физической точки зрения это тонкий слой, где параметры меняются быстро. 
% рисунок поверхность разрыва

Я написала вам условия на поверхности разрыва, следующие из законов сохранения. Напишем их конкретно для идеальной жидкости. В системе координат, в которой поверхность разрыва неподвижна эти условия (без учёта идеальности жидкости) выглядят следующим образом. Масса часто обозначается через букву $j$. Массовых сил нет, так как масса поверхности равна нулю. Итак законы сохранения массы, количества движения, энергии (притоки тепла, как правило, считаются равными нулю, так как переход через поверхность разрыва происходит достаточно быстро и процесс можно считать адиабатическим)
\begin{eqnarray*}
	v_{n_1}\,\rho_1 &=& v_{n_2}\,\rho_2 = j;\\
	j(\ve v_2-\ve v_1)&=&\ve P_{n_1} - \ve P_{n_2};\\
	j\left(\frac{v_2^2}2 + u_2 - \frac{v_1^2}2 - u_1\right) &=&
	  (\ve P_{n_2},\ve v_2)
	  - (\ve P_{n_1},\ve v_1)
	  -q_{n_1} + q_{n_2};\\
	j(s_2-s_1) &=& \Omega.
\end{eqnarray*}

Пусть теперь жидкость идеальная и сжимаемая, то есть $\ve P_n = -p\,\ve n$. Рассмотрим, так называемые, адиабатические скачки, то есть $\ve q=0$. Тогда
\begin{eqnarray*}
	v_{n_1}\,\rho_1 &= &v_{n_2};\\
	j(\ve v_2-\ve v_1) = (p_1-p_2)\ve n &\iff&
	  \begin{cases}
		  j(v_{n_2}-v_{n_1}) = p_1-p_2;\\
		  j(\ve v_{\tau_2} - \ve v_{\tau_1}) = 0.
	  \end{cases}\\
	j\left(\frac{v_2^2}2 + u_2 - \frac{v_1^2}2 - u_1^2\right) &=& 
	  p_1\,v_{n_1} - p_2\,v_{n_2};\\
	j(s_2-s_1)=0?&&\text{нет!}.
\end{eqnarray*}
Итого, касательные состовляющие не рвутся, рвутся только нормальные.

Если газ идеальный совершенный, а процесс адиабатический, то $s = c_v\,\ln\frac p{\rho^\gamma} + \const$, $p = A\,e^{\frac s{c_v}}\rho^\gamma$, $T\,ds = dq = 0$, $u = c_v\,T = \frac1{\gamma-1}\frac P\rho$.

У нас получается, что соотношений четыре на три неизвестных. Если из первых трёх получить четвёртое, не получим ноль. Но поверхность разрыва "--- это узкая зона, но велики скорости деформаций, поэтому там нужно обязательно учитывать вязкость.

Передняя сторона скачка "--- сторона, с которой жидкость входит в скачок. В данном случае пусть передняя сторона помечается индексом $1$.

Вся эта теория применяется для изучения движения транспорта или потока людей.
% картинка транпорт, поток людей
В этом случае распределение плотности людей "--- полна Римана и она опрокидывается. Люди быстро бегут и утыкаются в пробку, тогда пробка растёт в сторону, против движения людей. А может быть и наоборот, пробка рассасывается быстрее набега людей.

Из написанных нами соотношения (из первых трёх), зная всё с одной стороны, можно вычислить всё с другой.

Если скачок движется со скоростью $\ve D$, то в соотношениях нужно заменить (только в первом соотношении) скорость $ v_n$ на $v_n - D$. Тогда за скачком нужно задать ещё одно условие. В конкретных задачах обычно ясно, какое условие задать.

\subsubsection{3 факта о свойствах ударных волн в идеальной жидкости}
Без доказательства. Можно прочитать во втором томе Седова.
\begin{enumerate}
	\item Соотношения (1)--(3) допускает два типа ударных волн. Что вообще такое ударная волна: это разрыв, в котором $j$ не равно нулю (то есть те поверхности разрыва, которые не являются границами раздела сред. Если говорим о поверхности моря, то это тангенсальный разрыв или контактный разрыв, перетока массы через границы нет). Итак два типа ударных волн:
		\begin{enumerate}
			\item Скачки уплотнени: если 1 "--- передняя сторона, то определяется $\rho_2>\rho_1$. При этом всегда $p_2>p_1$.
			\item Скачки разрежения: если $1$ "--- передняя сторона, то определяется через $\rho_2<\rho_1$. Всегда при этом $p_2<p_1$.
		\end{enumerate}
	Мы это не выводим, но это следствия (1)--(3) и второго закона термодинамики.
\item Вытекает из второго закона термодинамики. Если уравнение состояния среды таково, что вторая производная $\left(\CP{^2p}{V^2}\right)_{s=\const}>0$, где $V = \frac1\rho$, то при переходе среды через скачок уплотнения энтропия растёт, а при переходе через скачок разряжения энтропия падает.

	Это утверждение называется теоремой Цемплена. Из него следует, что в газах $\left(\CP{^2p}{V^2}\right)_{s=\const}>0$ скачки разряжения невозможны.

	Например, совершенный газ: $\left(\CP{^2p}{V^2}\right)_s = \gamma(\gamma+1)c(s)\rho^{\gamma+2}>0$. Когда поршень отодвигается в эксперименте и не возникает ударной волны, но непрерывное уменьшение плотности.
	% рисунок поршень две штуки
\item Из второго закона термодинамики и условий (1)--(3) доказывается, что $v_{n_1}>a_1$, $v_{n_2}<a_2$, где $a_1$ "--- скоростью звука перед скачком, $a_2$ "--- скоростью звука за скачком (нормаль направлена в сторону 2), при условии, что $\left(\CP{^2p}{V^2}\right)_s>0$. Это значит, что при этом условии, стоящие скачки возможны только в сверхзвуковых потоках.
	% ещё какая-то картинка

	Это же утверждение равносильно тому, что скоростью скачка относительно газа, расположенного впереди скачка больше скорости звука, а по отношению к газу за скачком "--- меньше скорости звука.

	На самом деле факт номер три верен для любых сред, не обязательно условие на вторую производную. Но тогда он выводится не из второго закона термодинамики, а из условий эволционности, которые я не буду рассказывать. О них можно почитать в нашем задачнике.
\end{enumerate}
\subsection{Что такое вязкая жидкость}
Тензор напряжений состоит из двух частей $p_{ij} = -p\,g_{ij} + \tau_{ij}$, где $p$ не зависит явно от скорости "--- давление, $\tau_{ij}$ "--- функции компонент тензора скорости деформаций $e_{ij}$, которые равны нулю, если все $e_{ij}=0$. Также $\tau_{ij}$ могут зависеть от $T$, они называются компонентами тензора вязких напряжений.

Жидкость называется линейно-вязкой, если $\tau^{ij} = A^{ijkl}\,e_{kl}$. Если жидкость изотропная, то линейная связь представляется в виде закона Навье"--~Стокса
\[
  \tau_{ij} = \lambda\,\div\ve v\,g_{ij} + 2\,\mu\,e_{ij}.
\]
Здесь $\lambda,\mu$ "--- коэффициенты вязкости.

Уравнения движения вязкой жидкости называются уравнениями Навье"--~Стокса. Для несжимаемой жидкости (для сжимаемой ещё есть член $\frac{\lambda+\mu}\rho\grad\div\ve v$)
\[
	\DP{\ve v}t = \ve F - \frac1\rho\grad p + \nu\Delta\ve v
\]
Здесь $\nu = \frac\mu\rho$ "--- кинематический коэффициент вязкости.

Эти уравнения гораздо сложнее уравнений Эйлера. Для несжимаемой жидкости есть только один параметр, характеризующий вязкость. Если $\nu=0$, жидкость идеальна.

Вам было в прошлый раз рассказано, что такое течение Куэтта. Это стациональное, ламинарное (не знаем ещё, но интуитивно знаем) движение вязкой жидкости между двумя параллельными плоскостями, одна из которых движется, а другая неподвижна, при отсутствии перепада давления (или градиента давления).
% картинка линейное распределение скорости между пластинками.
Если вехняя пластинка движется со скоростью $v_0$, а нижняя покоится, то $v_x = v_0\frac yh$, а сила $\tau = \mu\frac {v_0}h$.

На самом деле, если имеется две сферы, то течение в слое между ними тоже называется течением Куэтта.

Вычислим $\tau$ для течения Куэтта на конкретном примере. Пусть $h=1\,\text{см}$, $v_0 = 1\,\text{м}/\text{с}$. Пусть жидкость вода $\mu = 10^{-3}\,\text{Па}\cdot\text{с}$ при комнатной температуре. Тогда $\tau = 10^{-1}\,\text{Па}$. Сравним с атмосферным давлением $p_a = 10^5\,\text{па}$. Отношение $\frac{\tau}{p_a}$ в миллион раз. Во многих случаях вязкостью можно принебречь, но не течением Куэтта, конечно, у него вообще движение вызвано только вязкостью.

\subsection{Плоское течение Пуазейля}
Это ламинарное стационарное течение вязкой жидкости между двумя параллельными неподвижными плоскостями, вызванное перепадом давления вдоль пластин.
% картинка и в трубе
Можео рассматривать движение и в трубе
