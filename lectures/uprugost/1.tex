\section{Лекция 1 (17)}
План у нас такой. Мы немного не разобрали два вопроса, которые встречаются в гос. экзамене. Распространение малых возмущений в газе и возмущение немалых возмущений. Поэтому первые несколько лекций будем продолжать жидкости.

Сегодня продолжение предыдущей лекции. Но волны будут не плоские, а сферические.

\begin{enumerate}
  \item Движение сжимаемой идеальной жидкости с малыми возмущениями (со сферическими волнами). Уравнение для потенциала скорости.
  \item Источник в сжимаемой жидкости. Запаздывающий потенциал.
  \item Распространение малых возмущений от движущегося источника. Эффект Доплера. Конус Маха.
\end{enumerate}

Нам нужно будет написать уравнения в сферических координатах.

Мы рассматриваем вот что. Считаем, что
\begin{enumerate}
\item Жидкость сжимаемая идеальная (прежде всего пренебрегаем трением);
\item Массовые силы не учитываем.
\item Движение баротропное, то есть $\rho = \rho(p)$.
\item Движение потенциальное, то есть $\ve v = \grad\phi$.
\end{enumerate}

Мы будем рассматривать движения, начинающиеся с покоя. Покой потенциален, значит, и дальнейшее движение будет потенциально. Уравнение неразрывности и уравнение Эйлера.
\begin{eqnarray}
\CP\rho t + v^i\CP\rho{x^i} + \rho\div \ve v &=& 0;\\
\CP{\ve v}t + v^i\CP{\ve v}{x^i} + \frac1\rho \underbrace{\DP p\rho}_{a^2}\grad
\rho &=&0;
\end{eqnarray}
Обозначают  $a^2 = \DP p\rho$ и называют эту величину скоростью малых возмущений.

Будем рассматривать малые возмущения. Будем называть фоном состояние $\ve v_0 = 0$, $\rho = \rho_0$, $p=p_0$. Переходим к переменным-возмущениям $\rho = \rho_0+\Til\rho$, $\Til v$, и~т.\,д. При этом $\frac{\Til\rho}{\rho_0}\ll1$.

После перехода к переменным-возмущениям и учёта потенциальности движения получаем систему.
\begin{eqnarray*}
  \CP{\Til\rho}t + \rho_0\Delta\phi &=&0;\\
  \CP\phi t + \frac{a_0^2}{\rho_0}\Til\rho&=&.
\end{eqnarray*}
Здесь $\Delta\phi = \div\grad\phi = \div\ve v$. Если делать подстановки одного уравнения в другое, то на каждую переменную получаем волновое уравнение.
\begin{equation*}
  \CP{^2\phi}{t^2} = a^2_0\Delta\phi;\qquad \CP{^2\Til\rho}{t^2} = a_0^2\Delta\Til\rho\dots
\end{equation*}

Движение с плоскими волнами мы рассматривали. Мы считаем в этом случае $\phi = \phi(t,x)$ "--- зависимость только от одной координаты. Тогда волновое уравнение становится уравнением колебания струны.
\[
  \CP{^\phi}{t^2} = a_0^2\CP{^2\phi}{x^2}.
\]
Для этого уравнение есть решение д'Аламбера $\phi = f_1(x-a_0\,t) + f_2(x+a_0\,t)$. Если $x - a_0\,t = \const$, то $f_1=\const$, то есть $f_1=\const$ в точках $x = \const + a_0\,t$.


Бегущая волна перемещается без изменения формы и амплитуды.

Как послушать, что говорят соседи. Нужно взять стеклянную банку и приложить к стене. Через банку пойдёт плоская волная почти без затухания. Рупор делают с этой же целью. Вот я говорю и звук распространяется вокруг сферически. Чтобы потерь было меньше, нужно сделать рупор.

\subsection{Сферические волны}
Пусть $\CP{^2\phi}{t^2} = a_0^2\Delta\phi$, $\phi = \phi(t,r)$, $r = \sqrt{x^2+y^2+z^2}$.

Получим формулу для оператора Лапласа в сферических координатах.
\[
  \Delta\phi = \div\grad\phi = \nabla_i(g^{ij}\nabla_j\phi) = g^{ij}\nabla_i\nabla_j\phi.
\]
Что нужно делать, чтобы на ходу получить итоговую формулу? Нужно знать, что есть набла от тензора с одним нижним индексом.
\[
  \nabla_i v_j = \CP{v_j}{x_i} - v_k\Gamma_{ij}^k = \CP{^2\phi}{x^i\dl x^j} - \CP\phi{x^k}\Gamma_{ij}^k.
\]
Введём сферические координаты следующего типа.
\begin{eqnarray*}
  x^1 &=& r\,\sin\theta\cos\lambda;\\
  x^2 &=& r\,\sin\theta\sin\lambda;\\
  x^3 &=& r\,\cos\theta.
\end{eqnarray*}
Нужно вычислить компоненты метрического тензора
\[
  ds^2 = dr^2 + r^2\,d\theta^2 + (r\,\sin\theta)^2d\lambda^2.
\]
Это компоненты с нижними индексами, а нам нужны будут с верхними. Ещё нам нужны некоторые символы Кристоффеля, но не все. Нужны только $\Gamma^1_{22} = -r$ и $\Gamma_{33}^1 = r^2$.

Ответ получится следующий. Если $\phi = \phi (t,r)$, то
\[
  \Delta\phi = \CP{^2\phi}{r^2} = \frac2r\CP\phi r = \frac1r\CP{^2(r\,\phi)}{r^2}.
\]

Напишем теперь уравнение для потенциала для малых возмущений со сферическими волнами.
\[
  \CP{^2\phi}{t^2} = \frac{a_0^2}r\CP{^2(r\,\phi}{r^2},\qquad
  \CP{^2}{t^2}(r\,\phi) = a_0^2\CP{r\,\phi}{r^2}.
\]
Таким образом $(r\,\phi)$ удовлетворяет уравнению колебания струны. Можем использовать решение д'Аламера
\[
  r\,\phi = f_1(r-a_0\,t) + f_2(r+a_0\,t);\qquad \phi = \frac{f_1(r-a_0\,t)}r + \frac{f_2(r+a_0\,t)}r.
\]
Что же это за решение? Это опять две волны. Пусть в начальный момент $\phi$ задано при $t=0$. Что будет при $t>0$? Будет так. Пусть $r-a_0\,t = r_0$, то есть $r = a_0\,t + r_0$ и $\phi(r,t) = \frac{f_1(r_0)}r$. $a_0$ "--- скорость волны, она не меняется. С другой стороны меняется амплитуда.

$\phi$ состоит из двух слагаемых: сходящаяся волна с растущей амлитудой, расходящаяся волна с убывающей амплитудой (как расходящиеся круги колеблются всё с меньшей амплитудой). Это уже не бегущая волна, которая не меняет форму. Форма здесь меняется.

\subsection{Расходящаяся волна как течение из источника}
Рассмотрим теперь только расходящуюся волну $\phi = \frac1rf_1(r-a_0\,t)$. Её можно трактовать, как движение от источника. Я хочу ввести такое обозначение 
\[
  \phi = \frac{f_1\left(a_0\left(\frac r{a_0} - 1\right)\right)}r = 
  \frac{f_1\left(-a_0\left(t - \frac r{a_0}\right)\right)}r = 
  \frac{-Q\left(t - \frac r{a_0}\right)}{4\,\pi\,r}.
\]
Само переобозначение определяется формулой
\begin{equation}
  f_1\left(-a_0\left(t - \frac r{a_0}\right)\right)\equiv - \frac{Q\left(t - \frac r{a_0}\right)}{4\,\pi}
\end{equation}

Решение $\phi=
  \frac{-Q\left(t - \frac r{a_0}\right)}{4\,\pi\,t}$ соотвествует источнику в начале координат. Почему это источник?
 Если $\phi=\const$, то $r=\const$ "--- это сфера. Можно посчитать расход через сферу радиуса $R$.
\[
  \int\limits_{\Sigma}v_r\,d\sigma = \CP\phi r\bigg|_{r=R}\cdot 4\,\pi,R^2 = 
  \left(\frac{Q'}{4\,\pi\,R\,a_0} + \frac Q{4\pi R^2}\right)4\,\pi\,R^2 = 
  \frac{Q'\cdot R}{a_0} + Q(t-R/a_0)\xrightarrow{R\to0}Q(t).
\]

Сравним источники в несжимаемой жидкости $\phi = -\frac{Q(t)}{4\,\pi\,t}$ и в сжимаемой жидкости $\phi = -\frac{Q(t-r/a_0)}{4\,\pi\,r}$. Пусть $Q(t)\ne0$ только при $t_1>t>0$.
\begin{itemize}
\item В несжимаемой жидкости $\phi\ne0$ во всех точках, если $Q\ne0$. Кто-то один сказал слово, то слышит сразу вся вселенная. Звук распространяется с бесконечной скоростью. Если я рассматриваю небольшие объекты, это нормально. Вот я говорю и задние парты, можно считать, мгновенно меня слышат.
\item  В сжимаемой жидкости получается, что $\phi\ne0$, когда аргумент у $Q$ лежит в промежутке
\[
  0< t - r/a_0<t_1\imp a_0\,t > r > a_0(t-t_1).
\]
Когда источник включился, это не значит, что уже есть движение вокруг. Движение находится только в~сферическом слое. Если источник включился и тут же выключился, то всё движение находится на~сфере, которая расширяется. Таким образом, если источник включился, это ещё не значит, что движение где-то есть. А если источник уже выключился, то движение ещё где-нибудь может быть. Это два основных отличия от случая несжимаемой жидкости.
\end{itemize}

\subsection{Эффект Доплера. Конус Маха}
Пусть источник движется с постоянной скоростью $u$. Будут разные картины при $u<,=,>a_0$.
