\section{Лекция 9}
\begin{enumerate}
\item Полуэмпирическая теория Прандтля для турбулентной потоков вдоль снетки.
\item Логарифмический профиль скорости в тубрулентных потоках вдоль твёрдой стенки.
\end{enumerate}
Мы с вами занимаемся турбулентными движениями, когда характеристики движения кроме своих средних значений испытывают некую пульсацию. Например, $v_x = \MID{v_x} + v'_x$, где $v'_x$ "--- польсационный добавок, $\MID {v'_x}=0$.

Написали Рейнольсово осреденение уравнений для несжимаемой жидкости.
\begin{eqnarray}
  \label{TurboM}\div\la\ve v\ra &=&0;\\
  \rho\DP{\MID{v_x}}t &=& \rho\MID{F_x} +   \CP{\big(\MID{P_{xx}}+T_{xx}\big)}x + \CP{\big(\MID{P_{xy}}+T_{xy}\big)}y + \CP{\big(\MID{P_{xz}}+T_{xz}\big)}z,\\
  \rho\DP{\MID{v_y}}t &=& \rho\MID{F_y} +   \CP{\big(\MID{P_{yx}}+T_{yx}\big)}x + \CP{\big(\MID{P_{yy}}+T_{yy}\big)}y + \CP{\big(\MID{P_{yz}}+T_{yz}\big)}z,\\
  \rho\DP{\MID{v_z}}t &=& \rho\MID{F_z} +   \CP{\big(\MID{P_{zx}}+T_{zx}\big)}x + \CP{\big(\MID{P_{zy}}+T_{zy}\big)}y + \CP{\big(\MID{P_{zz}}+T_{zz}\big)}z,\\
\end{eqnarray}
Если жидкость сжимаема, нужна ещё и термодинамика. А порой и химические реакции нужно учитывать. Но даже в нашем простейшем случае несжимаемой жидкости система не замкнута. Что нам нужно ещё:
\begin{enumerate}
\item Как выражается $\la p^{ij}\ra$ через $\la v_i\ra,\MID p$.
\item Как выражается $T_{ij}$ через хоть какие-нибудь осреднённые параметры.
\end{enumerate}
Второй пункт самый трудный в этой теории. Ведь $T_{ij} = -\MID{\rho\,v'_i\,v'_j}$ "--- компоненты тензора турбулентных напряжений, где $v'_i$ зависят не только от среды, но и от условий, в которых происходит движение. На стенке пульсации нет, пульсации могут гаситься.

Получается так, что для разных классов движений вводятся гипотезы и обрабатываются эксперименты. Просто предлагаются формулы или уравнения. Это называется полуэмпирическая теория. Авторы разные, предлагают разное. У некоторых только конечные соотношения, у других какие-то дифференциальные уравнения.

Напишем законы для $T_{ij}$, необходимые для замыкания системы уравнений. Для вязких напряжений в линейно-вязкой несжимаемой жидкости, то есть
\[
  p{ij} = -p\,g_{ij} + \tau_{ij},\quad \tau_{ij} = 2\,\mu\,e_{ij},
\]
$\mu$ "--- коэффициент вязкости, $e_{ij}$ "--- компоненты тензора скоростей деформаций.

Предположим, что $T_{ij} = 2\mu_T\MID{e_{ij}}$, где $\mu_T$ "--- коэффициент турбулентной вязкости. Это называется гипотезой Буссинеска.

А дальше для $\mu_T$ нужно давать какие-то законы. Замыкание мы ещё не выполнили.

А вообще эта гипотеза обобщается. Пока она не изотропная. В разных направлениях, разные вязкости. Поэтому используют тензор вместо одного коэффициента, то есть $T_{ij} = A^T_{ijkl}\MID{e_{kl}}$.

\subsection{Полуэмпирическая модель Прандтля для течения вдоль стенки}
Она была построена на аналогии между молекулярной вязкости и турбулентной. Частицы перескакивают из слоя в слой, возникает взаимодействие между слоями.

\subsubsection{Для каких потоков строится теория Прандтля}
\begin{enumerate}
\item \label{Pra1} Жидкость несжимаемая, однородная, вязкая (то есть могут быть касательные напряжение, не обязательна линейность вязкости по скоростям);
\item Рассматривается поток вдоль твёрдой стенки.
\item Направим ось $x$ вдоль средней скорости потока. Тогда предполагаем $\MID{v_y}=\MID{v_z}=0$.
\item Направим ось $y$ перпендикуряно стенке. Предполагаем, что течение плоско-параллельное, то есть всё не~зависит от $z$.
\item \label{Pra5} Всё не зависит от $x$ (в том числе нет градиента давления вдоль оси $x$, он есть где-то вверху, а у стенки уже несущественно).
\item Массовые силы не учитываются.
\item \label{Pra7} Движение в среднем стационарно.
\end{enumerate}

Уравнение неразрывности выполняется автоматом в силу предположения \eqref{Pra5}
\begin{eqnarray*}
  \div\la\ve v\ra &=&0;\\
  \rho\cancel{\DP{\MID{v_x}}t} &=& \rho\cancel{\MID{F_x}} +   \cancel{\CP{\big(\MID{P_{xx}}+T_{xx}\big)}x} + \CP{\big(\MID{P_{xy}}+T_{xy}\big)}y + \cancel{\CP{\big(\MID{P_{xz}}+T_{xz}\big)}z}.
\end{eqnarray*}
Преобразуем то, что осталось. Имеем $p_{xy} = \tau_{xy} = 2\,\mu\,e_{xy}$. Тогда \[
  \MID{p_{xy}} = 2\,\mu\,e_{xy} = \mu\mid[\bigg]{\CP{v_x}y+\CP{v_y}x} = \mu\CP{\MID{v_x}}y.
\]
Это просто, а для $T_{xy}$ надо строить четыре предположения Прандтля про $T_{xy} = -\MID{\rho\,v'_x\,v'_y}$.
\begin{enumerate}
\item Можно ввести понятие \textbf{поток смещения} $l_1$ как расстояние, на которое прыгают частицы за счёт турбулентности (поперёк потока).
\item Пульсации $v'_x$ связана с тем, что частицы переходят из слоя в слой и превносят количество движения. Предполагаем, что $v'_x = \MID{v_x}(y+l_1) - \MID{v_x}(y)=l_1\CP{\MID{v_x}}y$ (в круглых скобках указана точка, в которой замеряется средняя скорость).
\item Нет преимущества у какого-либо направления пульсации; пульсации по всем направлениям одного порядка, то есть $|v'_y| = \a\,|v'_x|$, где $\a\sim1$.

В силу первых трёх предположений имеем
\[
  |T_{xy}| = \MID{\rho\,\a\,l_1^2}
\bigg|\CP{\MID{v_x}}y\bigg|^2.
\]
\item Предположение на величину $\MID{\a\,l_1^2}$ будет ниже.
\end{enumerate}
Сначала поймём, как написать звёздочку без модулей с учётом знака. Рассмотрим два случая.
\begin{roItems}
\item $\CP{v_x}y>0$ . Тогда если $v'_y>0$, то $v'_x<0$.
% рисунок
Тогда $T_{xy} = -\MID{\rho\,v'_x\,v'_y}<0$.
\item Наоборот $\CP{v_x}y<0$. Тогда ясно, если если $v'_y>0$, то это означает, что снизу (у стенки) скорость большая, а сверху поменьше, то есть $v'_x>0$. Если $v'_y<0$, то $v'_x<0$. Тогда и $T_{xy}<0$.
\end{roItems}
Раскрываем модули
\[
  T_{xy} = \MID{\rho\,\alpha\,l_1^2}\bigg|\CP{\MID{v_x}}y\bigg|\CP{\MID{v_x}}y.
\]
Теория была бы закончена, если бы мы знали $\a\,l_1^2$.
\begin{enumerate}\setcounter{enumi}{3}
\item Обозначим $\MID{\alpha\,l_1^2}=:l^2$. Гипотеза состоит в том, что в течении у стенки задаётся формулой $l = \varkappa\,y$, где $\varkappa=\const$ находится из эксперимента.
\end{enumerate}
Окончательно, формула для $T_{xy}$ в теории Прандтля
\[
  T_{xy} = \rho\varkappa^2y^2\bigg|\CP{\MID{v_x}}y\bigg|\CP{\MID{v_X}}y.
\]
Можно записать эту формулу в виде $T_{xy} = \mu_T\CP{\MID{v_x}}y$ "--- аналогично формуле $\tau_{xy}$. Здесь $\mu_T$ "--- коэффициент турбулентной вязкости, но $\mu_T\ne\const$, а зависит от $y$ и от $\CP{\MID{v_x}}y$.

Оказалось, что для всех сред эта $\varkappa\approx0{,}4$ для потоков у стенки. Называется постоянной К\'{а}рмана.

\subsection{Задача об обтекании профиля в потоках у стенки при условиях \eqref{Pra1}--\eqref{Pra7}}
Покажем, что в турбклентной области профить скорости логаримфический. Показать очень просто для линейно-вязкой жидкости.
\[
  \DP{\big(\MID{p_{xy}}+T_{xy}\big)}y=0;\quad \MID{p_{xy}} = \MID{\tau_{xy}} = \mu\CP{\MID{v_x}}y, 
\MID{\tau_{xy}} + T_{xy} = \tau_0 = \const.
\]
В частности у стенки касательные напряжения $\tau_0$. Подставляем формулу для $T_{xy}$, имея в виду, что $\CP{v_x}y>0$ (стенка тормозит) $T_{xy} = \rho\varkappa^2\,y^2\left(\CP{\MID{v_x}}y\right)$. Интегрировать не будем.

Разобьём область на три части.
\begin{enumerate}
\item Течение в тонком слое у самой стенки (вязкий подслой). Там пульсации малы, так как стенка мешает. Поэтому там в этом тонком слое можно принебречь нелинейным $T_{xy}$. Имеем $\MID{\tau_{xy}} = \tau_0$, то есть $\CP{\MID{v_x}}y = \frac{\tau_0}\mu$,
\[
  \MID{v_x} = \frac{\tau_0}\mu\,y+ C.
\]
Но $C=0$ в силу условия прилипания, то есть $v_x|_{y=0}=0$. В общем, линейный профиль.
\item Теперь более-менее вдали рассмотрим, где $T_{xy}\gg\MID{\tau_{xy}}$ Пренебрегаем $\tau_{xy}$. Тогда $T_{xy} = \tau_0$ и
\[
  \rho\varkappa^2\,y^2\,\left(\CP{\MID{v_x}}y\right)^2 = \tau_0.
\]
Отсюда $\CP{v_x}y = \sqrt{\frac{\tau_0}\rho}\frac1\varkappa\,\frac1y$ и получаем погарифмический профиль
\[
  \MID{v_x} = \sqrt{\frac{\tau_0}\rho}\frac1\varkappa \ln y + C_1.
\]
Есть переходный слой. $C_1$ получается из склейки слоёв.
\end{enumerate}
В трубе тоже есть логарифмический профиль. Так что теория действительно работает.

Мы рассмотрели только частный случай движения "--- движение у стенки. Существуют другие полуэмпирические модели. Они более сложны, содержат большее количество констант. Наиболее популярная сейчас, так называемая, $k-\e$-модель; здесь $k$ "--- плотность энергии тубрулентных пульсаций, $k\sim\MID{(v'_x)^2+(v'_y)^2+(v'_z)^2}$, $\e$ "--- диссипация турбулентной энергии. И тогда просто по теории размерности остаётся необходимым выяснить $\mu_T(k,\e)$. Надо будет писать закон сохранения энергии, осреднять и по дороге делать добавочные предположения.
