\section{Лекция 6}
\begin{enumerate}
\item Плоское течение Пуазейля.
\item Течение между двумя плоскостями, одна из которых движется, причём имеется градиент (или перепад) давления вдоль потока (это есть комбинация Куэтта и Пуазейля).
\item Задача об обтекании тела вязкой жидкостью. Понятие пространственного и временн\'{о}го масштабов являения. Оценка величины членов уравнений Навье"--~Стокса.
\item Число Рейнольдса.
\item Приближения Стокста для движений с малыми числами Рейнольдса.
\item Движение с большими числами Рейнольдса.
\end{enumerate}

\subsection{Плоское течение Пуазейля}
Рассматриваем ламинарное стационарное течение вязкой жидкости между двумя параллельными неподвижными плоскостями, происходящее под действием перепада давления вдоль плоскостей.
% рисунок две неподвижных плоскости
Пусть есть два сечения и $p_1>p_2$.

Предположения о жидкости и движении
\begin{enumerate}
\item Жидкость несжимаемая, изотропная, линейно вязкая, коэффициент вязкости постоянен.
\item Движение стационарное.
\item Все частицы движутся параллельно градиенту давления. Если ось $x$ параллельна градиенту давления, ось $y$ направлена по нормали к плоскостям, то $v_y=v_z=0$, $v_x=v_x(x,y,z)$.
\item Движение плоскопараллельное $v_x = v_x(x,y)$.
\item Массовые силы не учитываются.
\end{enumerate}

Задача найти $v_x = v(x,y)$.

Математическая постановка задачи, то есть уравнения и граничные условия.
\begin{eqnarray}
\label{Pu1}\div\ve v&=&0;\\
\label{Pu2}\DP{v_x}t &=& -\frac1\rho\CP p x + \nu\,\Delta v_x;\\
\label{Pu3}\DP{v_y}t &=& - \frac1\rho\CP p y + \nu\,\Delta v_y;\\
\label{Pu4}\DP{v_z}t &=& - \frac1\rho\CP p z + \nu\,\Delta v_z;\\
\end{eqnarray}
Здесь $\nu = \frac\mu\rho$ "--- кинематический коэффициент вязкости.

Граничные условия на пластинках $v|_{y=0}=0$, $v|_{y=h}=0$. Ещё условия на давление $p|_{x=x_1}=p_1$, $p|_{x=x_2}=p_2$.

Из \eqref{Pu1} имеем $\CP{v_x}x = 0$ и $v_x = v(y)$.
Из \eqref{Pu3}, \eqref{Pu4} имеем $\CP py=0,\ \CP pz=0$, то есть $p=p(x)$.

Рассмотрим уравнение \eqref{Pu2}
\[
  \DP{v_x}t = \cancel{\CP{v_x}t} + v_x\CP{v_x}x + v_y\CP{v_x}y + v_z\CP{v_x}z.
\]
Каждое слагаемое равно нулю. Первое из-за стационарности движения, второе по уравнению \eqref{Pu1} третье по предположению $v_y=v_z=0$. Значит, $\DP {v_x}t=0$.

Далее $\Delta v_x = \CP{^2v_x}{x^2} + \CP{^2v_x}{y^2} + \CP{^2v_x}{z^2} = \DP{^2v}{y^2}$ в силу \eqref{Pu1}.

%рисунок ещё разок

Из уравнения \eqref{Pu2} имеем равенство $\DP px = \mu \DP{^2}{y^2}$. Левая часть не зависит от $y$, правая не зависит от $x$. Значит, это "--- равенство некоторых констант. Обозначают $\DP px = \const = -i$ и $p = -i\,x+\const$.

Из граничных условий получаем $i=\frac{p_1-p_2}{x_2-x_1} = \frac{\Delta p}l\ne0$, где $l=x_2-x_1$, а $\Delta p = p_1-p_2$.
Таким образом основное уравнение для определения скорости
\[
  \DP{^2}{v^2} = -\frac{\Delta p}{\mu\,l}.
\]
Получается уже не линейный, а параболический профиль скорости. Давайте его найдём. Интегрируем один раз $\DP vy = -\frac{\Delta p}{\mu\,l}\,y + C$. Интегрируем второй раз
\[
  v= -\frac{\Delta p}{2\,\mu\,l}\,y^2 + C\, y + C_1.
\]
Из граничных условий: при $y=0$ $y=h$ $C_1=0$, при $y=h$ получаем $C = \frac{\Delta p}{2\,\mu\,l}\,h$. Итого профиль скорости имеет вид
\[
  v = \frac{\Delta p}{2\,\mu\,l}(-y^2 + y\,h).
\]
% рисунок профиль скорости
Распределение скорости по периметру потоку называется профилем скорости.

Где $v_{\max}$? Из условия $\DP vy = 0$ получаем $y_{\max} = \frac h2$. А чему он равняется?
\[
  v_{\max} = v|_{y=\frac h2} = \frac{\Delta p}{8\,\mu\,l} h^2.
\]

Вычислим расход на единицу ширины, то есть
\[
  Q = \int\limits_0^h v_x\,dy = \frac{\Delta p}{2\,\mu\,l}\left(-\frac{h^3}3 + \frac{h^3}2\right) = \frac{\Delta p}{12\,\mu\,l}h^3.
\]
Часто пишут вот в таком виде
\[
  \frac{\Delta p}l = \frac{12\,\mu\,Q}{h^3}.
\]

Это течение называется течением Пуазейля, почему? Он был врач, многие тогда были учёными энциклопедическими. Изучал, что произойдёт с кровеносной системой человека, если ширина сосуда из-за стресса увеличится в два раза. Если течение рассматривать в круглой трубе, то получим $h^4$ в знаменателе. Соответственно, уменьшение диаметра сосуда вдвое, давление увеличится в 16 раз, и сердце разовётся. Вывели эти формулы экспериментально.

Теперь я хочу ввести понятие коэффициента сопротивления.
\[
  c_x = \frac{F_x}{\frac12\,\rho\,v^2\,S}.
\]
Очень удобная безразмерная величина. Здесь $v$ "--- скорость тела.

В трубопроводе аналог такого коэффициента обозначают $\lambda = \frac{\Delta p\,h}{\frac12\,\rho\,v_{\text{ср}}^2\,l}$, где $v_{\text{ср}} = \frac Qh=\frac{\Delta p}{12\,\mu\,l}\,h^2$. Можно выразить $\Delta p = \frac{12\,\mu\,l}{Q\,h^2}$ и подставить, получим
\[
  \lambda = \frac{24\,\mu}{\rho\,v_{\text{ср}\,h}} = \frac{24}{\RE}.
\]
Так появляется в рассмотрении число Рейнольдса
\[
  \RE = \frac{v_{\text{ср}}\,h\,\rho}{\mu} = \frac{v_{\text{ср}}\,h}\nu.
\]
\subsection{Течение между двумя плоскостями с подвижно плоскостью и с перепадом давления}
Посмотрим на течение, которое получается при этом. Такое течение часто встречается при обтекании тела.
% рисунок обтекание кружочка

Какой будет профиль скорости? Уравнения для течений Пуазейля и Куэтта линейные. Значит, решения можно складывать.
\[
  v = \frac{v_o}h\,y + \frac{\Delta p}{2\,\mu\,l}\,y(h-y).
\]
Граничные условия $v|_{y=0}=0$ и $v|_{y=h}=v_0$ выполняются.
Есть две ситуации
\begin{enumerate}
\item $\Delta p<0$. Тогда $v = y\left(\frac{v_0}h - \frac{|\Delta p|}{2\,\mu\,l}(h-y)\right)$. Если $|\Delta p|$ маленькое, то просто линейный профиль скорости течения Куэтта изгибается. А если $|\Delta p|$ большое, то есть точка нулевой скорости и область противотечения.
% рисунок профилей (такая буква l)
Часть жидкости, прилегающая к неподвижному телу, движется в противополодную сторону, оттесняет линии тока, двигающиеся в направлении подвижного тела. Это называется отрывным течением. Как с этим бороться? Нужно, чтобы тело было такое, что градиент давления был мал. Если к сфере пристроить хвостик, то можно уменьшить градиент давления.
% рисунок сфера с схвостиком
\end{enumerate}

\subsection{Обтекания и упрощения}
Если к течению прибавляется ускорение, то есть нелинейность, то ни одного аналитического решения. Течения эти очень сложные, даже на компьютерах эти течения тяжело рассчитывать. Пытались упростить уравнения в некоторых случаях. Надо как-то выбросить некоторые члены. Есть два подхода: выбросить нелинейные члены или выбросить слагаемые с вязкостью.

Напишем уравнение Навье"--~Стокса, например, для несжимаемой жидкости в проекции на ось $x$.
\[
\setcounter{vars}{2}
  \matder{v_x} = F_x - \frac1{\rho}\CP px + \nu\left(\Lap{v_x}\right).
\]
Здесь в левой части всё, кроме первого слагаемого "--- нелинейные инерционные члены, в правой части всё, что умножается на $\nu$, "--- вязкие члены. Нельзя сравнивать размерные величины. Введём характерные величины. Характерная величина $v_x$, это такая величина $V$, что $v_x\sim V$ почти во всех точках потока. Характерный линейный масштаб $L$ "--- это расстояние, на котором величина $v_x$ меняется на величину порядка её самой.
% картинка |v_x| от x
Если $\Delta x\sim L$, то $\Delta v_x\sim V$. И отношение $\frac{\Delta v_x}{\Delta x}\sim \frac VL$, а производные $\CP{v_x}x \sim\frac{\Delta v_x}{\Delta x}\sim\frac VL$. Тогда $L = \frac{V}{\CP{v_x}x}$.

Бывает так, что масштабы тела в одном направлении и в перпендикулярном совсем разные.

Введём ещё временной масштаб $T$. Это когда если $\Delta t\sim T$, то (на месте $v_x$ может быть любая величина) $\Delta v_x=\sim V$, то есть $\CP{v_x}{x}\sim \frac VT$.
