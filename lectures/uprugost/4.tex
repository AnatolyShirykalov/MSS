\section{Вязкая жидкость}
Вспомним ещё раз, как обстояло дело с идеальной жидкостью. Для идеальной жидкости тензор напряжений имеет вид $p^{ij} = -p\,g^{ij}$, где $g^{ij}$ в декартовых координатах просто символы Кронеккера. Если подставить это в вектор напряжений, получим
\[
 \ve P_n = p^{ij}\,n_j\ve e_i = -p g^{ij}\, n-J\,\ve e_i = -p n^i\,\ve e_i = -p\,\ve n.
\]
То есть в идеальной жидкости давление действует по нормали к площадке, давление, как в школе, сила, делённая на объём.

В вязкой жидкости имеем
\[
  p^{ij} = -p g^{ij} + \tau^{ij}(e_{\alpha,\beta}).
\]
Здесь $\tau_{ij}$ "--- тензор вязких напряжений. Можно предположить, что этот тензор даст только касательные напряжение, но это не так. Рассмотрим вектор напряжений
\[
  \ve p_n = p^{ij}\,n_j\ve e_i = (-p\,g^{ij}+\tau^{ij})n_j\,\ve e_i = 
  -p\,\ve n + \tau^{ij}\,n_j\,\ve e_i.
\]
Тогда $p_{nn} = \ve P_n\cdot \ve n = -p+\underbrace{\tau^{ij}\,n_j\,n_i}_{\ne0}$.

Далее, вспомним, что $e_{ij} = \frac12(\nabla_i v_j+\nabla_j v_i)$. Первый инвариант этого тензора
\[
  I_1(e) = g^{ij}e_{ij} = \frac12(g^{ij}\nabla_i v_j + g^{ij}\nabla_j v_i) = 
  \div\ve v.
\]
Давайте словаи скажем, что такое $e_{11}$. Это скорость относительного удлиненния координатного волокна, направленного по координате $x_1$. $e_{12}$ отвечает за скос между координатными волокнами.

\subsection{Закон Навье"--~Стокса}
Нам надо определить соотношение между $\tau^{ij}$ и $e_{\alpha,\beta}$. Самый простой вид соотношений, естественно, линейный: $\tau^{ij}(\e_{\a,\beta}) = A^{ijkl}e_{kl}$. При этом любым такое соотношение быть не может и общий вид из физических соображений задаётся двумя коэффициентами
\[
  \tau^{ij} = \l\,\div\ve v\,g^{ij} +2\,\mu\,e^{ij}.
\]
При этом тезор скоростей деформаций можно разложить на шаровой и девиатор $e^{ij}_d = e^{ij}-\frac13 I(e)g^{ij}$, его первый инвариант равен нулю. Разложение имеет вид
\[
  e^{ij} = \frac13 I_1(e) g^{ij} + e^{ij}_d.
\]
Первое слагаемое называется изменением объёма, а второе деформацией сдвига (без изменения объёма). В этом виде закон Навье"--~Стокса запишем
\[
 \tau^{ij} = \underbrace{(\l+2/3\mu)}_\zeta\div\ve v\,g^{ij} + 2\,\mu\,e^{ij}_d.
\]
Здесь $\zeta$ "--- объёмный коэффициент вязкости.

Ещё вводят другой параметр $\nu = \frac\mu\rho$ "--- кинематический коэффициент вязкости. В $\{L,M,T\}$ имеем $[p^{ij}] = \frac{M}{L\,T^2}$, $[e^{ij}] = \frac1T$. Тогда
\[
[\mu] = \frac M{L\,T};\qquad [\nu]=\frac{L^2}T.
\]
Это обоснование того, почему один из них называется динамическим (в нём есть масса).

Почти во всех задачах $\zeta$ принимается равным нулю, тогда $\l = -\frac23\mu$ и остаётся один коэффициент вязкости.

Итак выведем уравнения Навье"--~Стокса. Объединяем закон Навье"--~Стокса и общий закон движения
\[
  p^{ij} = -p\,g^{ij}+\l\,\div\ve v\,g^{ij} + 2\,\mu\,e^{ij};\quad
  \rho a^i = \rho\,F^i +\nabla p^{ij}.
\]
Пусть $\l,\mu=\const$. Тогда
\begin{multline*}
  \nabla_jp^{ij} = \nabla_j(-p\,g^{ij}) + \nabla_j(\l\,\div\ve g^{ij}) + \nabla_j(2\,\mu e^{ij}) = 
  -(\grad p)^i + \l\big(\grad (\div\ve v)\big)^i +
  \mu\nabla_j\nabla^i v^j + \mu\,\nabla_j\nabla_jv^i =\\=
  -(\grad p)^i + \l\big(\grad (\div\ve v)\big)^i +
  \mu\nabla^i\nabla_j v^j + \mu\,\nabla_j\nabla_jv^i =
  \mu \nabla^i\div\ve v + \mu(\Delta \ve v)^i.
\end{multline*}

Итого, получаем
\[
 \rho\DP{\ve v}t = \rho\ve F - \grad p + (\l+\mu)\grad\div\ve v + \mu\Delta\ve v.
\]

Важно то, что $\l,\mu$ предположили константы. Но они могут зависеть, например, от температуры. Замёрзшее масло в автомобиле вязкое, а температура может зависеть от координат и так не получится. Выписывать в общем случае не получится, понятно, как это делается.

\subsection{Граничные условия}
\subsubsection{Условие прилипания}
В уравнении Навье"--~Стокса можно положить $\l,\mu=0$, уравнение будет идентичным уравнению Эйлера. Но задача всё равно будет другая, отличная от задачи с идеальной жидкостью. Условия на границе было условием непроницаемости, а проскальзывать вдоль границы жидкость могла.

Условие принипания $\ve v_{\text{на пов. тела}} = \ve v_{\text{тела}}$.

\subsubsection{Свободная граница}
Если свободная граница записывается соотношением $f(x,y,z,t)=0$, то конематическое условие имеет вид
\[
  \CP ft + \ve v\grad f.
\]
Скорость нормального перемещения границы, как геометрического объекта, должна совпадать со скоростью частиц жидкости.

Динамическое условие $\ve P_{n\,\text{на границе}} = \ve P_{n\,\text{внешнее}}$.


\subsection{Вязкая несжимаемая жидкость. Замкнутая система уравнений}
Чтобы написать замкнутую систему, нужно чтобы число уравнений совпадало с числом неизвестных.

Определение несжимаемой жидкости такое $\DP \rho t=0$ (но не значит, что жидкость однородная. Можно взять воду и керосин в одной задаче). Тогда из уравнения неравзрывности имеем $\div v=0$. Из-за этого уравнение Навье"--~Стокса упрощается.
\begin{eqnarray*}
\rho\DP{\ve v}t &=& \rho\ve F - \grad p + \mu\Delta v;\\
  \div \ve v &=& 0\\
  \DP pt &=& 0.
\end{eqnarray*}
Неизвестные $\ve v,p,\rho$. 5 уравнений, 5 неизвестных. Если плостность заранее известна, то последнего уравнения нет.

\subsection{Отличие граничных условий от случая идеальной жидкости}
Ещё раз их подчеркнём. Вы наверняка решали такие задачи: идеальная жидкость, течение потенциальная. Что это такое? $\ve  = \grad\phi$. Из уравнения неразрывности 
\[
0=\div\ve v = \div\grad\phi = \Delta \phi/
\]
Наверняка решали задачу обтекания тела. Надо было решать уравнение Лапласа $\Delta\phi=0$ с условиями $v_n = \CP\phi n\Big|_{\Sigma}=0$, $(\grad\phi)_\infty =v_{\infty}$. Решение такое даёт, что $v_n|_{\Sigma}=0$, а $v_\tau\big|_\Sigma\ne0$.

А как с вязкой жидкостью?
\[
 \mu\Delta\ve v = \mu(\underbrace{\grad\div\ve v}_0 - \underbrace{\rot\rot\ve v}0)=0,
\]
так как $\rot\ve v = \rot\grad\phi=0$.

То есть снова получим уравнение Лапласа в случае задачи отыскания потенциального течения несжимаемой вязкой жидкости. Но краевые условия другие, гораздо более сложные. 

Вывод: различия не только в уравнениях и это очень важно.
\subsection{Уравнение притока тепла}
То, что мы с вами обсуждали, было у вас уже ранее. А теперь вспоминаем термодинамику. Первый закон термодинамики говорит, что есть такая функция полная энергии, которая разделяется на кинетическую и на внутреннюю и изменение выражается $d\mathcal E = d U + E_{\text{кин}} = dA^{(e}+dQ$ (можно и подробнее написать, но мы не будем). Можно написать теорема живых сил $dE_{\text{кин}} = dA^{(e)} + DA^{(i)}$ и уравнение притока тепла $d U = - dA^{(i)} = dQ$. Введём плотности $du = \DP um$ и $dq = \DP Qm$, получаем уравнение притока тепла
\[
  du = -\DP{A^{(i)}}m + dq.
\]
Если тензор $p^{ij}$ симметричный, что почти всегда и бывает, то
\[
  \DP{A^{(i)}_{\text{пов}}}m = -\frac1\rho p^{ij}\,e_{ij}\,dt = 
  -\frac1\rho(-p\,g^{ij} + \tau^{ij})\,e_{ij}\,dt = 
  \frac p\rho g^{ij}\,e_{ij}\,dt
 - \frac1\rho\tau^{ij}\,e_{ij}\,dt.
\]
Отметим, что
\[
  \frac p\rho g^{ij}\,e_{ij}\,dt = 
 \frac p\rho I_1(e)\,dt = \frac p\rho\div\ve v\,dt = -\frac p\rho\frac{dp}\rho = pd\frac1\rho.
\]
И из уравнения неравзрывности получаем, что
\[
  du = -pd\frac1\rho + \frac1\rho \tau^{ij}\,e_{ij}\,dt +dq.
\]
Именно такое уравнение притока тепла вы и знаете, наверное.

Теперь давайте вспомним второй закон термодинамики
\[
  T\,ds = dq + dq',\quad dq'\ge0.
\]
Отсюда $dq = T\,ds - dq'$. Тогда
\[
  du = -p\frac1\rho + \frac1\rho\tau^{ij}\,e_{ij}\,dt + T\,ds - dq'.
\]
Мы будем считать, что тождество Гиббса
\[
  du =-p\frac1\rho + T\,ds
\]
в вязкой жидкости тоже должно выполняться. Чтобы для покоящейся жидкости и для малых скоростей всё было логично.

Тогда $\frac1\rho\tau^{ij}\,e_{ij}\,dt$ "--- работа вязких напряжений (со знаком минус), "--- эта величина есть $dq'$ и должна быть $\ge0$. Делаем частный вывод: работа вязких напряжений (которые отрицательны) даёт уменьшение кинетической энергии.

Вам наверняка вводили вектор потока тепла $\ve q$ "--- сколько тепла протекает через поверхность (от объёма). Тогда
\[
  dQ = -\Gint\Sigma\ve q\ve n\,d\sigma\,dt = -\Gint V\div\ve q\,dV\,dt.
\]
Можно ввести плотность 
\[
  dq = \DP Qm = \frac1\rho\DP QV = -\frac{\div\ve q\,dV\,dt}{\rho\,dV} = -\frac\rho\div\ve q\,dt.
\]

А помните закон Фурье? При $\kappa=\const$ имеем
\[
  \ve \q = -\kappa\grad Tl\quad \div\ve q = \div(-\kappa\grad T) = -\kappa\Delta T.
\]
Таким образом, $dq = -\frac\kappa\rho\Delta T\,dt$. Теперь можно записать настоящее уравнение притока тепла
\[
  \DP ut = -p\DP{(1/\rho)}t + \frac1\rho\tau^{ij}\,e_{ij} + \frac\kappa\rho\Delta T.
\]

\subsection{Уравнения для распространения тепла в вязкой несжимаемой однородной жидкости}
Будем считать, что $u=c\,T + \const$, где $c$ "--- удельная теплоёмкость, пусть она постоянна. В наших предположенгиях $\tau^{ij} = 2\mu\,e^{ij}$. Тогда запишем полную систему уравнений
\begin{eqnarray*}
  \rho\DP{\ve v}t &=& \rho\ve F - \grad p + \mu\,\Delta\ve v;\\
  \div\ve v&=&0;\\
  c\DP Tt &=& \frac{2\,\mu}{\rho}e^{ij}e_{ij} + \frac\kappa\rho\Delta T.
\end{eqnarray*}
Пять уравнений, пять неизвестных: $\ve v,\rho,T$. Но первые четыре уравнения составляют замкнутую систему на $\ve v$, $p$. Распределение скоростей не зависет от температуры, но температура зависит от распределения скоростей. Но если я хочу всё-таки учесть зависимость вязкость от температура, я не могу использовать такой вид уравнения Навье"--~Стокса.
\subsubsection{Уравнение теплопроводности}
Вот вы знаете классическое уравнение теплопроводности $\CP Tt = a^2\Delta T$. Оно имеет какое-то отношение тому, что мы делаем? Пусть $\ve v=0$, $\rho=\const$. Тогда последнее уравнение из замкнутой системы
\[
  c\DP Tt = \frac\kappa\rho\Delta T.
\]
Почти уравнение теплопроводность, только производная не та, но жидкость покоится и полная производная совпадает с частной. Тогда $a^2 = \frac\kappa{\rho\,c}$.

\subsection{Пример замкнутой системы уравнений для вязкого совершенного теплопроводного газа}
Опять те же предположения, я их не обговариваю.
\begin{eqnarray*}
\rho\DP{\ve v}t &=& \rho\ve F -\grad p+ (\l+\mu)\grad\div\ve v + \mu\,\Delta\ve v;\\
\DP\rho t + \rho\,\div v&=&0;\\
\DP ut &=& -p\DP{(1/\rho)}t + \frac1\rho\tau^{ij}e_{ij} + \frac\kappa\rho\Delta T;\\
\tau^{ij} &=& \l\div\ve v\,g^{ij} + 2\,\mu e^{ij};\\
p &=& \rho\,T\,T;\\
u &=&c_v\,T + u_0.
\end{eqnarray*}
Семь уравнений, семь неизвестных $\ve v,u,\rho,p,T$

Если есть температура, то нужны и граничные условия для температуры. 
\begin{enumerate}
\item $T_{\text{пов.}} = T_{\text{заданн}}$;
\item теплоизволяция. На стенке $\ve q\ve q = q_n = -\kappa\CP Tn=0$.
\end{enumerate}
\subsection{Простая задача: течение Куэтта}
На воду положили фанерку и начали её тянуть с горизонтальной скоростью. Задача у нас плоская. Слой толщины $h$. Если на идеальную жидкость положил, то ничего не происходит: фанерка сама по себе, жидкость сама по себе. Но давайте считать жидкость вязкой и использовать условие прилипания. Будем считать, что 
\begin{enumerate}
\item $\ve v=(v_x,v_y)=( u,0)$;
\item $\CP{\ve v}t = 0$, значит, $ u =u(x,y)$.
\item несжимаемая, вязкая, однородная.
\item $\ve F=0$ (если это не так, то можно легко свести эту задачу к нашему случаю, просто появляется ещё гидростатическое распределение давления)/
\end{enumerate}
Очень быстро пишу систему уравнений
\begin{eqnarray*}
  \rho \DP{\ve v}t &=& -\grad p + \mu\,\Delta\ve v;\\
  \div v &=&0;\\
  \rho\DP ut &=& -\CP px + \mu\left(\CP{^2u}{x^2} + \CP{^2u}{y^2}\right);\\
  0&=&\CP py + 0;\\
  \CP ux &=& 0.
\end{eqnarray*}
У нас получается, что $p=p(x)$, $\DP ut = \CP ut + u\CP ux = 0$ и $u=u(y)$. Значит,
\[
  \CP px = \mu\CP{^2u}{y^2}.
\]
Функции разных переменных равняются, значит, они константы. Обозначим $\CP px = -i=\const$. Это перепад давления вдоль этого канала (фанерка ограничивает этот канал вместе со дном). Примем, что $i=0$, что нет перепада давления, мы насос не включаем. И что мы с вами получим:
\[
  \DP{^u}{y^2} = 0;\quad u = A\,y+ B.
\]
Используем граничные условия. На дне жидкость прилипает $u(0)=0$, $u(h) = u_{\infty}$. Тогда $B=0$, а $A = \frac{u_\infty}h$, то есть $u = \frac{u_\infty}hy$. Мы всё нашли. Распределение давление константа, рапсределение скоростей линейной.

Можно ещё посмотреть трение на нижней стенке.
\[
  \tau_{xy} = \mu\,e_{xy} = 2\mu\frac12\left(\CP{v_x}y + \CP{v_y}x\right) = \mu\CP uy  = \mu\frac{u_\infty}h =\tau_h.
\]
Эта величина вообще константа в любом слое, не только на дне.

Для получения этого решения мы использовали очень сильное предположение: скорость только горизонтальная. Это верно только для ламинарных течений.
