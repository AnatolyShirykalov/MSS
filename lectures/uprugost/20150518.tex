\section{Лекция 25}
\begin{enumerate}
\item Поверхность нагружения и поверхность текучести.
Условие  (критерий) пластичности или условие текучести.
Нагружение и разгрузка (уже говорили, нужно теперь формулу написать).
\item Различные (популярные) поверхности нагружения (или текучесои).
Условие пластичности Треска и Мизеса.
\item Определяющие соотношения в теории пластичности.
Деформационные теории и теории течения. Ассоциированный закон.
\item Уравнения терии Прандтля"--~Рейсса.
\end{enumerate}

Вспомним, что мы делали. Рисовали несколько разных диаграм.
В сегодняшней  и следующей лекции мы не учитываем зависимость от температура. На  самом деле везде в любом материале $\sigma_*(\e_{11}^p,T)$. Но если вы знаете диапазон температур, в котором будет работать ваш материал, вы при этой температуре измеряете $\sigma_*$ и работаете. Чтобы работать в другой температуре, измеряете ещё раз.
