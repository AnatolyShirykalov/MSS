\section{Лекция 15}
\begin{enumerate}
\item Решение задачи Ламе и его исследование;
\item Термодинамические соотношения в теории упругости.
\end{enumerate}

Вот такой короткий план на сегодня.
% рисунок:
% оси 135-135-90 z вправо;
% Два цилиндра с осью z, прорисованы сечения
% Подпись $r,\theta,z$

%рисунок
% две концентрических окружностей
% label $P_a$ внутри внутренней
% label $P_b$ вне внешней;
% подпись снизу $r=a\\r=b$
% стрелка из центра O вправо вниз
% ещё одна окружность между старыми
% стрелка из точки пересечения уже нарисованной стрелка и новой окружностью по касательной к новой окружности

Повторим кратко математическую постановку задачи: уравнения Навье"--~Ламе
\[
  (\lambda+\mu)\grad\div \ve w +\nu\Delta \ve w = 0.
\]
И граничные условия
\begin{itemize}
\item при $r=a$ условие $\ve P_n = -p_a\ve n$;
\item при $r=b$ условие $\ve P_n = -p_b\ve n$
\item при $z=\in\{0,l\}$ условие $w_n=0$, $\ve P_{n\tau}=0$ (не запрещается перемещаться в поперечном направление, но в продольном есть ограничители).
\end{itemize}

Предполагается, что $w_z=0$, $W_2=0$, а $w_1=w = w(r)$. Тогда
\[
  \div\ve w = \CP wr + \frac wr;\quad
  \Delta w_1 \CP{ }r\left( \CP wr + \frac wr \right) = \CP{^2w}{r^2} + \frac1r\CP wr - \frac w{r^2};\quad
  \Delta w_2 = \Delta w_3=0.
\]
Решение задачи есть $w = Ar+\frac Br$, где $A,B$ "--- константы, которые зависят от граничных условий.

Ещё мы вычислили
\[
 \e_{11} = \nabla_1w_1 =\CP wr = A-\frac B{r^2};\quad
 \e_{22} = \nabla_2w_2 = wr = A r^2 + B;\quad
 \e_{33} = \e_{ij}=0\text{ при } i\ne j.
\]

Ну и компоненты тензора напряжений
\[
  p_{ij} = \lambda J_1(\e)g_{ij} + 2\mu \e_{ij};\quad
  \div\ve w = 2A.
\]
Тогда
\[
  P_{11} = 2A\lambda + 2\mu\left( A-\frac B{r^2} \right) = 2A(\lambda+\mu) - \frac{2\mu B}{r^2}.
\]

Обозначим $A_1 = 2A(\lambda+\mu)$, $B_1 = 2\mu B$. Тогда
\begin{eqnarray*}
  P_{11} &=&  A_1 - \frac{B_1}{r^2};\\
  P_{22} &=&  2A\lambda r^2 + 2\mu r\left( A r + \frac Br \right) = A_1 r^2 + B_1;\\
  P_{33} &=& 2 A\lambda.
\end{eqnarray*}

Мы на этом и остановились. Теперь мы должны обратиться к граничным условиям и написать их так, чтобы константы можно было по ним определить. Записать через $w$.
\begin{itemize}
\item При $r=a$ имеем $P_{ni}= - p_a n_i$. По формуле Коши $P_{ni} = p_{ik}n^k$, то есть
\[
  p_{ik} n^k = -p_a g_{ij} n^ki,\quad i=1,2,3
\]

% рисунок
% две концентрические окружности
% внутри внуренней подпись $p_a$ и радиальные стрелки к окружности

Имеем $\ve n = n^1\ve{\text{э}}_1$, $n^2=n^3=0$, $n^1 = -1$. Значит,
\[
  -P_{11}|_{r=a} = p_a\imp P_{11}|_{r=a} = -p_a;\qquad P_{12}|_{r=a}=0;\qquad p_{13}|_{r=a}=0.
\]

\item Условие при $r=b$ будет почти таким же. Только нормаль будет с плюсом по радиусу
\[
  P_{11}|_{r=b} = -p_b;\qquad P_{12}|_{r=b} = P_{13}|_{r=b}=0.
\]
\item Граничное условие на торцах $z\in\{0,l\}$:
\[
  w_3|_{z\in\{0,l\}} = 0,\quad
  P_{13}|_{z\in\{0,l\}} = P_{23}|_{z\in\{0,l\}} = 0.
\]
\end{itemize}

Условия на торцах уже выполнены автоматом в силу нашего предположения. Я не дописала, что $P_{ij}=0$ при $j\ne i$ на всём решении. Условия на границах тоже частично выполнены. Осталось добиться выполния двух
\[
  P_{11}|_{r=a} = -p_a;\quad P_{11}|_{r=b} = -p_b.
\]
Получаются простенькие уравнения
\[
\begin{cases}
  A_{1} - \frac{B_1}{a^2} = -p_a;\\
  A_1 - \frac{B_1}{b^2} = -p_b.
\end{cases}
\]

Какие же формулы получаются
\[
B_1 = \frac{a^2b^2(p_a-p_b)}{b^2-a^2};\quad
A_1 = \frac{a^2 p_a - b^2 p_b}{b^2-a^2}.
\]

Собственно говоря, решение найдёно.
\begin{eqnarray*}
  P_{11} &=&  A_1 - \frac{B_1}{r^2};\\
  P_{22} &=& A_1 r^2 + B_1;\\
  P_{33} &=& 2\lambda A = \frac{\lambda}{\lambda+\mu} A_1 = 2\sigma A_1.
\end{eqnarray*}
Чтобы ощутить силы, нам нужны физические компоненты, так как $|\ve{\text{э}}_i| = \sqrt{g_{ij}}$.
\[
  \mathcal P = p_{11}\eg^1\eg^1 + p_{22}\eg^2\eg^2 + p_{33}\eg^3\eg^3 = 
  p_{11\text{физ}}\ve e^1\ve e^1 + 
  p_{22\text{физ}}\ve e^2\ve e^2 + 
  p_{33\text{физ}}\ve e^3\ve e^3 .
\]

Обозначения будут следующими
\[
  P_{11} = P_{11\text{физ}} = P_{rr};\quad
  P_{22\text{физ}} = P_{22}\frac1{r^2} = P_{\theta\theta};\quad
  P_{33\text{физ}} = P_{33} = P_{zz}.
\]

Записшем ответ
\begin{eqnarray*}
  P_{rr}&=& \frac{a^2 p_a}{b^2-a^2}\left( 1-\frac{b^2}{r^2} \right) - \frac{b^2 p_b}{b^2-a^2}\left( 1-\frac{a^2}{r^2} \right);\\
  P_{\theta\theta} &=& \frac{a^2 p_a}{b^2-a^2}\left( 1+\frac{b^2}{r^2} \right) - \frac{b^2p_b}{b^2-a^2}\left( 1+\frac{a^2}{r^2} \right);\\
   P_{zz} &=&  2\sigma\frac{a^2p_a - b^2p_b}{b^2-a^2}.
\end{eqnarray*}
Здесь $\sigma$ "--- коэффициент Пуассона, который вычисляется по формуле $\sigma = \frac{\lambda}{2(\lambda+\mu)}$.

\subsection{Исследуем это решение}
Сначала рассмотрим ситуацию, когда внешнее давление мало по сравнению с внутренним. Так в случае пушки. Пренебрежём $p_b$, то есть полагаем $p_b=0$. Тогда формулы становятся более короткими
\begin{eqnarray*}
  p_{rr} &=& \frac{a^2p_a}{b^2-a^2}\left( 1-\frac{b^2}{r^2} \right)<0;\\
  p_{\theta\theta} &=& \frac{a^2p_a}{b^2-a^2}\left( 1+\frac{b^2}{r^2} \right)>0;\\
  p_{rr} &=& 2\sigma\frac{a^2p_a}{b^2-a^2}>0.
\end{eqnarray*}
Если не сажать внутрь маленьких чёртиков, которые будут на себя тянуть, то давление внутри трубы $p_a>0$.

%рисунок
% штрихованый прямоугольник: ширина больше высоты;
% слева направо стрелка к левой стенке бруса;
% слева направо стрелка от правой стенки бруса посередине высоты бруса;
% справа налево стрелка от правой стенки бруса ниже той стрелки;

Под действием $P_{rr}$ происходит сжатие вдоль $r$, под действием $P_{\theta\theta}$ происходит растяжение вдоль $\theta$, под действием $P_{zz}$ растяжение вдоль $z$.

Что опаснее: растяжение или сжатие? Бетон слабее на растяжение, чем на сжатие; у стали примерно одинаково. Как правило, растягивающее напряжение опаснее.

У нас какое из растягивающий по модулю больше? $2\sigma<1\imp |P_{\theta\theta}|>|P_{zz}|$, и $P_{\theta\theta}$ "--- самое опасное напряжение.
$P_{\theta\theta}$ максимально при $r=a$, то есть на внутренней стенке
\[
 P_{\theta\theta\max} = p_a\frac{b^2+a^2}{b^2-a^a}.
\]
Вот эта величина должна быть меньше, чем предел прочности.
% рисунок
% две концентрические окружности
% профиль напряжений $P_{\theta\theta}$

Сами напряжения почти ($P_{zz}$ зависит слабо) не зависят от материала. Но у материалов разные пределы прочности.

В последнее время стало модно исследовать материалы с отрицательным коэффициентом Пуассона. Многие кристаллические материалы так устроены. Такие материалы называются ауксетиками.
% ри

$P_{\theta\theta}$ можно уменьшить увеличением $P_b$. Для этого артиллеристы делают составную трубу. Трубу меньшего радиуса нагреваем, а большего радиуса охлаждаем. И надеваем маленькую трубу на внешнюю. Внешняя труба после охлаждения становится напряжённой.

Следующий вопрос такой. Как влияет толшина трубы? Если $b\to\infty$, то $P_{\theta\theta}\to p_a$. А при стремлении к $b\to a$, то $P_{\theta\theta} \to\infty$.
Ничего меньше чем $p_a$ не получится.
% рисунок типа гипербола такая на $P_{\theta\theta}$.

Ещё у задачи Ламе есть решение для сферы. Но я её делать не буду. Мы разобрали одну из простейших. А если сложные, то они и решения имеют довольно сложные.

\subsection{Термодинамические соотношения в теории упругости}
Хочу написать соотношения, которые учитывают ещё температурные напряжения. Ведь от взыва пороха внутри трубы газы горячие. И влияние температуры очень важно.

Пишем первый и второй законы термодинамики в дифференциальной форме (уравнение притока тепла и уравнение для энтропии)
\begin{eqnarray*}
  du &=& -\frac1{dm} d A^{(i)} + dq + dq^{**},
\end{eqnarray*}
$u$ "--- плотность внутренней энергии, $dq$ "--- приток тепла к единице массы, $dq^{**}$ "--- приток энергии, отличный от работы сил и тепла.

В классической теории упругости $dq^{**} =0$ (в моментной теории упругости $dq^{**}$ "--- работа пар.
