\section{Вторая неделя февраля}
\begin{enumerate}
\item Распространение конечных возмущений в сжимаемой идеальной жидкости (газе). Волны Римана.
Эффект опрокидывания волн. Возникновение ударных волных.
\item  Ударные волны в сжимаемых идеальных жидкостях. Условия на ударных волнах. Возрастание энтропии при переходе частиц через ударную волну. Теорема Цемплена.
\end{enumerate}

\subsection{Конечные возмущения}
Конечные, значит, не малые. Будем иметь в виду следующие предположения о течении.
\begin{enumerate}
\item Жидкость идеальная сжимаемая.
\item Массовые силы не учитываются.
\item Движение баротропное, $\rho=\rho(p)$ или $p = p(\rho)$ (например, адиабатическое движение совершенного газа, когда энтропия всех частиц одинакова, то есть $p=C\,\rho^\gamma$, $C = A\,e^{\frac s{c_v}}=\const$)
\item Рассматриваем движение с плоскими волная, то есть
\[
  v_x = v(t,x),\quad v_y=v_z = 0,\quad \rho=\rho(x,t),\quad p=p(t,x).
\]
\item Ищем решения специального вида: $v = v(f),\ \rho=\rho(f),\ p=p(f)$, где $f=f(t,x)$.
\end{enumerate}
Последнее условие определяет волны Римана. Другое название "--- простые волны.

Оказывается, такие решения уравнений газовой динамики существуют.

Условие 5 равносильно условию $v=v(\rho)$, ну а $p$ и так есть $p(\rho)$.

Напишем систему уравнений («газовой динамики») для движения сжимаемой идеальной жидкости с плоскими волнами: уравнение неразрывности и проекция на ось $x$ уравнения Эйлера.
\begin{eqnarray*}
  \CP\rho t + v\CP\rho x + \rho \CP vx&=&0;\\
  \CP vt + t\CP vx + \frac 1\rho\CP px&=&0.
\end{eqnarray*}
Переобозначим последнее слагаемое $\frac1\rho\DP p\rho\CP\rho x =\frac{a^2}\rho\CP\rho x$. Мы теперь не будем пренебрегать вторыми слагаемыми, но зато $v=v(\rho)$.

\begin{eqnarray*}
  \CP \rho t + v\CP\rho x + \rho\DP v\rho\CP\rho x&=&0;\\
  \DP v\rho \CP\rho t + v\DP v\rho\CP\rho x + \frac{a^2}\rho\CP\rho x&=&0.
\end{eqnarray*}
Можно эту систему рассмотреть относительно неизвестных $\CP\rho t$ и $\CP\rho x$. Система однородная, значит, чтобы были нетривиальные решения, определитель должен равняться нулю.
\[
  \begin{vmatrix}
    1 & v + \rho\DP v\rho\\
  \DP v\rho & v\DP v\rho + \frac{a^2}{\rho}
\end{vmatrix}=0;\qquad \frac{a^2}\rho = \rho\left(\DP v\rho\right)^2;
\]
Значит, возможно два типа решений $v = \pm\int\frac a\rho\,d\rho$. Говорят о двух типах волн Римана.
\begin{roItems}
\item Первый тип, где $v = \int\frac a\rho\,d\rho$. Тогда $\DP v\rho =\frac a\rho$
\[
  \CP \rho t + (v+a)\CP\rho x;
\]
Это уравнение уже надо решать.
\item Второй тип, где $v = -\int\frac{a}{\rho}\,d\rho$, $\DP v\rho =-\frac a\rho$.
\[
  \CP\rho t + (v-a)\CP\rho x.
\]
\end{roItems}

Эти решения нельзя складывать, так как уравнения не линейные. Но решения ведут себя похожим образом. Рассмотрим первый тип, то есть
\[
  \DP v\rho = \frac{a}{\rho}.
\]

Я хочу переписать уравнение $\CP\rho t + (v+a)\CP\rho x=0$ вот в таком виде: $\DP\rho t=0$, где $\DP {}t = \CP{ }t + (v+a)\CP{ }x$ "--- производная вдоль линии $\DP xt = v+a$ (если бы $\DP xt = v$, то это была бы материальная производная, то есть вдоль траектории).

На линии $x(t)$ имеем $\rho=\rho(t,x) = \rho\big(t,x(t)\big)$. Таким образом на линии $\DP xt= v+a$ плотность не меняется, а значит и $v+a$ на этой линии не меняется. Значит, $x(t)$ "--- это просто прямая. Существуют линии (они имеют попарно разные наклоны), вдоль которых плотность сохраняется. Уравнение на эти линии есть $x - (v+a)\,t =\const$, но константа есть функция $\rho$, поэтому иногда пишут $x-(v+a)\,t = F(\rho)$.

Итак, что у нас получилось. В решении Римана если $x-(v+a)\,t = \const$, то $\rho = \const$. И меняться плотность может только при переходе от одной линии к другой. Это значит, что
\[
  \rho = \rho\big(x-(v+a)\,t\big);\quad v = v\big(x - (v+a)\,t\big).
\]
Это не тривиальные уравнения, справа функции от $\rho$.

Как же найти это решение, и каково поведение этого решения? Очень по форме похоже на бегущую волну. Пусть, например, при $t=0$ распределение плотности задано какое-то. Что будет при $t_1>0$? Чтобы в $t_1$ была такая же плотность, как в момент $t_0=0$ в точке $x_0$, нужен $x = x_0 + (v+a)\,t_1$. Разные значения плотност и будут переноситься с разными скоростями. Поэтому форма волны будет деформироваться. Большая плотность переносится быстрее, чем малая.

Если при $t=0$ распределение плотности задано. Можно найти наклон линий $x - (v+a)\,t = \const$. Вдоль каждой линии плотность сохраняется.

Как конкретнее зависит $v+a$ от плотности и вообще, что это за волна?
Исследуем такую величину 
$\DP { }{\rho}(v+a)$
на знак. Это равно
\[
  \DP { }{\rho}(v+a) = \DP v\rho + \DP a\rho = \frac a\rho + \frac1{2a}\DP{^2p}{\rho^2}\equiv
\frac1{2\,a\,\rho^4} = \DP{^2p}{V^2},
\]
где $V=\frac1\rho$ "--- удельный объём. Доказывается непосредственной выкладкой.
\[
  \DP{p}{\rho} = \DP pV\DP V{\rho} = -\frac1{\rho^2}\DP pV;\quad
  \DP{^2p}{\rho^2} = \frac{2}{\rho^3}\DP pV + \frac1{\rho^4}\DP{^2p}{V^2} = 
  -2\,\frac{a^2}{\rho} + \frac1{\rho^4} = -\frac{2\,a^2}\rho + \frac1{\rho^4}\DP p{V^2}.
\]
Всё зависит от знака во этого $\frac1{2\,a\,\rho^4}\DP{^2p}{V^2}$, что зависит только от самого газа и процесса.

Если газ или жидкость и процесс таковы, что $\DP{^2p}{V^2}>0$, то б\'{о}льшие значения плотности распространяются с б\'{о}льшей скоростью. Эта ситуация типична. В частности для адиабатических процессов. Газы с таким свойством частно называют «нормальным».

Пример адиабатического движения идеального совершенного газа $p=C\,\rho^\gamma = C\,V^{-\gamma}$ "--- адиабата Пуассона. Какая-то гипербола степени $\gamma$. Здесь $\DP{^2p}{\rho^2}>$. Даже по выпуклости кривой можно понять.

Пусть в начальный момент задано распределение плотности, $v>0$, движется направо вся жидкость, $a+v>v>0$. Это означает, что распределение плотности движется обгоняя частицы. Часть волны, где слева направо (по $x$), плотность падает (в начальный момент времени), называется волной сжатия. Плотность именно частиц будет увеличиваться. Часть же, где слева направо плотность возрастает, называется волной разрежения.

Пусть $\DP{^2p}{V^2}>0$. Происходит вот что. Большее значение плотности продвигается далеко, а меньше недалеко за один промежуток времени. Верхушка распространяется с большей скоростью, чем остальная часть волны. Волна разрежения становится положе, а волна сжатия "--- круче. В конце концов верхушка обгоняет хвост, нижняя часть отстаёт от верхушки. Физически такое отставание не осуществимо, так как невозможно в одной точке иметь три различных значений плотности. Это явление называется опрокидывание волны. Оно очень похоже на опрокидывание волны воды. На самом деле область неоднозначности заменяется на разрыв. Нужно делить решение на две непрерывные части и на скачок, расчитываемый из условий на поверхности разрыва.

Чем быстрее толкнуть поршень, тем быстрее опрокинется волна и возникнет ударная волна. Поршень можно и выдвигать, тогда будет волна расширения, которая она не опрокидывается.

Получается, что даже если начально решение было гладкое, могут уже в модели идеальной жидкости, появиться поверхности разрыва.

\subsection{Ударные волны}
Где прочитать про поверхности разрыва.
\begin{enumerate}
\item Седов МСС, том первый.
\item Ландау и Лифшиц «Гидродинамика».
\item Чёрный «Газовая Динамика».
\item МСС в задачах, там очень много задач про поверхности разрыва, в котором есть кое-что, что было изучено позже, чем были написаны указанные выше книги.
\end{enumerate}

Давайте напишем условия на поверхностях разрыва в сжимаемых идеальных жидкостях при условии, что движение адиабатическое. С одной стороны от поверхности разрыва значения переменных будем обозначать цифрой «1», с другой "--- «2». Если скоростью разрыва ноль, то условия на разрыве (закон сохранения массы, количества движения, энергии)
\begin{eqnarray*}
  \rho_1\,v_{n\,1} = \rho_2\,v_{n_2} &=&j\ (\equiv m);\\
  j\,(\ve v_2 - \ve v_1) &=& \ve P_{n_2}-\ve P_{n_1};\\ 
  j\left(\frac {v_2^2}2 + u_2 - \frac{v^2_1}2 - u_1\right) &=& (\ve P_{n_2}\cdot\ve v_2) - (\ve P_{n_1}\cdot\ve v_1) - q_{n_2} + q_{n_1}.
\end{eqnarray*}
Последние два слагаемый "--- приток тепла, если он есть.
