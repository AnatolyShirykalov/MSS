\section{Лекция 19}
\begin{enumerate}
\item Температурные напряжения в стенках трубы при наличии разности температур внутри и вне трубы (продолжение);
\item Плоские задачи теории упругости. Функция Эри.
\end{enumerate}

Теперь давайте вспомним задачу, которую мы решали. Как выглядела задача о трубе?
% рисунок
% труба сбоку длины $l$
% поперечное сечение с подписью температур $T_0$ и $T_1>T_0$.

Имеется круглая труба длиной $l$ круглого поперечного сечения. Температура снаруже $T_0$, внури $T_1$, причём $T_1>T_0$ и $T_0$ "--- та температура, при которой в отсутствие сил нет деформаций.

 Работаем в цилиндрической системе координат $x^1=r$, $x^2 = \theta$, $x^3=z$. Считаем, что $w_1 = w(r)$, $T=T(r)$, $w_2=w_3=0$.

Задача для определения температуры
\[
  \CP{J_1(\e)}t = 0;\quad
  \CP Tt=0;\quad
  \DP qt = \frac{\kappa}\rho\Delta T.
\]

Вот такая получается задача Дирихле
\[
\begin{cases}
  \Delta T = 0,& a<r<b;\\
  T=T_1,& r=a;\\
  T = T_0,& r=b.
\end{cases}
\]
Отсюда находится $T = T(r)$.

Задача для определения $\ve w$, $\e_{ij}$, $P_{ij}$ "--- это уравнения Навье"--~Ламе с учётом температурных напряжений и граничные условия
\[
  (\lambda+\mu)\grad\div\ve w + \mu\Delta\ve w - \alpha(3\lambda+2\mu)\grad T = 0.
\]
В проекции на ось $r$ имеем
\[
  (\lambda+\mu)\CP{ }r\div\ve w + \mu \Delta w - \alpha(3\lambda + 2\mu)\CP Tr = 0.
\]
Остальные две проекции в наших предположениях выполняются тождественно.

Граничные условия задачи: $\ve P_n|_{r=a} = \ve P_n|_{r=b}=0$; перемещений вдоль оси $z$ нет на границе $z\in\{0,l\}$, то есть $w_n|{z\in\{0,l\}} = 0$; $\ve P_{n\tau}|_{z\in\{0,l\}}=0$.

На деформации и напряжения имеем уравнения
\begin{eqnarray*}
  \e_{ij} &=&  \frac12(\nabla_i w_j+ \nabla_j w_i);\\
  P_{ij} &=&  \lambda J_1(\e)g_{ij} + 2\mu \e_{ij} - \alpha(3\lambda + 2\mu)(T-T_0)g_{ij}.
\end{eqnarray*}
Если $w_1 = w(r)$ и $w_2=w_3=0$, то $\e_{11} = \CP wr$, $\e_{22} = wr$, остальные деформации $\e_{ij}=0$.

Как теперь записать это в физических компонентах. Как будто бы мы возвращаемся в декартовы. $\e_{rr} = \e_{11\text{физ}} = \CP wr$. $\e_{\theta\theta} = \frac1{r^2}\e_{22} = \frac wr$. Остальные нули.

Теперь про напряжения
\begin{eqnarray*}
  P_{rr} =  P_{11} &=&  \lambda\left( \CP wr + \frac wr \right) + 2\mu\CP wr - \alpha(3\lambda+2\mu)(T-T_0);\\
  P_{\theta\theta} =  P_{22}\frac1{r^2} &=& \lambda\left( \CP wr + \frac wr \right) + 2\mu\frac wr - \alpha(3\lambda+2\mu)(T-T_0);\\
  P_{zz} &=& \lambda\left( \CP wr + \frac wr \right) - \alpha(3\lambda+2\mu)(T-T_0).
\end{eqnarray*}

Вычислили на предыдущей лекции, что $w = Ar + \frac Br + \frac{f(r)}r$, где $f(r) = \beta\int\limits_a^r r(T-T_0)\,dr$, $\beta = \frac{\alpha(3\lambda+2\mu)}{\lambda+\mu}$. Ещё мы вычичисли
\[
  P_{rr} = \frac{\Omega}{b^2-a^2}\left( 1-\frac{a^2}{r^2} \right) - \frac{2\mu f(r)}{r^2};\quad
  \Omega = 2\mu f(b).
\]

И можно то же проделать на $P_{\theta\theta}$
\[
  P_{\theta\theta} = A_1 + \frac{B_1}{r^2} + \frac{2\mu f}{r^2} - 2\mu\beta(T -T_0).
\]

И последняя компонента
\[
  P_{zz} = \frac{\lambda}{\lambda+\mu} A_1 - 2\mu \beta(T-T_0)
\]

Это ещё не конец, так как надо ещё подставить туда температуру. То есть её надо найти. Кое-что можно увидеть сразу
\begin{itemize}
\item При $r=a$ имеем $T=T_1$, $f(a)=0$.
\[
  P_{rr}=0,\quad
  P_{\theta\theta} = \frac{2\Omega}{b^2-a^2} - 2\mu\beta(T_1-T_0);\quad
  P_{zz} = \frac{\lambda}{\lambda+\mu} - 2\mu\beta(T_1-T_0).
\]
\item При $r=b$ имеем $T=T_0$, $\Omega = 2\mu f(b)$.
\[
  P_{rr}=0;\quad
  P_{\theta\theta} = \frac{2\Omega}{b^2-a^2} >0;\quad
  P_{zz} = \frac{\lambda}{\lambda+\mu}\frac{\Omega}{b^2-a^2}>0.
\]
\end{itemize}
Там где температура не участвует, мы можем что-то сказать хотя бы про знак.

Определим же $T$. Нам нужно вычислить оператор Лапласа
\[
\Delta T = 
g^{ij} \nabla_i \nabla_j T = 
g^{11} \nabla_1 \nabla_1 T + 
g^{22} \nabla_2 \nabla_2 T + 
g^{33} \nabla_3 \nabla_3 T.
\]
Вычислим каждое слагаемое
\[
  \nabla_1 T = \CP Tr;\quad
  \nabla_2 T = \CP T\theta=0;\quad
  \nabla_3 T = 0.
\]
Это мы скаляр дифференцировали, дальше получили вектор.
\begin{eqnarray*}
  \nabla_1\nabla_1 T &=&  \CP{ }r\left( \CP Tr \right) - \underbrace{\nabla_k T\Gamma_{11}^k}_0 = \CP{^2T}{r^2};\\
  \nabla_2\nabla_2 T &=& \CP{ }\theta(\nabla_2 T) - \nabla_k T\Gamma^k_{22} = \CP Tr r;\\
  \nabla_3\nabla_3 T &=& 0.
\end{eqnarray*}

Таким образом, $\Delta T = \CP{^2T}{r^2} + r\CP Tr = \frac1r\CP{ }r\left( r\CP Tr \right)$. Это для скаляра. А для вектора, конечно, выражение было сложнее у оператора Лапласа.

Интегрируем $r\CP Tr = C_1$, $T = C_1\ln r + C_2$. Константы $C_1,C_2$ находятся из граничных условиях $T|_{r=a} = T_1$, $T|_{r=b} = T_0$.

Для $T_1>T_0$ получаем
\[
  T- T_0 = (T_1 - T_0) \frac{\ln\frac rb}{\ln\frac ab}>0.
\]
То есть константы $C_1 = \frac{ T_1 - T_0}{\ln\frac ab}$, $C_2 = T_0 - \frac{T_1-T_0}{\ln\frac ab}\ln b$.

Надо ещё найти $f(r)$.
\[
  f(r) = \beta\int\limits_0^r r(T-T_0)\,dr = 
 \frac{\beta(T_1-T_0)}{\ln\frac ab}\int\limits_a^r r\ln\frac rb\,dr =
  \beta\frac{(T_1-T_0)}{\ln\frac ab}\left[ 
    \frac{r^2}2\ln\frac rb - \frac{a^2}2\ln\frac ab - \frac{r^2-a^2}4
   \right].
\]
Тогда и 
\[
  \Omega = 2\mu f(b) = \frac{\beta(T_1 - T_0)}{\ln\frac ab} \left[ 
    -\frac{a^2}2\ln\frac ab - \frac{b^2-a^2}4
   \right].
\]

И далее подставляем в напряжения.
\begin{eqnarray*}
  P_{rr} &=& \frac{\mu \beta(T_1-T_0)}{\ln\frac ba}\left[ 
    -\ln\frac br - \frac{a^2}{b^2-a^2}\left( 1-\frac{b^2}{r^2} \right)\ln\frac ba
   \right];\\
  P_{\theta\theta} &=& \frac{\mu \beta(T_1-T_0)}{\ln\frac ba}\left[ 
    1 - \ln\frac br - \frac{a^2}{b^2-a^2}\left( 1+\frac{b^2}{r^2} \right)\ln\frac ba
   \right].
\end{eqnarray*}

Чем считать производные и исследовать поведение этих функций, лучше на компьютере построить и посмотреть.
\subsection{Плоские задачи теории упругости}
Плоское деформированное состояние, то есть существует плоскость (плоскость $xy$), что 
\begin{roItems}
\item все перемещения параллельны этой плоскости.
\item все характеристики не зависят от расстояния до этой плоскости, то есть $w_1=w_1(x,y)$, $w_2 = w_2(x,y)$, $w_3=0$.
\end{roItems}

В частности, задача, которая у нас была, предполагала, что перемещение вдоль одной оси отсутствовало.

% рисунок
% наверх ось $z$ скозь следующее
% [===]
%  ||
%  ||
%  ||
% [===]
% Далее другой рисунок этой же делаод, вид со стороны 135-135-90
% какие-то примеры деталей, вытянутых вдоль оси

Деформации в плоском деформированном состоянии $\e_{11},\e_{12},\e_{22}$ зависят от $x,y$; а вот $\e_{13} = \frac12\left( \CP {w_1}{x^3} + \CP {w_3}{x^1} \right)=0$, $\e_{23}=0$, $\e_{33} = 0$.

Будем рассматривать задачи изотермические, то есть $T=T_0$. Попробуем написать закон Гука
\[
  P_{ij} = \lambda J_1(\e)\delta_{ij} + 2\mu\e_{ij}.
\]
Из напряжени будут $P_{11},P_{12},P_{22}$, зависящие от $x,y$. И условия $P_{13} = 2\mu\e_{13} = 0$, $P_{23}=0$ означают, что напряжения на боковой поверхности не должны иметь компоненты вдоль $z$. $P_{33} = \lambda J_1(\e)= P_{33}(x,y)$.

Напишем уравнения равновесия в плоском деформированном состоянии
\begin{eqnarray*}
  \CP{P_{11}}{x} + \CP{P_{12}}y + \rho_0 F_x &=& 0;\\
  \CP{P_{21}}{x} + \CP{P_{22}}y + \rho_0 F_y &=& 0;\\
  \rho_0 F_z &=& 0.
\end{eqnarray*}

То есть для того, чтобы было плоское деформированное состояние, необходимо, чтобы $F_z=0$.
