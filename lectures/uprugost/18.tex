\section{Лекция 18}
Я пропустила один пункт в прошлой лекции.
\begin{enumerate}
\item Система уравнений для адиабатических процесов в изотропной линейно термоупругой среде. В частности, адиабатическая модель упругости. 
\item Задача о температурных напряжениях в стенках трубы.
\end{enumerate}
Ну и хватит.

\subsection{Адиабатические процессы}
Если процессы изотермические, то температурны напряжения и деформации отсутствуют. Там закон Гука, а температура должна быть задана.

Существует (и много таких) процессы, когда температура разная. В частности, например, если мы рассматриваем звук, распространяющийся в упругой среде, как распространяется звук в стенке. Или как возникает землетрясение. В этих случая процессы происходят очень быстро и тепло не успевает переходить. Это адиабатические процессы. Можно в этом случае написать замкнутую систему уравнений.

Что такое адиабатический процесс. Это процесс, в котором $dq=0$ в каждой частице. В упругой среде выполняется соотношение, которое можно назвать вторым законом термодинамики (мы знаем, что в нашей ситуации оно же может играть роль уравнения притока тепла), $T\,ds = dq=0$, если процесс адиабатический. Таким образом в адиабатическом процессе постоянна в каждой частице. Мы можем написать для этого соотношение
\[
   s = \frac{\alpha(3\lambda + 2\mu)}{\rho_0} J_1(\e) + c_\e\frac{T-T_0}{T_0} + s_0.
\]
Если $s=s_0$, то 
\[
  T-T_0 = -\frac{\alpha(3\lambda + 2\mu)}{c_\e}\frac{T_0}{\rho_0} J_1(\e).
\]
Это и есть аналог адиабаты Пуассона в газе. Здесь $J_1(\e) = g^{ij}\e_{ij}$ "--- относительное изменение объёма.

Теперь можем подставить эту связь в закон Гука с учётом температурных напряжений.
\[
  P^{ij}  = \lambda J_1(\e) g^{ij} + 2\mu\e^{ij} - \alpha(3\lambda + 2\mu)(T-T_0) g^{ij}.
\]
А $T-T_0$ есть какой-то коэффициент, умноженный на $J_1(\e)$. Поэтому первый и последний члены объединяются. Получаем
\[
  P^{ij} = \left( \lambda + \frac{\alpha^2(3\lambda + 2\mu)^2}{c_\e}\frac{T_0}{\rho_0} \right)J_1(\e) g^{ij} + 2\mu\e^{ij}.
\]
Введём обозначение $\lambda_{\text{ад}} = \lambda + \frac{\alpha^2(3\lambda + 2\mu)^2}{c_\e}\frac{T_0}{\rho_0}$. Тогда
\[
  P^{ij} = \lambda_{\text{ад}} J_1(\e)g^{ij} + 2\mu \e^{ij}.
\]
Коэффициенты $\lambda_{\text{ад}}$ "--- адиабатические модули упругости. По форме закон совпадает с законом Гука при $T=T_0$.

Оценим величину этих коэффиентов. В таблицах вы обычно не найдёте $\lambda$ и $\mu$. Вместо них приводятся модуль Юнга и коэффициент Пуассона. Их можно выразить. Обозначим $\lambda_{\text{ад}} = \lambda(1+e)$. Тогда
\[
  e = \frac{\alpha^2(3\lambda + 2\mu)^2}{\lambda c_\e}\frac{T_0}{\rho_0};\quad
  \lambda = \frac{E\sigma}{(1+\sigma)(1-2\sigma)};\quad
  3\lambda + 2\mu = 3K = \frac{E}{1-2\sigma}.
\]
Отсюда
\[
  e = \frac{E(1+\sigma)}{\sigma(1-2\sigma)}\frac{\alpha^2}{c_\e}\frac{T_0}{\rho_0}.
\]
Для стали $E = 2\cdot 10^{11}\frac{\text{кг}}{\text{м}\,\text{с}^2}$,
$\sigma = 0{,}25$,
$\alpha = 1{,}25\cdot 10^{-5} \frac1{\text{К}}$,
$T_0 = 300\text{К}$,
$c_\e = 460\frac{\text{м}^2}{\text{с}^2\text{К}}$,
$\rho_0 = 7{,}8\cdot 10^3\frac{\text{кг}}{\text{м}^3}$.
Получается
\[
  e = 0{,}026 = 2{,}6\%.
\]

Какие же уравнения получаются для адиабатичкских процессов в изотропной линейной термоупругой среде.
\begin{eqnarray*}
\rho_0\CP{^2w^i}{t^2} &=& \rho_0 F^i + \nabla_j P^{ij};\\
P^{ij} &=&  \lambda_{\text{ад}}J_1(\e)g^{ij} + 2\mu \e^{ij};\\
\e_{ij} &=&  \frac12(\nabla_iw_j + \nabla_jw_i).
\end{eqnarray*}
Это замкнутая система переменных $w_i,\e_{ij},P^{ij}$. Но если вам нужна ещё температура, то добавляется
\[
  T = T_0 - \frac{\alpha(3\lambda+2\mu)}{c_\e}\frac{T_0}{\rho_0}J_1(\e).
\]

\subsection{Задача о температурных напряжениях в стенках трубы}
Внутри трубы температура одна, вне "--- другая. Только за счёт этого будут разные деформации в разных местах этой трубы. Мы можем сразу сказать, что будет, если в трубе очень горячий газ, а снаружи очень холодный. Внутри частицы хотят расшириться. А внешняя часть не хочет расширяться. Но внутренняя растягивает внешнюю, а внешняя стремится внутреннюю вернуть в исходное положение. Сделаем для этого точную постановку задачи.

\subsubsection{Физическая постановка задачи}
\begin{enumerate}
\item Имеется труба длины $l$, радиусы сечения $a$ и $b$.
% рисунок:
% горизонтальные концентраческие цилиндры:2 справа видно сечение, слева нет
% сечение труды: две концентрические окружности, подписи радиусов a,b
% ограничивающие плоскости с торцов (вертикальные)
\item Материал стенок: изотропный линейный термоупругий теплопроводный, модули упругости константы (это некое упрощение, они зависят от температуры на самом деле).
\item Давление внутри $P_a=0$, вне $P_b=0$, массовые силы не учитываем. В силу линейности задач, решения можно будет сложить, если понадобится\footnote{Уже можно почуствовать, что при нагреве трубы изнутри разрывающие напряжения на внутренней стенке станут меньше.}.
\item Перемещения вдоль оси нет, то есть труба с торцов ограничена двумя гладкими жёсткими плоскостями.
\item Температура вне трубы $T_0$, а внутри $T_1>T_0$. Причём $T_0$ "--- та температура, при которой при отсутствии сил деформации равны нулю. $T_0=\const$, $T_1=\const$.
\end{enumerate}

Можно задачу ставить и наоборот: внутри температура чтобы меньше была, чем снаружи.

\subsubsection{Математическая постановка задачи}
Равновесие и отсутствие массовых сил сводятся к уравнению $\nabla_j P^{ij} = 0$.
Закон Гука принимает вид
\[
  P^{ij} = \lambda J_1(\e) g^{ij} + 2\mu \e^{ij} - \alpha(3\lambda+2\mu)(T-T_0)g^{ij}.
\]
И связь деформаций с вектором напряжений
\[
  \e_{ij} = \frac12(\nabla_i w_j+\nabla_j w_i).
\]
Уравнение притока тепла $\Delta T=0$ (считаем, что распределение температуры стационарное).

Теперь к этой системе уравнений напишем граничные условия.
\begin{itemize}
\item При $r=a$ выполнено $\ve P_n=0$ (так как $p_a=0$) и $T=T_1$;
\item При $r=b$ выполнено $\ve P_n=0$ (так как $p_a=0$) и $T=T_0$;
\item При $z\in\{0,l\}$ выполнено $w_{n} = 0$, $\ve P_{n\tau} = 0$.
\end{itemize}

Общий вывод из этой постановки задачи. Уравнение и граничные условия на температуру отделяются, не зависят от механики, можно найти заранее температуру $T$ внутри стенок независимо от распределения деформаций и напряжений. Потом можно подставить в закон Гука и решать механическую задачу.

\subsubsection{Решение задачи}
Итак, $T$ находится из задачи $\Delta T=0$ при $a<r<b$, $T|_{r=a}=T_1$, $T|_{r=b}=T_0$. Задача Дирихле. Будем считать, что $T(r)$ уже найдёна и рассмотрим задачу механики, найдём $P^{ij}$. 

Напишем уравнения Навьё"--~Ламе с учётом температурных напряжений. Разобём напряжения на те, что порождены перемещениями, и температурные.
\[
  \nabla_j P^{ij}_w + P^{ij}_T=0.
\]
Тогда для первого слагаемого мы выражение уже знаем. 
Таким образом,
\[
  (\lambda+\mu)\grad\div\ve w + \mu \Delta\w - \alpha(3\lambda + 2\mu)\grad(T-T_0)=0.
\]
Считаем, что $w_1 = w(r)$, $w_2=w_3=0$ в цилинтрических координатах $x^1=r$, $x^2=\theta$, $x^3 = z$. Что мы про цилиндрические координаты знаем: $g_{11} = g_{33} = g_{22}=r^2$.
\[
  \Gamma_{22}^1 = -r;\quad
  \Gamma_{12}^2 = \Gamma_{21}^2 = \frac1r.
\]

Используем результаты лекций 14--15 про задачу Ламе о напряжениях под действием давления $p_a$ и $p_b$, потому что очень всё похоже здесь. Там было такое, что
\[
  \div\ve w = \CP wr + \frac wr.
\]
Тогда уравнения Навье"--~Ламе в компонентах выглядит следующим образом. В проекции на ось $r$ имеем
\[
  (\lambda+2\mu)\CP{ }r\left( \CP wr + \frac wr \right) = \alpha(3\lambda+2\mu) = CP{(T-T_0)}r.
\]
Это уравнение сразу можно проинтегрировать
\[
  \frac1r\CP{rw}{r} = \CP wr + \frac wr = \frac{\alpha(3\lambda+2\mu)}{\lambda+2\mu} (T-T_0) + 2A.
\]
Здесь $2A$ "--- удобное обозначение константы интегрирования. Интегрируем ещё раз
\[
  wr = \beta\int\limits_a^r \xi(T-T_0)\,d\xi = Ar^2 + B.
\]
Обозначим интеграл вместе с $\beta$ известной функцией $f(r) = \beta\int\limits_a^r \xi(T-T_0)\,d\xi$.

Ответ для перемещений получился такой
\[
  w = A r + \frac{B}r + \frac{f(r)}r.
\]

Теперь по этому ответу нужно найти деформации и напряжения. Заметим, что $f(a)=0$ и что $\CP fr = \beta r (T-T_0)$.
\[
  \e_{11} = \CP wr = A - \frac B{r^2} + \beta(T-T_0) - \frac f{r^2}.
\]
Это деформации, а напряжение
\[
  P_{11} = \lambda \div \ve w + 2\mu f - \alpha(3\lambda+2\mu) (T-T_0);\quad
  \div w = \CP wr + \frac wr = A - \frac B{r^2} + \beta(T-T_0) - \frac{f}{r^2} + A + \frac{B}{r^2} + \frac{f}{r^2} = 2A + \beta(T-T_0).
\]
И это понятно,
\[
  P^{11} = \lambda \big(2A + \beta(T-T_0)\big) + 2\mu\left( A - \frac B{r^2} +\beta(T-T_0) - \frac f{r^2} \right) - \alpha(3\lambda + 2\mu)(T-T_0) = 
  2A(\lambda+2\mu) - \frac{2\mu B}{r^2} + \beta(T-T_0)(\lambda+2\mu) - \alpha(2\lambda+2\mu)(T-T_0) - \frac{2\mu f}{r^2}.
\]
Раньше мы обозначали $P_{11} = A_1 -\frac{B_1}{r^2} - \frac {2\mu f}{r^2}$. И сейчас чему равняется $\beta$?
\[
  \beta = \frac{\alpha(3\lambda+2\mu)}{\lambda+2\mu};\quad
  (\lambda+2\mu)\beta= \alpha(3\lambda+2\mu).
\]
И два слагаемых сокращаются. Итак
\[
  P_{11} = A_1 - \frac{B_1}{r^2} - \frac{2\mu f}{r^2}.
\]

Найдём $A_1$ и $B_1$ из граничных условий.
\begin{itemize}
\item На внутренней стенке $P_{11}|_{r=a} = -p_a=0$, кроме того $f(a)=0$. Таким образом
\[ 
  A_1 - \frac{B_1}{a^2} = 0.
\]
(В задаче Ламе было справа $-P_a$.)
\item На внешней стенке $P_{11}|_{r=b} = -p_b=0$. Таким образом
\[
  A_1 - \frac{B_1}{b^2} = \frac{2\mu f(b)}{b^2}.
\]
(В задаче Ламе было справа $-p_b$.)
\end{itemize}
Выражения для $A_1$ и $B_1$ можно прямо взять из решения задачи Ламе с подстановкой других $p_a=0$ и $p_b = -\frac{2\mu f(b)}{b^2}$.
\[
  A_1 = \frac{\Omega}{b^2-a^2},\quad
  B_1 = \frac{a^2\Omega}{b^2-a^2};\quad
  \Omega = 2\mu f(b).
\]
