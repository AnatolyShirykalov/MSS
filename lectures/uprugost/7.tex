\section{Лекция 7}
До сих пор у вас были точные решения.

Мы решаем уравнения Навье"--~Стокса, дополненное уравнением неразрывности.
\begin{eqnarray*}
  \rho\DP{\ve v}{t} &=& \rho\,\ve F - \grad p + \mu\Delta\ve v;\\
   \div\ve v &=&0.
\end{eqnarray*}
Для решения задач добавляют граничные условия.

Оказывается, уравнения очень сложные. Полная производная содержит нелинейную часть уравнения
\[
  \DP{\ve v}t = \matder {\ve v}.
\]
Этот кусочек и мешает нам получать точные решения. Поэтому необходимо использовать некоторые приближения.

Запишем уравнение Навье"--~Стокса в проекции на ось $x$.
\[
  \rho\left(\matder{v_x}\right) = -\CP px + F_x + \mu\left(\Lap{v_x}\right).
\]
Это уравнение есть уравнение количества движения, представляет баланс сил.

Введём такую терминологию. Будем считать, что есть характерные $V$, $L$, то есть $v_x\sim V$, $\CP{v_x}x\sim\frac{\Delta v_x}{\Delta x}\sim \frac VL$. Тогда $v_x\CP{v_x}x\sim V\cdot\frac VL = \frac{V^2}L$. А вторые производные $\CP{v_x}{x^2} = \frac V{L^2}$. Таким образом, отношение инерционных членов к вязким имеет порядок
\[
  \frac{\rho\frac{V^2}{L}}{\mu\frac V{L^2}} = \frac{\rho\,V\,L}\mu = \RE.
\]

\subsection{Приближение Стокса}
Пусть $\RE\ll1$. Тогда уравненями Стокса называется
\begin{eqnarray*}
\rho\DP{\ve v}t &=& \rho\ve F - \grad p + \mu\Delta\ve v;\\
\div\ve v&=&0.
\end{eqnarray*}
Когда это удобно? Обтекание сферы стационарным течением при отсутствии массовых сил. Тогда силы давления компенсируются силами вязкости 
\begin{eqnarray*}
\grad p &=&\mu\Delta v;\\
\div\ve v&=&0.
\end{eqnarray*}
В такой постановке уже можно решать.

Есть существенное ограничение у такой постановки. Если $\nu = \frac\nu\rho = 0{,}01\frac{\text{см}^2}{\text{с}}$ и $\RE<1$, то $VL<0{,}01\frac{\text{см}^2}{\text{с}}$. Если $V<1\frac{\text{м}}{\text{с}}$, то размер сферы должен быть очень маленький $L<0{,}01\text{см}$.

Имеет место формула Стокса для силы сопротивления
\[
  W = 6\,\pi\,\mu\,a\,v_\infty.
\]
Давайте эту формулу получим из теории размерности.

Обращаю ваше внимание: у нас малые числа $RE = \frac{\rho\, v_\infty\,a}\mu$. Когда говорят про Стоксовое приближение, говорят о медленном движение в очень вязкой жидкости.
\[
  W = f(a,v_\infty,\mu).
\]
Введём $\{L,M,T\}$. Запишем размерности $[a] = L$, $[v_\infty] = \frac LT$, $[\mu] = \frac M{L\,T}$. Давайте попробуем найти такую комбинацию величин, чтобы
\[
  \left[\frac W{a^\a\,v_\infty^\b\,\mu^\g}\right] = M^0\,L^0\,T^0.
\]
Имеем $M^1\, L^1\, T^{-2} = L^\a\,L^\b\, T^{-\b} M^\g\,L^{-\g}T^{-\g}$. Отсюда
\[
\begin{cases}
  \a +\b-\g = 1;\\
  \g=1;\\
  -2 = -\b-\g.
\end{cases}
\]
Отсюда видно, что $\a=\b=\g$. По $\pi$-теореме $\exists C=\const\colon W = C\,a\,v_\infty\,\mu$.

Эта задача о медленном движении маленьких сфер (эксперимент показывать похожие результаты при $\RE<\frac12$). Капелька тумана в воздухе.

\subsection{Течение Хилл"--~Шоу}
Между вертикальными стенками. Число рейнольдса между такими стенками оказывается малым, если течение медленное, а стенки расположено близко. Визуализация обтекания цилиндра идеальной жидкостью сделана именно в лотке Хилл"--~Шоу.

Есть такое техническое устройство, как подпишник. Внешняя часть называется цапфой. Два цилиндра, между ними жидкость. По сравнению с радиусом подшипника толщина слоя жидкости маленький. Возникают огромные вязкие силы. Из уравнения Стокса здесь получаются близки к действительности движения.

Ещё применение годится в задачах фильтрации.

Итак, для постановки стокса принебрегается нелинейными членами, связанными с ускорением.
\subsection{Теория пограничного слоя}
%рисун
Если у нас есть обтекаемое тело. Оно стоит. На поверхности тела скорости ноль. Если отойти подальше, идёт поток. Если посмотреть на профиль скорости, то на уже небольшом расстоянии от тела скорость выйдет на стабильный уровень. Пусть $\ve = (u,v)$. Тогда члены $\tau_{xy} = \mu\frac uy$ близко к телу сущесвенны, а далеко нет больших перепадов. Это соображения наводят на мысль о том, что при больших числах рейнольдса можно влияние вязкости учитывать только в тонком слое, называемом пограничным, а в пограничным не учитывать. Дальше будем склеивать решения по непрерывности.
%
Напишем 
\begin{eqnarray*}
  \CP ut + u\CP ux + v\CP uy &=&-\frac1\rho\CP px + \mu\left(\CP{^2u}{x^2} + \CP{^2u}{y^2}\right);\\
  \CP vt + u\CP vx + v\CP vy &=&-\frac1\rho\CP py + \mu\left(\CP{^2v}{x^2} + \CP{^2v}{y^2}\right);\\
  \CP ux + \CP vx &=& 0.
\end{eqnarray*}
Давайте обезразмерим.  $u_1 = u/V$, $v_1 = v/V$, $x_1 = x/L$, $y_1 = y/L$, $p_1 = \frac p{\rho V^2}$, $t_1=t\frac VL$.
\[
  \CP ut = \frac{v^2}L\CP{u_1}{t_1};\quad
  u\CP ux = \frac{V^2}L u_1\CP{u_1}{x_1};\quad
  \CP{^2u}{x^2} = \frac V{L^2}\CP{^2u_1}{x^2}.
\]
Что же у меня получилось
\[
  \frac{V^2}L\left(\CP{u_1}t + u_1\CP {u_1}{x_1}+v_1\CP{u_1}{y_1}\right) = 
  - \frac{V^2}L\CP{p_1}{x_1} + \frac{\nu V}{L^2}\left(\CP{^2u_1}{x_1^2} + \CP{^2u_1}{y_1^2}\right).
\]
Если сократим, вот что получится.
\[
  \CP{u_1}t + u_1\CP {u_1}{x_1}+v_1\CP{u_1}{y_1} = 
  - \CP{p_1}{x_1} + \frac{\nu}{L\, V}\left(\CP{^2u_1}{x_1^2} + \CP{^2u_1}{y_1^2}\right).
\]
Итого, в безразмерном виде уравнение имеет вид (я не сказал, но вы поняли, что массовые силы я не пишу)
\begin{eqnarray*}
  \CP ux + u\CP ux + v\CP uy &=& -\CP px + \frac1{\RE}\left(\CP{^2}{x^2} + \CP{^2u}{y^2}\right);\\
  \CP ux + u\CP ux + v\CP uy &=& -\CP px + \frac1{\RE}\left(\CP{^2}{x^2} + \CP{^2u}{y^2}\right);\\
  \CP ux + \CP vy &=& 0.
\end{eqnarray*}
Что будет при больших числах $\RE$? Казалось бы, можно считать жидкость идеальной. Но постановки задач-то вязкой и идеальной жидкости разные. Нам надо добиться такого решения, чтобы было выполнено условие прилипания (или условие на свободной поверхности). В общем решение для идеальной жидкости краевым условиям удовлетворять не будет.

А что будет если положить $\frac1{\RE}=0$? Тогда изменится порядок уравнения, станет меньше констант, меньшему количеству условий решение сможет удовлетворить. Рассмотрим обыкновенно дифференциальное уравнение
\[
  m \ddot x + k \dot x + cx=0.
\]
Рассмотрим следующие случаи малой массы.
\begin{azItems}
\item $m=0$. Тогда $x = A e^{-\frac{ct}k}$.
\item При $m\ne0$, но $m\ll\frac{k^2}{4c}$, то $x = A_1 e^{-\frac{ct}k} + A_2 e^{-\frac{kt}m}$. Потребуем, чтобы $A_1=A$, то есть с какого-то $t$ это решение стало близко к предыдущему пункту, а вторая константа из условия $x(0)=0$. Имеем $A_2 = -A_1 = -A$. Тогда $x = A(e^{-\frac{ct}k} - e^{-\frac{kt}m})$. Быстрое решение быстро выходит на медленное.
% рисунок содержательный
\end{azItems}
Это пример Прандтля.

Пусть поверхность тела прямолинейная и есть какой-то пограничный слой. Обобщначим толщину пограничного слоя через $\d_p$
% рисунок
Имеем характерные величины $\rho, V, L$. Тогда $\CP ux \sim 1$, $\d=\frac{\d_\text{п}}L\ll1$.
Из уравнения неразрывности $\CP vy\sim 1$, $\frac{\Delta v}{\Delta y}\sim 1$, $\Delta y\sim\d$, значит, $\Delta v \sim\d$, $v\sim \d$.

Мы предполагаем, что силы инерции, силы вязкости и силы давления должны быть одного порядка. Насильственное предположение: $\CP ut\sim 1$, $\frac px\sim 1$ и $\frac1{\RE}\sim \d$. Остальные слагаемые в первом уравнении оцениваются автоматически. И после предположений можно оценить все из второго. В итоге получаем $\CP py\sim \d$, а это изменение давление поперёк пограничного слоя. То есть $\Delta p \sim \d$. Отсюда можем сделать вывод, что давление не зависит от $y$, а зависит от $(x,t)$.

И последнее $\CP{^2u}{y^2}\gg\CP{^u}{x^2}$.

Вернёмся к размерным переменным и ещё поговорим. Следующие уравнения называются уравнением Прандтля.
\begin{eqnarray*}
  \CP ut + u\CP ux+ v\CP uy&=&-\frac1\rho + \nu\CP{^2u}{y^2};\\
  \CP ux + \CP vy&=&.
\end{eqnarray*}
  Нет $\CP{^2u}{x^2}$, это можно было выкинуть и по интуиции. И второго уравнения нет. Но из того, что $p(x,y)$, мы эти уравнения решить не можем. Но мы будем считать давление не искомой, а известной величиной. Мы его получим при склейке с решением задачи о движении идеальной жидкости. Итак в уравнении пограничного слоя неизвестные есть $u,v$.

Внешний поток обтекает не само тело, а некоторое тело, включающее пограничный слой. Но пограничный слой достаточно тонкий.

У нас получилось, что $\frac1{\RE}\sim \delta$. А что мы этими буквами обозначили $\RE = \frac{VL}\nu$, $\delta = \frac{\d\text{п}}L$. Тогда $\delta\sim \sqrt{\frac\nu{VL}}$, а $\d_\text{п} \sim \sqrt{\frac{\nu\,L}v}$.

Есть некоторые проблемы использования этой теории. Одна из проблем связана с отрывом пограничного слоя. Вы изучали обтекание сферы идеальной жидкостью.
% рисунок
Поток семитричный, есть парадокс Даламбера. Когда частицы верхнюю точку пролетает, она тормозится, а давление увеличивается. Но если жидкость вязкая, то вблизи сферы происходит подтормаживание. Не хватает инерции, чтобы двигаться против давления. Частицы останавливаются, перепад давления оказывается в другую сторону и частицы отрываются.
% рисунок отрыв
Отрыв приводит к большой силе сопротивления.

Уравнения погран слоя можно интегрировать только до точки отрыва.
% рисунок обычный профиль скорости и профиль с отрывом чуть ли не скрипичный ключ
Предельный случай, когда профиль скорости имеет вертикальную касательную (в $y(x)$. Ясно, что это и будет точкой отрыва, давайте её найдём из уравнения Прандтля.
\[
  u\CP ux + v\CP uy = -\frac1\rho\CP px + \nu\CP{^2u}{y^2}\bigg|_{y=0\imp u=v=0}.
\]
и $\frac1\rho\CP px\Big|_{y=0} = \nu\CP{^2u}{y^2}\Big|_{y=0}$. Рассмотрим случаи
\begin{roItems}
\item  $\CP{^2u}{y^2}\Big|_{y=0}<0\iff \CP px\Big|_{y=1}<0$;
\item  $\CP{^2u}{y^2}\Big|_{y=0}>0\iff \CP px\Big|_{y=1}>0$;
\item  $\CP{^2u}{y^2}\Big|_{y=0}=0\iff \CP px\Big|_{y=1}=0$;
\end{roItems}
Этот перепад давления и подгоняет частицы оторваться. Происходит обратное подтекание.

Для пространственного случая эти же идеи применимы.

Маргирита Эрнестовна рассказывала плоское течение Пуазейля. Течение между двумя плоскими пластинами. Расстояние между пластинами $h$, расстояние, которое рассматриваем $l$. $v_{\text{ср}} = \frac{Q}{h} = \frac{\Delta p}{12\mu l}h^2$.
% рисунок 
Если у нас садовый участок. Высота столба воды $10$ м. Длина горизонтальной трубы $l=100$ м, $h = 0{,}5$ м. Бак создаёт одну атмосферу и ещё атмосфера сама давит. А на выходе только одна атмосфера. Тогда $v_{\text{ср}} = \frac{10^5\,0{,}5^2}{12\cdot 10^{-3}\cdot 100}\frac{\text{м}}{\text{с}}\approx 20\frac{\text{км}}{\text{с}}$.
%рисунок садовый участок

Вообще когда попытаемся эту теорию применить для расчёта течения Волги, получим $v = 12{,}5\text{км}/\text{с}$. Корабли бы вылетали с речки со второй космической скоростю. Почему такой бред? Подкрасим жидкость. Потекут полосочки, но при увеличении скорости полосочки пропадут. Начиная с какой-то скорости появляются интенсивные поперечные составляющие. 

Предположение о ламинарности очень сильное. Всё определяется числом $\RE = \frac{\rho\,v_{\text{ср}}\,d}{\mu}$. Ламинарность при маленьких. Переход при $\RE_{\text{кр}}\approx 2\,400$ в круглой трубе.
