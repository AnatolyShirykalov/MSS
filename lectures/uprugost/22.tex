\section{Лекция 23}
\begin{enumerate}
\item Теорема единственности решения задач о равновесиит линейно"--~упругого тела.
\item Волны в упругих средах. Плоские волны в неограниченной упругой среде. Продольные и поперечные волны.
\item Некоторые упругие эффекты при деформировании твёрдых тел.
\end{enumerate}<++>

Математическая формулировка задачи. Уравнение равновесия
\[
 \nabla_j P^{ij} + \rho F^i = 0;
\]
Закон Гука с учётом температурных напряжений
\[
  P^{ij} = \lambda J_1(\e)g^{ij} + 2\mu \e^{ij} - \alpha(3\lambda + 2\mu)(T-T_0) g^{ij}.
\]
Или, для анизотропной среды, $P^{ij} = A^{ijkl}\e_{kl}+ B^{ij}(T-T_0)$.
Кроме того, выражение компонент тензора напряжений через вектор перемещения
\[
  \e_{ij}=\frac12(\nabla_i w_j+\nabla_j w_i).
\]
Или уравнений совпестности.
