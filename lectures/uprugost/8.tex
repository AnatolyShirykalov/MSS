\section{Что-то новое}
\begin{enumerate}
\item Турбулентность. Осреднение параметров турбулентности движения. Свойства операции усреднения.
\item Осреднение уравнения движения. Уравнения Рейнольдса.
\item Тензор турбулентных напряжений. Физический смысл.
\item Проблема замыкания уравнений Рейнольдса. О полуэмпирических теориях турбулентности.
\end{enumerate}
\subsection{Понятия турбулентности и осреднения}
Движение называется турбуленным, если характеристики движения испытывают нерегулярные пульсации в пространстве и во времени.
%картинка 1

В одной точке и в соседней могут скорости меняться в разы. Когда горная река течёт это даже глазом видно. В реке Волге глазом не видно, но пульсации там есть и они существенные.

Турбулентные движения возникают не обязательно при больших числах Рейнольдса. $\RE = \frac{v\,L}{\nu}$. Вот если $\RE<\RE_{\text{крит}}$ (для каждого течения это критическое значение своё), то течение получается, как бы его ни возмущали, ламинарным.

Как ввести осреднение и осреднить все наши уравнения. Можно усреднять по объёму, можно осреднять по времени.
\subsubsection{Осреднение по времени}
В каждый момент времени $t$ выбирается интервал величины $T$, в который попадает достаточно много пульсаций, но «осреднённое» движение меняется мало.
Среднее значение велицины $\phi(x^i,t)$ обозначается
\[
\la \phi\ra = \frac1T\int\limits_{t-\frac T2}^{t+\frac T2}\phi(x^i,\tau)\,d\tau.
\]
У операции осреднения есть шесть свойств.
\begin{Ut}
  Если $C$ "--- константа, то $\la C\ra = C$.
\end{Ut}
\begin{Ut}
 Если осредняем ещё раз, $\big\la\la\phi\ra\big\ra = \la \phi\ra$.
\end{Ut}
\begin{Ut}
  Из линейности интеграла $\la \phi+f\ra = \la \phi\ra + \la f\ra$.
\end{Ut}
Осреднить уравнение значит осреднить каждый его член. Это пока простые свойства. 
\begin{Ut}
  Как найти среднее значение произведения ($\la f\phi\ra\ne\la f\ra \la\phi\ra$).
\[
  \MID fg = \MID f\MID \phi + \MID{f'\phi'}.
\]
\end{Ut}
\begin{Proof}
  Введём, так называемую, пульсационную добавку $\phi' = \phi-\la\phi\ra$. То же самое для любой величины, например, $f = \la f\ra +f'$. Среднее значение от пусльсационной добавки равно нулю.
\[
  \la \phi'\ra = \big\la \phi - \la \phi\ra\big\ra.
\]
Теперь считаем среднее от произведения.
\[
  \la fg\ra = \Big\la \big(\la f\ra + f'\big)\big(\la \phi\ra + \phi'\big)\Big\ra = 
 \big\la \MID f \MID g + \MID f \phi' + f'\MID \phi + f'\phi'\big\ra =
  \MID f\MID g + \MID{f'\phi'}.
\]
Если $f'$ и $\phi'$ как-то согласованы по знаку, то среднее значение произведения может быть не ноль.
\end{Proof}
\begin{Ut}
  Прямо по определению осреднения $\MID[\Big]{\CP\phi{x^i}} = \CP{\MID\phi}{x^i}$.
\end{Ut}
\begin{Ut}
  $\MID[\Big]{\CP\phi t} = \CP{\MID\phi}t$.
\end{Ut}
\begin{Proof}
  Аккуранто вычисляем левую часть равенства.
\[
  \MID[\bigg]{\CP\phi t} = \iMID T\CP{\phi(x^i,\tau)}\tau\,d\tau = 
  \frac 1T\left[\phi\left(x^i,t+\frac T2\right) - \phi\left(x^i,t-\frac T2\right)\right]
\]
Теперь правую по теореме о дифференцировании интеграла по параметру.
\[
  \CP{ }t\iMID T\phi(x^t,\tau)\,d\tau =
  \frac 1T\left[\phi\left(x^i,t+\frac T2\right) - \phi\left(x^i,t-\frac T2\right)\right]
\]
\end{Proof}
\subsection{Осреднение уравнений движения. Уравнения Рейнольдса}
Посмотрим на сами уравнения
\[
  \rho\DP{\ve v}t = \rho\ve F + \nabla_i\ve P^i.
\]
В декартовой системе координат
\[
  \rho\DP{v_x}t = \rho F_x + \CP{P_{xx}}x + \CP{P_{xy}}y + \CP{P_{xz}}z.
\]
Здесь $P_{ij}$ "--- компоненты тензора напряжений. Полученное осреднение будет верно для любых сред, не обязательно линейно вязких.

Преобразуем величину
\begin{multline*}
  \rho\DP{v_x}t = \rho\left(\matder {v_x}\right) + v_x\underbrace{\left(\CP\rho x + \CP{\rho v_x}x + \CP{\rho v_y}y + \CP{\rho v_z}z\right)}_{=0\text{ ур. нер-ти}} = \\=
  \CP{\rho v_x}t + \CP{\rho v_x v_x}x + \CP {\rho v_x v_y}y + \CP{\rho v_y v_z}z.
\end{multline*}
Теперь только производные входят в сумму. Я уже пропагандировала эту формулу. Она ведь удобна и для численных методов, где производные заменяются на разности. А если разность нужно умножить на значение, то значение в какой точке брать: начальной или конечно? Это повлияет на решение.

\subsubsection{Вывод уравнения Рейнольда при $\rho=\const$}
Это осреднение уравнений движения. Возьмём уранение в проекции на ось $x$
\[
  \CP{\rho v_x}t + \CP{\rho v_x v_x}x + \CP {\rho v_x v_y}y + \CP{\rho v_y v_z}z = \rho F_x + \CP {P_{xx}}{x} + \CP{P_{xy}}y + \CP{P_{xz}}z.
\]
Имеем $v_i = \MID{v_i} + v_i'$, силы могут пульсировать $F_x = \MID{F_x}+ F'_x$, $P_{xx} = \MID{P_{xx}} + P'_{xx}$.
И самое интересное следующее, считаем при $\rho=\const$, то есть среда только однородная и несжимаемая среда.
\[
  \MID{\rho v_x v_x} = \rho\MID{v_x}\MID{v_x} + \MID{\rho v_x'v_x'}.
\]

Применяем осреднение ко всему тому уравнению. Все пульсационные добавки переносим в правую часть.
\[
  \CP{\rho\MID{v_x}}t + \CP{\rho\MID{v_x}\MID{v_x}}x + 
  \CP{\rho \MID{v_x}\MID{v_y}}y + \CP{\rho\MID{v_x}\MID{v_z}}z = 
  \rho\MID{F_x} + 
  \CP{\big(\MID{P_{xx}}+T_{xx}\big)}x + 
  \CP{\big(\MID{P_{xy}}+T_{xy}\big)}y + 
  \CP{\big(\MID{P_{xz}}+T_{xz}\big)}z,
\]
где $T_{ij} = -\MID{\rho v'_iv'_j}$.

Получили уравнение на среднее значение скорости. 

Получим ещё осреднение уравнения неразрывности для $\rho=\const$
\[
  \CP{ v_x}x + \CP{v_y}y + \CP{ v_z}z=0. 
\]
Осредняем
\[
  \CP{ \MID{v_x}}x + \CP{\MID{v_y}}y + \CP{ \MID{v_z}}z=0. 
\]
Поэтому $
  \CP{\rho\MID{v_x}}t + \CP{\rho\MID{v_x}\MID{v_x}}x + 
  \CP{\rho \MID{v_x}\MID{v_y}}y + \CP{\rho\MID{v_x}\MID{v_z}}z =\rho\DP{\MID{v_x}}t $.

Таким образом, результат усреднения имеет вид
\begin{eqnarray*}
  \rho\DP{\MID{v_x}}t &=& \rho\MID{F_x} +   \CP{\big(\MID{P_{xx}}+T_{xx}\big)}x + \CP{\big(\MID{P_{xy}}+T_{xy}\big)}y + \CP{\big(\MID{P_{xz}}+T_{xz}\big)}z,\\
  \rho\DP{\MID{v_y}}t &=& \rho\MID{F_y} +   \CP{\big(\MID{P_{yx}}+T_{yx}\big)}x + \CP{\big(\MID{P_{yy}}+T_{yy}\big)}y + \CP{\big(\MID{P_{yz}}+T_{yz}\big)}z,\\
  \rho\DP{\MID{v_z}}t &=& \rho\MID{F_z} +   \CP{\big(\MID{P_{zx}}+T_{zx}\big)}x + \CP{\big(\MID{P_{zy}}+T_{zy}\big)}y + \CP{\big(\MID{P_{zz}}+T_{zz}\big)}z,\\
\end{eqnarray*}
Здесь $T_{ij} = -\MID{\rho v_i'v_j'}$ "--- компоненты тензора турбулентных напряжений. Вообще в индексных обозначениях уравнение Рейнольдса имеет вид
\[
  \rho\DP{\MID{v^i}}t = \rho\MID{F^i} + \CP{\big(\MID{P^{ij}} + T^{ij}\big)}{x^j}.
\]
А тензор турбулентных напряжений $T^{ij} = -\MID{\rho v'^iv'^j}$.

Турбулентные напряжения похожи на молекулярные. Но гораздо более существенные. Какая масса переносится через площадку: $\MID{-\rho v'_x \ve v}d\sigma$ "--- это и будет увеличение количества движения на площадке. А если хочу вектор напряжений, надо поделить на $d\sigma$

\subsection{Проблема замыкания}
Мы усреднили уравнение неразрывности и уравнения движения. Имеем четыре уравнения на пятнадцать неизвестных. Нам нужно уравнение на $\MID{p_{ij}}$, оно считается известным. Если жидкость линейно-вязкая, то есть
\[
  P_{ij} = -p g_{ij} + \tau_{ij},\quad \tau _{ij} = 2\mu e_{ij}.
\]
то  $\MID{p_{ij}} = -\MID{p} g_{ij} + 2\mu\MID{e_{ij}}$, где $\MID{e_{ij}} = \frac12\left(\CP{\MID{v_i}}{x^j} + \CP{\MID{v_j}}{x^i}\right)$.

А что делать с тензором турбулентных напряжений непонятно. Пульсации могут быть самые разные для одной и той же среды. Они могут зависеть от граничных условий, от сил, которые действуют. Для этих $T^{ij}$ нельзя написать универсаьные уравнения.
