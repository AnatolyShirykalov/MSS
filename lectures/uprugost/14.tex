\section{Лекция 14}
\begin{enumerate}
\item  Добавление о свойствах системы уравнений линейной теории упругости (возможность суперпозиции решений).
\item Задача Ламе. Физическая и математическая постановки.
\item Уравнения и граничные условия в цилинтрической системе координат при наличии цилинтрической симметрии.
\item Решение задачи Ламе и его исследование.
\end{enumerate}
\subsection{О суперпозиции}
Так как уравнения и граничные условия в линейной теории упругости линейные и граничные условиям задаются на поверхности тела до деформации, то сумма решений для одного и того же тела при разных массовых и поверхностных силах будет тоже решением задачи, в которой массовые силы равны сумме массовых сил для разных задач и поверхностные силы есть сумма поверхностных сил для разных задач.
% рисунок

\subsection{Задача Ламе о напряжении и деформации в стенках трубы из упругого материала, которые возникают под действием внутреннего и внешнего давления.}
% рисунок: две вертикальные параллельные плоскости ограничивают два круговых цилидра с образующей, ортогональной плоскостям, порождённых концентрическими окружностями. $Z$ по общей оси цилиндров, длина $l$.

Физическая постановка задачи
\begin{enumerate}
\item Геометрия: круглая труба; радиусы сечений $a$, $b$, длина $l$.
\item Труба находится в равновесии.
\item Материал трубы линейно упругий изотропный, модули упругости постоянны.
\item Массовые силы не учитываются.
\item Давление внутри трубы $p_a=\const$, а вне трубы $p_b=\const$.
% рисунок: кольцо, радиальные стрелки: извне к внешней окружности с подписью $p_b$ и изнутри внутренней окружности с подписью $p_a$;
\item $T=T_0$, где $T_0$ "--- температура, при которой в отсутствие сил деформаций нет (то есть не учитываем возможные температурные напряжения и деформации).
\item Перемещения вдоль оси трубы отсутствуют. Например, труба бесконечно длинная, а мы рассматриваем кусок длины $l$ или торцы упираются в жёсткие гладкие (следующий пункт) плоскости, которые не дают частицам трубы перемещаться в продольном направлении, как на картинке нарисовано.
\item Торцевые плоскости не препятствуют перемещению частиц трубы в поперечном (оси цилиндров) направлении.
\end{enumerate}
Требуется найти напряжения в стенках трубы. Не привысят ли они критические значения, при которых трубы начнёт пластически деформироваться или появятся трещины.

Если слишком большое давление в трубе, трещины начнут появляться на внутренней стенке.

Имеет смысл всё писать в цилинтрической системе координат. Но мы запишем в инвариантом виде систему уравнений.
\subsection{Система уравнений в криволинейной системе координат}

Как мы писали уравнения Навье"--~Ламе в векторной форме
\[
  (\lambda+\mu)\grad\div\ve w + \mu\Delta\ve w = 0.
\]

Напишем выражения для $\div$, для $\grad$ и оператор $\Delta$.
\[
  \div\w = \nabla_i w^i = g^{ij}\nabla_iw_j.
\]
А про оператор Лапласа мы знаем в декартовых координатах
$
  \Delta w_i = \CP{^2w_i}{x^k\dl x^k} = \delta^{kl}\CP{^2w_i}{x^k\dl x^l}.
$
\[
  \Delta w_i = g^{kl}\nabla_k\nabla_l w_i.
\]
И для градиента частные производные скаляра совпадают с ковариантными. Итак, уравнения в компонентах имеют вид
\[
  (\lambda+\mu)\CP{ }{x^i} \div\ve w + \mu g^{kl}\nabla_k\nabla_l w_i = 0.
\]

Теперь граничные условия. Границы определяются из $r=a$, $r=b$, $z=0$, $z=l$.
% рисунок: 135-135-90 оси: x наверх, z вправо, у лево-низ; в области z\in[0,l] кольцевой цилиндр  с образующей $z$. Подпись длины $l$
\begin{itemize}
\item При $r=a$ $\ve P_n = -p_a\ve n$.
\item При $r=b$ $\ve P_n = -p_b\ve n$.
\item При $z=0$ и $z=l$ имеем $w_z=0$ и $\ve P_{n\tau} = 0$.  
\end{itemize}

После того, как найдём $w_i$  вычислим $\e_{ij}$ и $p_{ij}$, где
\[
 \e_{ij} = \frac12(\nabla_i w_j + \nabla_j w_i).
\]

Основная цель решения $p_{ij} = \lambda J_1(\e) g_{ij} + 2\mu\e_{ij}$.
\subsection{Цилиндрическая система координат}
Обозначения $x^1 = r$, $x^2 = \theta$, $x^3=z$.
% рисунок: 135-135-90 оси (x вверх); плоскость z=0; диск в плоскости z=z с центром 0,0,z;
% пунктирная вертикаль из центра диска наверх;
% точка на границе диска выше 0 левее пунктирной вертикали с подписью $M$; 
% отрезок из центра диска к M с подписью r внутри угла с пунктирной вертикалью
% подпись острого угла между радиус-вектором к M из центра диска и пунктирной вертикалью $\theta$
% отрезок от M к плоскости z=0 параллельный оси z с подписью $l$

% рисунок:
% окружность;
% горизонтальная ось $x$
% вправо-вверх 45 вектор $\ve э_1$ радиально из точки M границы круга;
% влево-вверх из той же точки 135 Э_2
% подпись точки $M$ выше
% отрезок из центра круга в $M$ с подписью $r$
% подпись угла от оси x к последнему отрезку $\theta$.

Имеем следующую информацию об этой системе координат
\begin{eqnarray*}
 (dr)^2 &=& ds^2 = dr^2 + r^2d\theta^2 + dz^2 = g_{ij} dx^i\,dx^j;\\
  (g_{ij}) &=& 
\begin{pmatrix}
1 & 0   & 0\\
0 & r^2 & 0\\
0 & 0   & 1
\end{pmatrix};\\
  (g^{ij}) = (g_{ij})^{-1} &=& 
\begin{pmatrix}
1 & 0   & 0\\
0 & \frac1{r^2} & 0\\
0 & 0   & 1
\end{pmatrix};\\
g_{ij} &=&  (\ve {\text{э}}_i\cdot \ve {\text{э}}_j);\pau |\ve{\text{э}}_i| = \sqrt{g_{ii}};\pau 
  |\ve{\text{э}}_1| = 1;\pau 
  |\ve{\text{э}}_2| = r;\pau 
  |\ve{\text{э}}_3| = 1.
\end{eqnarray*}
Теперь надо пересчитать все наши операторы
\begin{eqnarray*}
\nabla_i w_j&=& \CP{w_j}{x^i} w_k\Gamma^k_{ji};\\
\Gamma^k_{ij} &=&  \frac12 g^{ks}\left( 
  \CP{g_{si}}{x_j} +
  \CP{g_{sj}}{x_i} -
  \CP{g_{ji}}{x_k}
 \right);\\
 \Gamma^1_{22} = -r,&&
 \Gamma_{12}^2 = \Gamma_{21}^2 =\frac1r\pau \text{остальные }0;\\
  \nabla_i T_{kl} = \CP{T_{kl}}{x^i} - T_{\alpha l}\Gamma_{ki}^\alpha - T_{k\alpha}\Gamma^\alpha_{li};\\
  T &=&  T_{kl}
 \ve{\text{э}}^k
 \ve{\text{э}}^l.
\end{eqnarray*}
Все формулы в цилинтрической системе координат. Предположение один даёт $\ve w = w^1
 \ve{\text{э}}_1 + w_1
 \ve{\text{э}}^k$, $w_2=0$, $w_3=0$. При этом 
 $|\ve{\text{э}}^1| = 
 |\ve{\text{э}}_1| = 1$ и $
 \ve{\text{э}}^1
 \ve{\text{э}}_1$.

Предположение два даёт $w_1 = w(r)$. Таким образом,
\begin{eqnarray*}
  \e_{11} &=&  \nabla_1w_1 = \CP{w_1}{x^1} - \cancel{w_k \Gamma_{11}^k} = \CP wr;\\
  \e_{22} &=&  \nabla_2w_2 = \cancel{\CP{w_2}{x^2}} - w_k\Gamma^k_{22} = rw_1 = rw;\\
  \e_{33} &=&  \nabla_3w_3 = \cancel{\CP{w_3}{x^3}} - \cancel{w_k\Gamma_{33}^k} = 0;\\
  \e_{12} &=&  \frac12(\nabla_1 w_2 + \nabla_2 w_1) = \frac12\left( \cancel{\CP{w_2}{x^1}} -\cancel{w_2\Gamma^k_{21}} + \cancel{\CP{w_1}{x^2}} - \cancel{w_k\Gamma_{12}^k} \right);\\
  \e_{13} &=& \e_{33}=0;\pau \e_{11}\ne0\ne\e_{22}.
\end{eqnarray*}

Теперь
\[
  \div\ve w = J_1(\e) = g^{\alpha\beta}\e_{\alpha\beta}.
\]
В нашей задаче $\div\ve w = J_1(\e) = \CP wr + \frac wr$. Поэтому
\begin{eqnarray*}
  p_{11} &=&  \lambda\left( \CP wr + \frac wr \right) + 2\mu\CP wr;\\
  p_{23} &=&  r^2\lambda\left( \CP wr + \frac wr \right) + 2\mu r w;\\
  p_{33} &=&  \lambda\left( \CP wr + \frac wr \right).
\end{eqnarray*}

При $i\ne j$ имеем $p_{ij} = 2\mu\e_{ij}=0$.

Вычислим оператор Лапласа от $w_i$.
\[
  \Delta w_1 = g^{\alpha\beta}\nabla_\alpha\nabla_\beta w_1 = 
g^{11}\nabla_1\nabla_1 w_1 +
g^{22}\nabla_1\nabla_1 w_1 +
g^{33}\nabla_1\nabla_1 w_1
\]
От скаляра просто считать, а здесь компонента вектора
\[
  \nabla_1\nabla_1 w_1 = \nabla_1(\nabla_1 w_1) = \CP{ }{x^1}(\nabla_1 w_1) - 
  \cancel{\nabla_k w_1\Gamma^k_{11}} - 
  \cancel{\nabla_1 w_k\Gamma^k_{11}} = 
  \CP{ }r\left( \CP{w_1}r - w_k\Gamma_{11}^k \right) = \CP{ }r\CP{w_1}{r} = \CP{ }r\left( \CP wr \right).
\]

Теперь вычислим
\[
  \nabla_2(\nabla_2 w_1) = 
  \cancel{\CP{ }{x^2}(\nabla_2 w_1)} - 
  \nabla_k w_1\Gamma_{22}^k - 
  \nabla_2 w_k\Gamma_{12}^k = 
  (\nabla_1 w_1)\cdot r 
  - (\nabla_2 w_2)\frac1r = 
  r\CP wr - w_1.
\]
Ну а $\nabla_3(\nabla_3 w_1) = 0$.

Итак, оператор Лапласа
\[
  \Delta w_1 = \CP{ }r\left( \CP wr \right) +\frac1r\CP wr - \frac w{r^2} = 
  \CP{ }r\left( \CP wr + \frac wr \right).
\]

А уравнения Навье"--~Ламе
\begin{eqnarray*}
  (\lambda+\mu)\CP{ }r\left( \CP wr + \frac wr \right) + \mu\CP {}r\left( \CP wr + \frac wr \right)&=& 0;\\
  (\lambda+2\mu)\CP{ }r\left( \CP wr + \frac wr \right)&=& 0.
\end{eqnarray*}
Таким образом, $\CP wr + \frac wr = \const =: 2A$ "--- так обозначают эту постоянную.
\[
  wr = r^2 A + B;\qquad w = Ar + \frac Br.
\]
Это "--- решение уравнений Навье"--~Ламе при условии, что $w_1 = w(r)$, $w_2=w_3=0$ и равновесие и массовых сил.
