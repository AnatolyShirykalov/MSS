\section{6 ноября 2014}
Мы сегодня должны с вами лекцию 10 попробовать. Мы хотели заниматься граничными условиями. Мы их не разобрали, а они важны.

\begin{enumerate}
  \item Граничные условия на свободной поверхности в идеальной жидкости.
  \item Потенциальные течения несжимаемой идеальной жидкости. Уравнение Лапласа для потенциала скорости. Граничные условия.
  \item Задачи о нахождении потенциала скорости при отсутствии свободных поверхностей в качестве границ. Свойство линейности этих задач (важно именно это подчеркнуть).
  \item Примеры потенциальных течений незжимаемой жидкости.
\end{enumerate}

\subsection{Граничные условия на свободной поверхности в идеальной жидкости}
  Примеры твёрдых поверхностей: подводная лодка в воде, обтекание автомобиля, рыба-меч (скорость до 230 км/ч), примеры, касающиеся самолётов, аэродинамические трубы, обтекание зданий. Суть условий на твёрдой границы это жидкость не протекает через границу. Если жидкость протекает, поверхность называется пористой. Иногда специально отсасывают протекающую жидкость.

Второй тип поверхностей: форма и движение не заданы. Говорим мы про поверхности разрыва.
% рисунок 1
Разрыв чего? Скорости, плотности и ещё может быть чего. Свободный границы "--- это границы, через которые среда не перетекает. Все колебания связаны с тем, что сама жидкость ходит. Нет потока массы через границу. Движение этой границы нужно найти.

Примеры свободных поверхностей: поверхность волны (цунами, например, или корабельный волны), кавитация на торпеде Шквал,
% рисунок 2 (корабельные волны)
нефтепровод с неполным заполнением
% рисунок 3 (нефтепровод)
Если образуются волны, запирающие сечение, то течение может затормозиться.

Ещё пример это гидравлический прыжок. Справа идёт река, а слева начинается прилив из моря. И вода поступает из моря в реку и они сшибаются.
% рисунок 4 (гидр. прыжок)
 Прыжок распространяеится с довольно большой скоростью
% рисунок 5 (раковина)
Даже когда вы в раковине умываетесь, происхоит стоячий гидравлический прыжок.

Какие условия на свободный границе? Одного условия не хватает, нужно два. Одно называется кинематическое, другое "--- динамическое.
% рисунок 6 (
По одну сторону одна среда, по другую "--- другая. Условие сохранение массы говорит о том, что 
\[
  \ve v_n - D\big|_{\text{пов. }1} = (\ve n_2 - D); \ve v_n\big|_{\text{пов. разр.}} = \ve D_n.
\]
Это условие непроницаемости.

Кинематическое условие на свободной поверхности это как раз
\[ \ve v_n = \ve D_n,\]
где $\ve D_n$ "--- скорость поверхности.

Условие из закона сохранение количества движения
\[
  (\ve P_n)_1 = (\ve P_n)_2.
\]
Это называется динамическим граничным условием на свободной границе.
% рисунок 7

Тензон напряжений мы задать не можем. Это 9 компонент. Мы можем задать только вектор напряжений относительно нормали к границе.

Поскольку скорость границы мы не знаем, а нам хотелось бы форму границы вычислить. Поэтому делается так. Есть другая запись кинематического условия на свободной поверхности. Пусть уравнение свободной поверхность заранее неизвестно, но обозначим его $f(t,x,y,z)=0$. Условие, что $\ve v_n = \ve D_n$, означает, что 
% рисунок 8
частицы, которые находятся на поверхности, находятся на ней всегда, не могут с неё сойти (и по физике и по формальному), то есть для координат этих частиц всегда выполнено условие $f(t,x,y,z)=0$. Здесь уже $x,y,z$ "--- эйлеровы координаты точек частиц среды, лежащий на поверхности. Поэтому мы можем это соотношение продифференцировать по времени, считая, что $\DP xt = v_x$ и аналогично с $y,z$. Когда продифференцируем, получится кинематическое условие такое
\[
  \CP ft + v_x\CP fx+v_y\CP fy+v_z \CP fz = 0
\]
при $f(t,x,y,z)=0$. Или $\DP ft=0$ при $f(t,x,y,z)=0$.

Теперь это уравнение на $f$.

Пример. Если я рассматриваю волны на поверхности воды.
% рисунок 9 (волны на поверхности воды)
Есть дно и есть свободная поверхность. Уравнение поверхности будет такое $z = h(t,x,y)$, ну или можно написать $f = h(t,x,y) - z= 0$. Тогда кинематические условия пишутся вот так:
\[
  \CP ht + v_x \CP hx+v_y \CP hy - v_z = 0 \Leftarrow z=h(x,y,z).
\]
Или $v_z = \CP ht$ при $z = h(t,x,y)$ "--- кинематическое граничное условие на поверхности волны.

\subsubsection{Динамическое граничное условие в идеальной жидкости}
Теперь давайте динамическое условие, считая жидкость идеальной. Что мы делаем? Мы должны это $\ve P_n\big|_{\text{св. пов.}} = \ve P_n\big|_{\text{внеш.}} = \ve P_{na}$ (индекс ${}_a$ обозначает атмосферное) преобразовать с предположением $\ve P_n = -p\ve n$. Поэтому динамическое условие пишется, как
$ p = p_a$ на поверхности $f(t,x,y,z)= 0$.

В примере с волнами $p = p_a$ при $z = h(t,x,y)$.

Это условие можно переписать, как условие на скорость. Это можно сделать, например, если существует интеграл Коши"--~Лагранжа. Но пока я этого не буду делать.

Мы написали условия довольно сложные. Какая-то неизвестная скорость, всё нелинейное. Задачи со свободными границами неизмеримо сложнее чем задачи, где этих свободных границ нет.

Сейчас есть много пакетов для расчёта движения сред. Чтобы ими пользоваться, главное уметь правильно задать граничные условия. Тогда можно уже на кнопку нажать.

\subsection{Потенциальные течения несжимаемой идеальной жидкости. Уравнение Лапласа для потенциала скорости. Граничные условия}
Течения в океанах все вихревые. Плотность везде разная, всё достаточно трудно устроено. Поэтому сразу напижу, что жидкость несжимаемая.

Движение потенциально означает, что $\ve v = \grad\phi$ "--- уравнение потенциальности. Жидкость несжимаема, значит $\div\ve v = 0$ "--- уравнение неразрывности. Ну что это означает вместе взятое:
\[
  \CP{^2\phi}{x^2}+  \CP{^2\phi}{y^2}+  \CP{^2\phi}{z^2} = 0.
\]
Это уравнение называется уравнением Лапласа. А $\Delta = \CP{^2}{x^2}+\CP{^2}{y^2}+\CP{^2}{z^2}$ называется оператором Лапласа. Функция $\phi$, являющаяся решением, называется гармонической функцией.

Из этого уравнения можно найти $\phi$ и, соответственно, скорости. А другие величины уже будут находиться из других законов.

Теперь надо надо написать для этого уравнения Лапласа граничные условия. Тогда уже можно будет решать задачу. Коронной нашей задачей будет задача о движение сферы в нехжимаемой жидкости.
\subsubsection{Граничные условия для уравнения Лапласа в задачах о движении жидкости}
Сначала запишем условия на поверхности твёрдого тела.
% рисунок 10
\[
  \ve v_n\big|_{\text{пов. тв. тела}} = \ve v_{n\text{ тела}}.
\]

Если же $\ve v = \grad \phi$, то $v_x = \CP \phi x$ и~т.\,д. Тогда можно написать, что $\ve v_n = \CP\phi { n}$, где $\dl n$ "--- расстояние вдоль нормали. Я могу взять на минуточку ось $x$ в направлении нормали.
% рисунок 11
$ \CP\phi n = \yo {\Delta n}0 \frac{\phi(n_0+\Delta n) - \phi(n_0)}{\Delta n}$.

Просто вот такой вот факт $v_s = \CP\phi s$. Тогда условие на поверхности твёрдого тела в случае потенциального движения пишется так
\[
  \CP\phi n\bigg|_{\text{на пов. тела}} = v_n\big|_{\text{тела}}.
\]
Вот так задаётся ещё условие непроницаемости. Если нет землетрясений.
% рисунок 12
А что написать на свободной границе.

В Алма-Ате строили плотину методом взрыва. Ученики Лаврентьева расчитали, чтобы при взрыве все камни летели в одно место.

Давайте теперь напишем условие на свободной поверхности для потенциальной скорости. Можно ли их записать, как условие на $\phi$? Сначала кинематическое условие. Это легко
\[
  \CP ft + \CP\phi x\CP fx+\CP\phi y\CP fy+\CP\phi z\CP fz = 0 \Leftarrow f(t,x,y,z)=0.
\]
Неизвестные здесь уже теперь $\phi$ и $f$.

Теперь динамическое условие: $p\big|_{f=0} = p_a$. Можно записать, как условие на потенциал, если существует интеграл Коши"--~Лагранжа. А он существует? жидкость идеальна, движение баротропно, движение потенциальное, а потенциал массовых сил тоже существует, иначе бы движение тоже не было  бы потенциальным. Значит, интеграл существует. Осталось понять, как именно будет выглядеть динамическое условие на свободной границе.

Интеграл Коши"--~Лагранжа выполнен везде в жидкости:
\[
  \CP\phi t +\frac{v^2}2 + \frac p\rho-W= \frac{p_a}{\rho}.
\]
Справа вообще может стоять любая функция времени, поставим удобную нам константу. Это влияет на вид зависимости потенциала от времени, но не влияет на скорость. $\frac{p_a}{\rho}$ пишут тут довольно часто.

Тогда на поверхности $f=0$ условие $p=p_a$ и интеграл Коши"--~Лагранжа вместе дают
\[
  \CP\phi t + \frac12 \left[\left(\CP\phi x\right)^2+\left(\CP\phi y\right)^2+\left(\CP\phi z\right)^2\right] - W = 0\Leftarrow f=0.
\]
Это условие очень сложное. Поэтому книги по задачам о движении волн очень большие и сложные. Исследовать это дело очень сложно.

Задачи о схлопывании пузырьков (о разрушениях, ими наносимых), задачи о взрывах, задачи о струях.
\subsection{Задачи о потенциальном движении несжимаемой жидкости в случае отсутствия свободных поверхностей}
Будем считать, что
\begin{enumerate}
  \item Область, где движется жидкость, ограничена.
% рисунок 13
Бак в ракете. В Баке находится какое-то тело. Тогда уравнение $\Delta \phi = 0$, если на граниее заданы условия $\CP\phi n\big|_{\text{на гр.}}=0$, называется внутренней задачей Неймана.

Если же задана сама функция $\phi$, то имеем задачу Дирихле.

Уравнение Лапласа линейно и граничные условия тоже линейны. Значит, внутрення задача Неймана линейна. Раз задача линейная, то для неё есть много хороших методов решения.
  \item Область неограничена, то есть содержит бесконечно удалённую точку. Это идеализация, но оказывается, для этого уравнения её можно сделать.

Здесь задачи бывают двух типов. Например
\begin{enumerate}
  \item Тело движется в безграничной покоящейся жидкости. (Внешняя задача Неймана.)
% рисунок 14
Если нет земли, сплошной воздух, летит самолёт. Если я отойду от тела на расстояние порядка нескольких линейных размеров, то можно считать, что точка бесконечно удалена.

Итак задача: $\Delta \phi = 0$ всюду вне тела, а на границе тела меем $\CP \phi n\big|_{\text{на пов. тела}} = v_{n\text{ тела}}$, и на бесконечности задана скорость $\ve v=0$, то есть $\grad\phi\big|_\infty =0$. (Одно добавочное условие появилось, градиент на бесконечности.)
  \item Ещё бывает задача, где тело обтекается потоком, но это то же самое. Например, исследования в аэродинамической трубе. Или просто здания обтекаются потоком.
% рисунок 15
В этом случае на поверхности тела $\CP \phi n\big|_{\text{на пов. тела}} = 0$ и ещё $\grad\phi\big|_\infty = \ve v_\infty$ задано. Это тоже называется внешней задачей Неймана.
\end{enumerate}
\end{enumerate}
В общем, если нет свободный границ, то задача линейна.
\subsection{Примеры потенциальных течений несжимаемой жидкости}
Из сложения этих течений я буду потом получать более сложные.
\begin{enumerate}
  \item Поступательное движени вдоль оси $x$. Здесь $\phi = v_0 x$, $\CP \phi y = \CP\phi z = 0$, $\Delta \phi = 0$.
  \item Источник или сток в начале координат. Что это такое. Потенциал $\phi = - Q{4\pi r}$, где $r = \sqrt{x^2+y^2+z^2}$. Хочется это разобрать, как устроено течение, удовлетворяет ли это течение уравнению Лапласы.

Как поступим? Можно ввести наукообразие, ввести сферические координаты. Но здесь всё достаточно просто, будем работать в декартовых.
\[
  v_x = \frac{Q}{4\pi r^2}\frac{x}r = \frac{Qx}{4\pi r^3},\quad v_y = \frac Q{4\pi r^2}\frac yr,\quad v_z = \frac Q{4\pi r^2}\frac zr.
\]
Или, векторно
\[
  \ve v = \frac{Q}{4\pi r^2}\frac{\ve r}{r}.
\]
% рисунок 16 (или два рисунка, источник и сток)
Если $Q>0$, источник; $Q<0$ "--- сток.
\end{enumerate}
